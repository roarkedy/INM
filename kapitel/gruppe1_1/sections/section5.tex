\section{Anwendung des Informationsmanagements am Beispiel von Hochschulen - MiB}
\label{anwendung_des_inm_auf_hs}
Mit der praktischen Anwendung der bisher erörterten Erkenntnisse zum Thema 
Informationsmanagement an Hochschulen befasst sich das nachfolgende Kapitel. 
Hierbei wird insbesondere analysiert, inwieweit der Bereich des Immatrikulations- und 
Prüfungsamtes als zentraler Knotenpunkt in der Informationsverteilung dienen kann und 
welche Auswirkungen sich für die Bibliothek und die Organisation von Rechnerpools ergeben 
können, um ein ganzheitliches Informationsmanagement führen zu können.

\subsection{Immatrikulations- und Prüfungsamt}
\label{immatrikulations_und_pruefungsamt}
In Hochschulen, bei denen ein Informationsmanagement Anwendung findet, bildet das Immatrikulations- und Prüfungsamt eine Art interne Informationszentrale, die weitere Bereiche mit notwendigen Informationen versorgt. 

Betrachtet an einem Beispiel bedeutet dies Folgendes: Bei Immatrikulation eines neuen Studierenden wird diesem vom Immatrikulationsamt eine Matrikelnummer zugewiesen und seine Stammdaten ins HIS (vergleiche hierzu Abschnitt \ref{paragraph_trends_his}) eingepflegt. 

Nun ist es Aufgabe des Immatrikulationsamtes, das HIS zu einer Art Schnittstelle für alle wichtigen Hochschulbereiche, wie z.B. die Bibliothek, die Mensa oder auch die Verwaltung von Computerräumen, zu machen, sodass diese Bereiche via Eingabe der Matrikelnummer auf für sie wichtige Studierendendaten zugreifen können.

Um den Datenschutz der Studierenden zu garantieren, wäre hierfür eine Lösung mittels individueller Rechtezuweisung für jeden Bereich denkbar.

Der Datensatz im HIS wäre nicht nur zentral für alle Hochschulbereiche verfügbar, sondern 
auch jederzeit auf aktuellstem Stand, sodass Redundanzen ausgeschlossen werden können. 
Zur Minimierung des Verwaltungsaufwandes, wäre es denkbar, bei Stammdatenänderung 
durch das Immatrikulationsamt eine automatisch generierte E-Mail an alle beteiligten 
Bereiche mit den aktualisierten Informationen über den Studierenden zu versenden, was 
einem ganzheitlichen Informationsmanagement entsprechen würde.

Auch nach außen hin stellt das HIS eine zentrale Anlaufstelle für alle wichtigen 
Informationen wie Raumpläne, Kontaktdaten der Lehrenden und Prüfungsmodalitäten dar. Bei 
Ausfall einer Veranstaltung beispielsweise kann dieses dort direkt publik gemacht werden oder nach Prüfungsanmeldung der Studierenden kann im HIS kann schnell und komfortabel aus den Anmeldedaten ein zentraler Raumbelegungsplan erzeugt werden.
\footcite{evalag_eckpunkte_2012}

Bei der Notenvergabe meldet der Prüfer die Noten der Studierenden an das Prüfungsamt, das 
diese in das HIS einpflegt. Die Studierenden haben nun die Möglichkeit zentral ihre Noten 
abzurufen. Auch die Fachbereiche, die über die Leistungen ihrer Studierenden informiert 
werden sollten, können auf diese Daten zugreifen.

Die Sammlung und Bereitstellung an zentraler Stelle wie dem HIS minimiert 
Abstimmungsmodalitäten zwischen den verschiedenen Hochschulbereichen, reduziert den 
Arbeitsaufwand für die erneute Erfassung und Verwaltung der Studierendendaten in dem 
jeweiligen Bereich und garantiert einen stets konsistenten Datensatz.


\subsection{Bibliotheken}
Hochschulbibliotheken werden tagtäglich mit einer Menge an Informationen und Daten 
konfrontiert. Von deren Besitz eines EDV-Systems zur Erfassung der Ausleihe inklusive Ablauf 
der Fristen und Stammdaten des Studierenden kann an dieser Stelle ausgegangen werden, da 
die Grundfunktionalität des Bibliothekensystems ansonsten kaum gewährleistet wäre. Als 
weitere Basisfunktion sei die Autorisierung der Studierenden zu nennen. Bei der Ausleihe 
wird in Hochschulbibliotheken über das System geprüft, ob dieser Studierende durch 
Immatrikulation dazu berechtigt ist, an dieser Hochschule Bücher auszuleihen. 
Im Zuge eines angewandten Informationsmanagements wäre es von Vorteil, die Stammdaten der Studierenden direkt aus dem HIS auszulesen. 

Aufbauend auf dieses Grundsystem existieren Lösungen, die das Bibliothekswesen mittels 
Informationsvermittlung, -speicherung und -auswertung für zahlreiche 
Einsatzmög-lichkeiten bereichert. Jede Hochschule sollte sich etwas Zeit nehmen, sich mit 
einer EDV-Lösung zu befassen, die neben der elektronischen Erschließung der 
Ausleihfaktoren auch Werkzeuge zur statistischen Erfassung, Messung und Bewertung der 
Bestandsentwicklung und des Leihverhaltens bietet. Aus diesen statistischen Daten können 
Rückschlüsse auf das Verhalten der Studenten gezogen und wichtige Erkenntnisse für den 
weiteren Bestandsaufbau gezogen werden.\footcite[9 ff.]{merkle_aufbau_2004}

Je nach Größe der Bibliothek ist es sinnvoll, sich grundlegend Gedanken darüber zu machen, 
welche Mitarbeiter für die Medienbestellung zuständig sind und wer die 
Entscheidungskompetenz hierfür besitzt. Eine kontinuierliche Abstimmung optimalerweise 
mittels zentralem Verwaltungssystem untereinander ist unumgänglich, um 
Doppelbestellungen zu vermeiden und das Budget möglichst gewinnbringend für die 
Studierenden einzusetzen. 

Die Mitarbeiter, die für die Medienbestellungen zuständig sind, sollten sich stetig auf dem 
Laufenden halten, welche Neuerungen es auf dem Büchermarkt gibt, um diese Werke 
möglichst aktuell in den Bestand aufnehmen zu können und den Studierenden eine aktuelle 
Ausleihe zu garantieren. Die Bibliotheksleitung könnte über Kooperationen mit anderen 
Hochschulen zum Austausch von Neuerungen oder auch zum Tausch von Dubletten 
nachdenken, um dem Gesamtkonzept eines gelebten Informationsmanagements gerecht zu 
werden. 

Die Studierenden könnten via Newsletter oder Website der Bibliothek darüber informiert werden, welche Neuerungen in den Bücherbestand aufgenommen wurden. 

Ab einer gewissen Bibliotheksgröße könnte auch ein www-Online-Katalog angedacht werden, der das Repertoire der Bibliothek abbildet und wichtige Informationen nach außen trägt.

Ohne diese zentralen Informationsplattformen wäre ein Informationsmanagement an der Hochschule überflüssig.


\subsection{Rechnerpools}
Die Organisation der Nutzung von Rechnerpools zieht ohne zentrales 
Informationsmanagement einige Probleme nach sich. Doppelbelegungen und unnötig 
leerstehende Computerräume sind die Folge eines fehlenden zentralen Belegungssystems.

Das bereits erläuterte HIS könnte um genau diese Funktion erweitert werden. Die Lehrenden 
können sich im HIS einen Computerraum für ihre Lehrveranstaltungen verbindlich 
reservieren und bei Ausfall der Veranstaltung wieder für die Allgemeinheit freigeben. Da die 
Raumbelegung an zentraler Stelle geschieht, ist auch hier der klare Vorteil, dass der Plan 
jederzeit auf aktuellem Stand ist und von jedem Lehrenden oder Studierenden eingesehen 
werden kann, was Verzögerungen, die bei der Suche eines geeigneten Computerraums auf 
herkömmlichem Wege, eliminiert. 
