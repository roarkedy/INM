\section{Ziele des Informationsmanagements}
Das Informationsmanagement verfolgt zwei grundlegende Zielsetzungen. Das erste Ziel ist die Koordination der Informationslogistik bzw. die Gewährleistung der adressatengerechten Informationsversorgung. Das zweite Ziel ist die Unterstützung der Unternehmensziele durch eine zielgerichtete und wirtschaftliche Steuerung der Informatik.

\subsection{Koordination der Informationslogistik}
In erster Linie ist das Ziel des Informationsmanagements, tatsächlich relevante Information von der Menge an verfügbaren und eventuell unnützen Informationen zu trennen, die für einen Entscheidungsprozess benötigt werden. Hierzu muss jedoch erst einmal ein Informationsbedarf vorliegen, der die Art, Menge und Beschaffenheit der Informationen bestimmt und auf dessen Grundlage eine Entscheidung getroffen werden kann.\\

Die Definition des Informationsbedarfs hängt einerseits vom Entscheider, andererseits von den Anforderungen der zu treffenden Entscheidung ab.
Der Informationsbedarf lässt sich grundsätzlich in zwei Kategorien einteilen: in den objektiven und den subjektiven Informationsbedarf.\\

Der \textbf{objektive} Informationsbedarf wird in erster Linie durch die Entscheidung festgelegt und baut auf der Aufgabenbeschreibung des Entscheiders und den jeweiligen Marktgegebenheiten auf.
Der \textbf{subjektive} Informationsbedarf wird primär durch den Entscheider festgelegt. Welche Informationen für die Entscheidung relevant sind, werden durch die Einschätzungen und Präferenzen des Entscheiders mitbestimmt.\\

Aus der Überschneidung des objektiven und subjektiven Informationsbedarfs entsteht die Informationsnachfrage, die wiederum maßgeblich vom Informationsangebot abhängt. Somit legt der Informationsbedarf

\begin{itemize}
	 \item die Beschaffenheit (Qualität),
	 \item den Zeitpunkt der Lieferung,
	 \item den Ort, an dem geliefert wird und
	 \item das Medium, über das geliefert wird		 
\end{itemize}
in Bezug auf die Information fest. Im Hinblick auf die Unternehmensziele sollten die Informationen als Ressource angesehen werden. 

\subsection{Informationsmanagement als Unterstützung der Unternehmensziele}
Das Informationsmanagement bildet einen Teil der Unternehmensführung ab, der die Steuerung der Informatik (d.h. Mitarbeiter, Prozesse, organisatorische Teilbereiche und die eingesetzten Informationstechnologien) zur Verantwortung hat. Diese Informatik und deren Leistungen sollte dabei auf die Unternehmensziele ausgerichtet sein.\\

\emph{Muss noch weiter ausgearbeitet werden}