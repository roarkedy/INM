% This file was converted to LaTeX by Writer2LaTeX ver. 1.4
% see http://writer2latex.sourceforge.net for more info
\documentclass{article}
\usepackage[utf8]{inputenc}
\usepackage[T1]{fontenc}
\usepackage[ngerman]{babel}
\usepackage{amsmath}
\usepackage{amssymb,amsfonts,textcomp}
\usepackage{array}
\usepackage{supertabular}
\usepackage{hhline}
\usepackage{caption}
\usepackage[pdftex]{graphicx}
\makeatletter
\newcommand\arraybslash{\let\\\@arraycr}
\makeatother
\setlength\tabcolsep{1mm}
\renewcommand\arraystretch{1.3}
\title{}
\begin{document}
\section[4\ \ Konzept zur Erreichung der Sollsituation]{4\ \ Konzept zur Erreichung der Sollsituation}
\subsection[4.1\ \ Umsetzungsplanung]{4.1\ \ Umsetzungsplanung}
\subsection[4.1.1\ \ Positionsbestimmung]{4.1.1\ \ Positionsbestimmung}
Für einen erfolgreichen Umsetzungsplan mit der Zielsetzung einer Neuordnung des Informationsmanagements an einer
Hochschule ist eine Positionsbestimmung der aktuellen Situation von elementarer Bedeutung. Hierzu muss der Ist-Zustand
des aktuellen Informationsmanagements mit der Zielformulierung des avisierten Informationsmanagements an der Hochschule
erfasst und abgeglichen werden. 


\bigskip

Da solche Veränderungen in der Regel einen langwierigen Prozess darstellen, ist es ratsam, Prioritäten zu definieren und
die einzelnen Teilbereiche anhand der Dringlichkeit umzusetzen. 


\bigskip

Ist die Position bestimmt, kann davon ausgehend ein entsprechender Migrationsplan (vgl. Kapitel 4.1.3) und, wenn noch
nicht geschehen, ein Changeplan (vgl. Kapitel 4.1.2) erstellt werden. Je nach Art und Umfang der Veränderungen sollte
allerdings das Change Management nicht erst nach der Positionsbestimmung angewandt werden, sondern schon bei der
Zielbestimmung – also mit in die Erarbeitung des möglichen Soll-Zustands einfliessen. 


\bigskip

Diese Ausarbeitung wird sich aus Gründen der Komplexität im praxisbezogenen Teil nicht auf das gesamte
Informationsmanagement der Hochschule Emden/Leer beziehen können. Exemplarisch wird daher eine Umsetzungsplanung an den
Beispielen des Dokumentenmanagements Alfresco und der Erstellung eines responsive Designs der Webpräsenz der Hochschule
erarbeitet.


\bigskip


\bigskip

Alfresco (vgl. Kapitel 4.1.4.2) wird derzeit noch nicht an der Hochschule eingesetzt. Zur Zeit werden Dokumente in
verschieden Systemen verwaltet und zugäglich gemacht. Für die Webpräsenz wird derzeit ein TYPO3 (vgl. Kapitel 4.1.4.1)
-System in der Version 4.5 LTS (Long Term Support) genutzt, welches noch nicht für mobile Endgeräte optimiert ist. 


\bigskip

4.1.2\ \ Change Management

\subsection[4.1.2.1\ \ Grundlagen des Change Managements]{4.1.2.1\ \ Grundlagen des Change Managements}
Die Umsetzung einer Neuordnung des Informationsmanagements an einer Hochschule bedeutet auch Wandel und Veränderungen.
Um das optimal zu steuern, bedarf es spezieller Managementtechniken, welche sich unter dem Begriff Change Management
zusammenfassen lassen. Im Vordergrund aller Betrachtungen steht der Faktor Mensch, denn für eine erfolgreiche Umsetzung
von Veränderungen ist die aktive Unterstützung der Betroffenen von erheblicher Bedeutung\footnote{\ Lauer, Thomas.
2014. Change Management.}.


\bigskip

Grundlegend können soziale Veränderung in einem 3 Phasen Modell abgebildet werden:



\begin{center}
\includegraphics[width=11.098cm,height=1.372cm]{INM41HalfmannChristianAbgabe150531-img/INM41HalfmannChristianAbgabe150531-img001.png}
\captionof{figure}[3{}-Phasen Modell nach Kurt Lewin]{3-Phasen Modell nach Kurt Lewin}

\end{center}
\begin{enumerate}
\item Die Betroffenen sollen für Veränderungen motiviert werden.
\item Die Betroffenen sollen für Veränderungen trainiert werden und der Veränderungsprozess vollzogen werden. \ 
\item Die die Veränderungen sollen Stabilisiert werden. 
\end{enumerate}

\bigskip

Nach Thomas Lauer sollte das Change Management grundsätzlich an drei Punkten ansetzen:


\bigskip

\begin{itemize}
\item Individuum:

Das Individuum beschreibt jeden Einzelnen. Ohne die Mitarbeit der Betroffenen ist ein Wandel unmöglich. Das Change
Management soll nicht nur die Fähigkeiten des Einzelnen an neue Herausforderungen anpassen, sondern auch die positive
Einstellung gegenüber der Ziele des Wandels aller Betroffenen fördern. \ 
\end{itemize}

\bigskip

\begin{itemize}
\item Unternehmensstruktur (bzw. Hochschulstruktur)

Die Unternehmensstruktur umfasst Aufbau- und Ablauforganisation sowie Strategien und Ressourcen. Veränderungen in diesen
Bereichen sind auf dem Papier grundsätzlich einfach.
\end{itemize}

\bigskip


\bigskip

\begin{itemize}
\item Unternehmenskultur (bzw. Hochschulkultur) 

Die Unternehmenskultur beschreibt dauerhafte, über lange Zeit gewachsene Strukturen die für Einstellung, Werte und
Regeln des Umgangs verantwortlich sind. 


\bigskip
\end{itemize}
In den meisten Fällen bringt ein Wandel in den oben genannten Bereichen Veränderungen in allen Dimensionen mit sich, die
sich wechselseitig beeinflussen\footnote{\ Sonntag, Karlheinz; Stegmaier Ralf; Michel, Alexandra. (2008). Change
Management an Hochschulen: Konzepte, Tools und Erfahrungen bei der Umsetzung}. 

So ist z. B. ein Wandel ohne die Einbeziehung der Unternehmenskultur oftmals zum scheitern verurteilt\footnote{\ Lauer,
Thomas. 2014. Change Management.}. Das heißt, für ein erfolgreiches Change Management sollten grundsätzlich die
Abhängigkeiten der Bereiche untereinander berücksichtigt werden.


\bigskip

Veränderungen bedeuten Neues und Ungewohntes für alle Betroffenen. 

Betroffene müssen veränderte Aufgaben erledigen, neue Technologien und Methoden erlernen, erneut soziale Beziehungen zu
Kollegen, Vorgesetzten oder Kunden aufbauen, mit Problemen in der Implementierungsphase umgehen und ggf. ihre Werte in
Einklang mit neuen Standards und Zielen der Organisation bringen\footnote{\ Sonntag, Karlheinz; Stegmaier Ralf; Michel,
Alexandra. (2008). Change Management an Hochschulen: Konzepte, Tools und Erfahrungen bei der Umsetzung}. Dies kann zu
Zweifeln und Widerständen führen, was im schlimmsten Fall ein Scheitern des gesamten Vorhabens bedeuten kann. Als
Ursache eines gescheiterten Wandels steht der Widerstand an oberster Stelle. Mangelhafte Prozessteuerung, zu schnelles
Veränderungstempo und unklare Zielsetzungen spielen dabei ein wichtige Rolle und können Gründe für eben diesen
Widerstand sein\footnote{\ Lauer, Thomas. 2014. Change Management.}. 


\bigskip

Die Bereitschaft zum Wandel nimmt zu, wenn die Betroffenen überzeugt sind, das die Veränderungen ihnen persönlich
nutzen, ihre Identität nicht bedroht ist und ihre Werte und Ziele mit dem Wandel in Einklang gebracht werden können.
Des weiteren wird die Bereitschaft zum Wandel gefördert wenn die Betroffenen über Fähigkeiten verfügen, die den
veränderten Anforderungen gerecht werden. Die Aufgabe des Change Management ist es, durch Information, Partizipation,
Unterstützung (z. B. Coaching) und Anreizgestaltung den Betroffenen die Zweifel und Unsicherheit zu
nehmen\footnote{\ Sonntag, Karlheinz; Stegmaier Ralf; Michel, Alexandra. (2008). Change Management an Hochschulen:
Konzepte, Tools und Erfahrungen bei der Umsetzung}. 


\bigskip

\begin{itemize}
\item Kommunikation:

Nach Thomas Lauer ist einer der entscheidendsten Erfolgsfaktoren des Change Managements die Kommunikation. Kommunikation
schafft Transparenz und damit Orientierung und dient damit auch der Beilegung von Widerständen. Damit ist aber auch ein
potenzieller Misserfolg eines Change Managements auf die Kommunikation zurückzuführen. Fehlinterpretationen und
Missverstände können schnell zu Konflikten führen\footnote{\ Lauer, Thomas. 2014. Change Management.}.


\bigskip

Es sollten also entsprechende Kommunikationsstrategien und Kommunikationspläne erarbeitet werden, um die Betroffenen für
die Veränderungen zu gewinnen und Missverständnissen aus dem Weg zu gehen. In der Startphase sollten die Betroffenen
über Gründe, Ziele, Notwendigkeit, Nutzen und den zeitlichen Ablauf informiert werden. Aber auch potentielle Risiken
und Schwierigkeiten sollten von Anfang an offen kommuniziert werden\footnote{\ Sonntag, Karlheinz; Stegmaier Ralf;
Michel, Alexandra. (2008). Change Management an Hochschulen: Konzepte, Tools und Erfahrungen bei der Umsetzung}. In der
Durchführungsphase ist es wichtig die Kommunikation aufrecht zu erhalten. Beispielsweise können Projektfortschritte
regelmäßig an alle Betroffenen weitergegeben werden. Durch das Aufrechterhalten der Kommunikation können frühzeitig
Widerstände erkannt und überwunden werden\footnote{\ Lauer, Thomas. 2014. Change Management.}. 


\bigskip
\item Partizipation:

Ein weiterer wichtiger Erfolgsfaktor ist die Partizipation. Das Einbinden möglichst vieler Betroffenen in den Change
Prozess erhöht die Motivation und hilft den Betroffenen sich mit den Veränderungen zu identifizieren. Haben Betroffene
andere Positionen oder Sichtweisen gegenüber des Wandels als die Organisation müssen diese Widerstände nicht gleich
negativ ausgelegt werden. Die Ideen und Vorschläge der Betroffenen können in den Change Prozess einfliessen und
weiterhin Veränderungen optimieren. 


\bigskip
\item Unterstützung:

Besonders wenn es um neue Technologien, Werkzeuge oder Verfahren geht, ist Unterstützung für die Betroffenen gefordert.
Die Unterstützung hat zum Ziel, die Betroffenen des Wandels auf die zusätzlichen oder neuen Anforderungen
vorzubereiten. In den Meisten Fällen geschieht das in Form von Weiterbildung oder Coaching. 
\end{itemize}

\bigskip


\bigskip

\begin{itemize}
\item[] Des weiteren fördert eine vom Unternehmen ausgehende Weiterbildung nicht nur den Aufbau von Qualifikationen und
die Erweiterung des Wissens der Betroffenen, sondern auch die Motivation. Den Betroffenen wird das Gefühl gegeben, dass
in sie investiert wird und damit auf eine langfristige Partnerschaft gesetzt wird.
\end{itemize}

\bigskip

In dem Sinne ist die Aufgabe des Change Managements also nicht die Definition von Zielen, es soll den Weg des gesamten
Vorhabens vom Ausgangspunkt bis zum Ziel gestallten\footnote{\ Lauer, Thomas. 2014. Change Management.} und den
Betroffenen des Wandels ihre Zweifel nehmen. 


\bigskip

Auch bei perfekt geplanten Change-Projekten können Widerstände nicht ausgeschlossen werden. Das Change Management sollte
in der Lage sein auf diese Widerstände zu reagieren. Sollte sich in dem laufenden Change Prozess herausstellen, dass
bestimmte Bedingungen nicht mehr aktuell sind, sollten Ziele und Veränderungen angepasst und neu formuliert
werden\footnote{\ Sonntag, Karlheinz; Stegmaier Ralf; Michel, Alexandra. (2008). Change Management an Hochschulen:
Konzepte, Tools und Erfahrungen bei der Umsetzung}. \ 


\bigskip

Der Wandel kann nur gelingen wenn die Betroffenen hinter den Plänen stehen und die Veränderungen unterstützen. Im Fall
der Hochschule treten einige Besonderheiten auf, auf welche im nächsten Kapitel genauer eingegangen wird. \ 

\subsection[4.1.2.2 \ \ Change Management an Hochschulen]{4.1.2.2 \ \ Change Management an Hochschulen}
Im vorherigen Kapitel wurden die Adressaten des Change Managements als Betroffene betitelt. Diese sind im klassischen
Fall Mitarbeiter eines Unternehmens, in dem Veränderungen vorangetrieben werden sollen. Diese Mitarbeiter sind \ meist
Bestandteil einer klaren Hierarchie, an dessen oberster Stelle das Management steht, von welchem der Wandel initiiert
wird. 


\bigskip

Im speziellen Fall von Hochschulen setzen sich die Betroffenen aus Professoren, wissenschaftlichen Mitarbeitern,
Verwaltungsmitarbeitern und Studierenden zusammen, welche autonome Endscheidungen treffen. Studenten entscheiden, was
sie lernen, Dozenten entscheiden welche Inhalte sie lehren\footnote{\ Langenbeck, Ute; Suchanek, Justine; Höschler,
Barbara. (2011). Change Managment an Hochschulen}.


\bigskip

Hinzu kommen Fakultäten, Fachbereiche und Institute, welche sich selbst organisieren und nahezu autonom und unabhängig
von einander agieren\footnote{\ Sonntag, Karlheinz; Stegmaier Ralf; Michel, Alexandra. (2008). Change Management an
Hochschulen: Konzepte, Tools und Erfahrungen bei der Umsetzung}. 

Dies erschwert die Kommunikation untereinander sowie das Erschaffen von Synergien und das Entwickeln übergeordneter
Ziele und Strategien. \ \ \ 


\bigskip

Ein erfolgreiches Change Management muss also die besonderen Gegebenheiten der Organisation Hochschule bei der
Gestaltung und Auswahl entsprechender Maßnahmen berücksichtigen und auf sie eingehen. 


\bigskip

Hierzu sollten die Betroffenen innerhalb der Hochschule frühzeitig in die Zielformulierung von Change Prozessen
eingebunden werden. So kann Raum für Diskussionen geschaffen werden, denn die unterschiedlichen Bereiche der Hochschule
vertreten oft unterschiedliche Interessen was zur Verschleppung oder Verzögerungen von Entscheidungen führen kann. In
solchen Fällen kann ein Austausch mit externen Experten oder internen Stäben helfen, Entscheidungen
voranzutreiben\footnote{\ Sonntag, Karlheinz; Stegmaier Ralf; Michel, Alexandra. (2008). Change Management an
Hochschulen: Konzepte, Tools und Erfahrungen bei der Umsetzung}.

Studien zu Change Management an Hochschulen haben herausgefunden, dass auch hier Information und Partizipation wichtige
Elemente des Change Managements sind:


\bigskip

\begin{itemize}
\item Eine Studie zur Evaluation der Strategieumsetzung an der Universität Heidelberg könnte belegen, dass Partizipation
und die Qualität der Information positive Auswirkungen gegenüber Veränderungen bei den wissenschaftlichen Mitarbeitern
und Studierenden hatte. Je besser die Betroffenen über die Ziele der Veränderungen informiert wurden und desto mehr
eigene Ideen sie in die Veränderungen einbringen konnten, desto eher wurde der Wandel positiv bewertet und die
Bereitschaft gesteigert, aktiv an der Umsetzung mitzuwirken\footnote{}.
\end{itemize}

\bigskip

\begin{itemize}
\item Bei Veränderung des Curriculums und der Einführung neuer Prozesse und Strukturen des Qualitätsmanagements an einem
amerikanischen Collage zeigte sich, dass es nicht nur darum geht, Partizipation zu erhöhen, sondern auch darum,
Lehrende so anzuleiten, dass Entscheidungen nicht zu autoritär getroffen werden, noch durch zu starke
Gleichberechtigung in die Länge gezogen oder gar verhindert werden\footnote{\ Cohen, A. R.; Fetters, M.; Fleischmann,
F. (2005). Major change at Babson \ \ College: Curricular and administrative, planned and otherwise. Advances in
\ \ Developing Human Resources.}. 


\bigskip
\item Eine weitere Studie zur Implementierung von E-Learning konnte belegen, dass integratives Change Management
erforderlich ist, um Veränderungen nachhaltig zu implementieren. Wurden Maßnahmen wie z. B. Training, Beratung oder
didaktische Szenarien aufeinander abgestimmt, wirkt sich das positiv auf die Nutzung von E-Learning
aus\footnote{\ Fuchs, M. (2007). Change Management an Hochschulen. Die strategische Integration von
Bildungsinnovationen. \& Sch\"{o}nwald, I. (2007). Change Management in Hochschulen – Die Gestaltung soziokultureller
Ver\"{a}nderungsprozesse zur Integration von E-Learning in die Hochschullehre.}. 
\end{itemize}

\bigskip

Aus den Studien wird ersichtlich, dass Information und Partizipation wichtige Element des Change Managements darstellen.
Aber auch Schulungen, Trainings, oder Coaching spielen eine große Rolle. Allerdings kann es zur Herausforderung werden,
potenzielle Teilnehmer für Weiterbildungen aus dem Kreise der Professoren oder der Hochschulführung zu gewinnen, da
diese auf ihrem Fachgebiet als Experten gelten und eine Teilnahme an solchen Weiterbildungsangeboten als Ausdruck
persönlicher Defizite werten könnten. Dennoch bietet es sich an, bei komplexen Veränderungen zusätzliche Kompetenzen
durch Training oder Coaching zu erschließen\footnote{\ Fisch, Rudolf; Müller, Andrea (Hrsg.). 2008. Veränderungen in
Organisationen.}. 


\bigskip

Durch die besonderen Strukturen und Gegebenheiten einer Hochschule, muss \ ein potentielles Change Management möglichst
sensibel agieren und alle relevanten Akteure informieren und partizipieren lassen, um am Ende auch den gewünschten
Erfolg und somit die avisierten Ziele zu erreichen.


\bigskip

4.1.2.3 \ \ Changeplan

Wie Eingangs erwähnt, kann hier nicht auf das gesamte Informationsmanagement der Hochschule eingegangen werden. Daher
wird der Changeplan sich exemplarisch auf die Beispiele Alfresco und Responsive Design beziehen, wobei es sich in
beiden Fällen um Veränderungsprozesse im IT-Bereich handelt. \ \ 


\bigskip

Der Changeplan soll unter Betrachtung der beschrieben Grundlagen des Change Managements sowie der besonderen
Rahmenbedingungen an Hochschulen erstellt werden. 


\bigskip

Hinzu kommt, dass es sich hier um Veränderungsprozesse im IT-Bereich handelt. Hier gibt es in der Praxis zwei
Herangehensweisen an das Change Management. Zum einen die deterministische Sichtweise, welche Technik als Ausgangspunkt
für alle Veränderungen und Gestaltungsmaßnahmen sieht, und zum anderen die sozio-technische Sichtweise, welche
technisches und soziales gemeinsam optimiert\footnote{\ Feldmüller; Dorothee; Mütter, Jan. (2007). Change Management \&
IT Projekte.}. In dem besonderen Fall einer Hochschule und deren Gegebenheiten, sollte die sozio-technische
Herangehensweise der deterministischen vorgezogen werden (vgl. Kapitel 4.1.2.2). \ 


\bigskip

Angelehnt an die genutzten Change Management Tools welche zur Unterstützung der Strategieumsetzung an der Universität
Heidelberg eingesetzt wurden, könnten die Tools für die Neuordnung des Informationsmanagements an der Hochschule
Emden/Leer wie folgt aussehen.


\bigskip

\begin{center}
\bottomcaption{Change Management Tools}
\tablefirsthead{}
\tablehead{}
\tabletail{}
\tablelasttail{}
\begin{supertabular}{|m{3.675cm}|m{3.675cm}|m{3.675cm}|m{3.675cm}|}
\hline
Veränderungsprozesse steuern &
Information und Kommunikation &
Partizipation &
Konsolidierung nach dem Go Live\\\hline
Definition einer Projektstruktur &
Kommunikationspläne &
Feedback zur Optimierung der Veränderungsprozesse &
Support\\\hline
Controlling durch Statusberichte &
Informationsveranstaltungen &
Training / Coaching &
~
\\\hline
\end{supertabular}
\end{center}

\bigskip

In wieweit und in welchem Umfang sich die einzelnen Bausteine für die Vorhaben Responsive Website und Alfresco
Dokumentenmanagement einsetzen lassen, soll in den folgenden Kapitel näher betrachtet werden.


\bigskip

4.1.2.3.1 \ \ Responsive Website

Im Rahmen der Erstellung eines responsive Designs der Webpräsenz der Hochschule soll gleichzeitig eine aktuelle TYPO3
Version migriert werden. 


\bigskip

Beide Vorhaben stellen eine technische Migrationen dar. Inhalte und Funktionen \ der Webpräsenz werden von Veränderungen
nicht betroffen sein. Lediglich im Layout, welches durch die responsive Implementierung für alle Medien optimal
dargestellt wird, werden leichte Veränderungen wahrzunehmen sein. 


\bigskip

Bei den Anwendern wird es dadurch keine Veränderungen bei Prozessen, Arbeitsweisen oder dem benötigtem Wissen geben.
Daher ist auf psychologischer Ebene also kein umfangreiches Change Management von Nöten, da hier auch nicht mit
Widerständen zu rechnen ist. 


\bigskip

Jedoch bietet es sich an, bei einer Neu-Implementierung auch eventuelle Verbesserungen, sei es von Funktionen, Layout
oder Usability, gleich mit zu implementieren. Dafür sollten alle relevanten Akteure (hier die Verantwortlichen der
Internetauftritte der verschiedenen Bereiche) in das Vorhaben einbezogen werden und die Möglichkeit haben Vorschläge
und Wüsche zu äußeren und über diese zu diskutieren. 


\bigskip

Das eigentliche Change Management richtet sich in diesem Fall an die IT-Mitarbeiter welche die neuen Systeme aufsetzen
und pflegen. Aber auch hier werden die Betroffen nicht vor neue Aufgaben, Prozesse oder Arbeitsweisen gestellt, da die
Migration neuer Systeme im Aufgabenfeld eines IT-Mitarbeiters verankert ist. 


\bigskip

4.1.2.3.2 \ \ Alfresco

Das Change Management für die Umstellung auf das Dateimanagement Alfresco ist dabei etwas Umfangreicher als bei der
Erstellung eines responsive Designs für die Webpräsenz der Hochschule. 


\bigskip

Hier handelt sich um ein grundlegend neues System an der Hochschule. Daher sollten frühzeitig alle relevanten Akteure in
den Endscheidungsprozess Einbezogen werden. Es empfiehlt sich einen Kommunikationsplan zu erstellen um schon frühzeitig
eine Übersicht dafür zu bekommen wann welche Informationen an wenn und auf welchem Weg kommuniziert wird. 


\bigskip

Im Sinne der Partizipation sollten alle relevanten Akteure die Möglichkeit haben während des Veränderungsprozess ihr
Feedback zur Diskussion zu stellen. Diese Möglichkeit könnte beispielsweise auf einer Informationsveranstaltung, welche
durch die Hochschulleitung organisiert wird, wahr genommen werden. 


\bigskip

Hierbei können bei Bedarf weitere Anforderungen in das Lastenheft aufgenommen werden. Die Aufgabe des Managements ist es
dabei zwischen den geforderten Anforderungen abzuwägen, sodass ein „Nein“ zu Änderungen oder Erweiterungen des
Lastenhefts auch zielführend sein kann. Denn Entscheidungen auf dieser Ebene schaffen weitere Veränderungen für die
Betroffenen\footnote{\ Kleinhesseling, Frank. (2011). Change Management IT-Projekte erfolgreich umsetzen.}. 


\bigskip

Nach Analyse aller Feedbacks und der Optimierung der Zielsetzung kann die Migration des neuen Systems (Alfresco)
beginnen. Zur Erhöhung der Akzeptanz, \ ist es nach wie vor wichtig auch in dieser Phase die Kommunikation,
beispielsweise durch Statusberichte, mit den Betroffenen aufrecht zu erhalten. \ 


\bigskip

Des Weiteren können Workshops und Weiterbildungen den Betroffen dabei helfen, ihre Zweifel weiter abzubauen und sich mit
den neuen IT-System vertraut zu machen. Hierzu bietet Alfresco beispielsweise eigens entwickelte Trainings für
Entwickler, Administratoren und End User an (vgl. Kapitel . 4.1.4.2.3).


\bigskip

Konnte durch das Change Management eine Vielzahl von Zweifeln und Vorbehalte der Betroffenen gegenüber der Veränderungen
abgebaut werden, so treten die tatsächlichen Veränderungen erst nach der Migration und dem Go Live des neuen Systems in
vollem Umfang in Kraft. In dieser Phase muss die gewonnene Akzeptanz der Betroffenen weiter untermauert werden und
sollte nicht durch mögliche Probleme mit dem neuen Software-System in Ablehnung oder gar Verweigerung münden.
Entsprechende Support Angebote könnten hier Abhilfe schaffen. Alfresco bietet für Nutzer der Enterprice Edition ein
umfangreiches \ Support an (vgl. https://www.alfresco.com/de/node/1084). 

\subsection{}
\clearpage\subsection[4.1.3\ \ Migrationsplan (Marco)]{4.1.3\ \ Migrationsplan (Marco)}

\bigskip


\bigskip


\bigskip


\bigskip


\bigskip


\bigskip

Literatur


\bigskip

Cohen, A. R.; Fetters, M.; Fleischmann, F. (2005). Major change at Babson \ \ College: Curricular and administrative,
planned and otherwise. Advances in \ \ Developing Human Resources. Zitiert nach Langenbeck, Ute; Suchanek, \ \ Justine;
Höschler, Barbara. 2011. Change Managment an Hochschulen: Zu \ \ den Potentialen des Coaching einer
Expertenorganisation. In Höschler, \ \ Barbara; Suchanek, Justine (Hersg.). Wissenschaft und Hochschulbildung \ \ im
Kontext von Wirtschaft und Medien. VS Verlag f\"{u}r Sozialwissenschaften \ \ Springer Fachmedien Wiesbaden GmbH 2011.


\bigskip

Feldmüller; Dorothee; Mütter, Jan. (2007). Change Management \& IT Projekte. I\ \ nformation und Technik NRW.
http://www.it.nrw.de/informationstechnik/ \ \ Services/IT\_Veroeffentlichungen/Ausgabenarchiv/ausgabe1\_2007/
\ \ schwerpunkte/z091200751\_s15.pdf. 30. Mai 20115.


\bigskip

Fuchs, M. (2007). Change Management an Hochschulen. Die strategische \ \ Integration von Bildungsinnovationen. Zitiert
nach Langenbeck, Ute; \ \ Suchanek, Justine; Höschler, Barbara. 2011. Change Managment an \ \ Hochschulen: Zu den
Potentialen des Coaching einer \ \ Expertenorganisation. In Höschler, Barbara; Suchanek, Justine (Hersg.).
\ \ Wissenschaft und Hochschulbildung im Kontext von Wirtschaft und Medien. \ \ VS Verlag f\"{u}r Sozialwissenschaften
Springer Fachmedien Wiesbaden \ \ GmbH 2011.


\bigskip

Kleinhesseling, Frank. (2011). Change Management IT-Projekte erfolgreich \ \ umsetzen. IHK Wirtschaftsforum Juli 2011.
http://www.frankfurt-\ \ main.ihk.de/branchen/mediacity/tk\_it/it\_mittelstand/change\_management/. \ \ 30. Mai 2015. 


\bigskip


\bigskip


\bigskip

Langenbeck, Ute; Suchanek, Justine; Höschler, Barbara. (2011). Change \ \ Managment an Hochschulen: Zu den Potentialen
des Coaching einer \ \ Expertenorganisation. In Höschler, Barbara; Suchanek, Justine (Hersg.). \ \ Wissenschaft und
Hochschulbildung im Kontext von Wirtschaft und \ \ Medien. VS Verlag f\"{u}r Sozialwissenschaften {\textbar} Springer
Fachmedien \ \ Wiesbaden GmbH 2011.


\bigskip

Lauer, Thomas. (2014). Change Management. Springer-Verlag Berlin \ \ \ \ Heidelberg 2010, 2014. 


\bigskip

Sch\"{o}nwald, I. (2007). Change Management in Hochschulen – Die Gestaltung \ \ soziokultureller
Ver\"{a}nderungsprozesse zur Integration von E-Learning in \ \ die Hochschullehre. Zitiert nach Langenbeck, Ute;
Suchanek, Justine; \ \ Höschler, Barbara. 2011. Change Managment an \ \ Hochschulen: Zu den \ \ Potentialen des
Coaching einer Expertenorganisation. In Höschler, Barbara; \ \ Suchanek, Justine (Hersg.). Wissenschaft und
Hochschulbildung im \ \ Kontext von Wirtschaft und Medien. VS Verlag f\"{u}r Sozialwissenschaften \ \ Springer
Fachmedien Wiesbaden GmbH 2011.


\bigskip

Sonntag, Karlheinz; Stegmaier Ralf; Michel, Alexandra. (2008). Change \ \ Management an Hochschulen: Konzepte, Tools und
Erfahrungen bei der \ \ Umsetzung. In Fisch, Rudolf; Müller, Andrea (Hrsg.). Veränderungen in \ \ \ \ Organisationen.
VS Verlag f\"{u}r Sozialwissenschaften {\textbar} GWV \ \ \ \ \ \ Fachverlage GmbH, Wiesbaden 2008. 


\bigskip


\bigskip


\bigskip


\bigskip


\bigskip


\bigskip


\bigskip


\bigskip
\end{document}
