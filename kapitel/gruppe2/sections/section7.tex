\section{Bewertung und Gewichtung (TK)}
\label{section_bewertung_gewichtung}
Abschießend kann gesagt werden, dass Informationen zentral gesammelt und zur Verfü-gung gestellt werden. 
Wichtige Systeme wie das "'Laufwerk Y"' und die E-Learning Plattform (Moodle) kommen in allen 
Fachbereichen und in Teilen der Verwaltung zum Einsatz. Durch den starken Kooperationsverbund werden 
zentrale Dienste, wie Springer Link, WISO und video2brain für die Studierenden und Mitarbeiter dezentral zur 
Verfügung gestellt. 

Bei der Repräsentation von Informationen nach außen verfügt die Hochschule über eine Pressestelle und eine 
Marketingabteilung. Es existiert eine feste CD-Reglung für alle Abteilungen und Bereiche. Ebenso ist ein 
einheitliches Erscheinungsbild über die Webseite der Hochschule Emden/Leer gegeben.

Durch diverse Arbeitsgruppen ist der Erfahrungs-, Wissens- und Informationsaustausch für wichtige Bereiche 
bereits gegeben. Durch die Arbeitsgruppe ZDF, WEB und Moodle werden zentrale Systeme zur 
Wissenserhaltung und Informationsbereitstellung gepflegt. Da die Arbeitsgruppen abteilungsübergreifend 
agieren, besteht auch zwischen den einzelnen Bereichen eine Schnittstelle, ohne die autarken Fachbereiche 
einzuschränken. 

In Bezug auf Serviceorientierung und IT-Sicherheit lässt sich sagen, dass SSO (Single-Sign-On) für einige 
Bereiche bereits zum Einsatz kommt (siehe Kapitel \ref{realisierung_der_serviceorientierung}). Ebenso werden 
Teile des IT-Grundschutzes erfolgreich an der Hochschule umgesetzt. Dies sind erste Schritte zu einem 
vollständigen Informationsmanagement. Es fehlt jedoch grundsätzlich ein zentrales System für den direkten 
Zugriff und zur Weiterleitung auf weitere Informationssysteme. In Kapitel 
\ref{immatrikulations_und_pruefungsamt} wird beschrieben, dass in Hochschulen, die ein 
Informationsmanagement einsetzen, dieses meistens im Bereich Immatrikulations- und Prüfungsamt (HIS) 
angesiedelt ist. 

Neben diesem zentralen Sammelpunkt, fehlt auf der organisatorischen Seite eine weitere Instanz, die die 
Zuständigkeiten auf der strategischen Ebene regelt. Wie in Kapitel \ref{subsubsection_cio} beschrieben, findet im 
klassischen Informationsmanagement für Unternehmen das Management häufig durch einen CIO (Chief 
Information Officer) statt. In Hochschulen wird in den meisten Best Practise Beispielen wie in Kapitel 4 
beschrieben, dies oft durch Dachorganisationen realisiert.

Auch wenn die Hochschule bereits diverse Arbeitsgruppen einsetzt, so ist diese Instanz des 
Informationsmanagements bisher unbesetzt. Ein Informationsmanagement, wie es in Kapitel 
\ref{begriffsdefinition_inm} beschrieben ist, wird derzeit an der Hochschule nicht vollständig praktiziert. Einen 
möglichen Lösungsvorschlag um das bisherige, teilweise integrierte Informationsmanagement zu verfeinern 
ist in Kapitel 6 ausführlich dargestellt. Auch über ein mögliches Konzept zur Erreichung dieser Soll-Situation 
werden in Kapitel 7 und 8 Aufschluss gegeben.