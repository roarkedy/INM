\section{Repräsentation von Informationen (TiK)}
Ein wichtiger Aspekt bei der Hochschule ist die Repräsentation von Informationen. Hier spielt sowohl das Erscheinungsbild nach außen sowie die Repräsentation von Informationen innerhalb der einzelnen Bereiche eine essenzielle Rolle. Für diese externe und interne Kommunikation ist die Presse- und Öffentlichkeitsarbeit der Hochschule Emden/Leer zuständig (siehe \ref\subsection{Präsidium}). Nachfolgend soll auf die wichtigsten Teilbereiche Studentengewinnung, Corporate Identiy und das Handling von Bewerberdaten eingegangen werden.

\subsection{Studentengewinnung}
Wie in Kapitel \ref{section_zustaendigkeiten} bereits erwähnt, obliegt die Zuständigkeit der des Marketings dem Präsidialbüro. Generell ist die Studentengewinnung wie folgt aufgeteilt: Zum Einen existiert eine zentrale Studienberatung, bei welcher sich Interessenten direkt informieren können und zum Anderen werden regelmäßig Besuche der Hochschule bei Schulen in der Region durchgeführt. Mitglieder der jeweiligen Fachbereiche berichten vor Ort in der jeweiligen Schule über die Inhalte der jeweiligen Studiengänge. Neben diesen beiden Maßnahmen zur Studentengewinnung verfügt die Hochschule Emden/Leer ebenso über Onlinemedien, mit dessen Hilfe sich die Interessenten speziell im Bezug auf den gewünschten Studiengang und den generellen Ablauf im Studium informieren können.

\subsection{Corporate Identity}
Die Hochschule verfügt über eine Corporate Design (CD) Reglung, welche auf der Webseite öffentlich eingesehen werden kann. Diese wird von dem Bereich Marketing zur Verfügung gestellt und gepflegt.\footnote{\url{http://www.hs-emden-leer.de/einrichtungen/praesidialbueropresse-und-oeffentlichkeitsarbeit/corporate-design.html}}

Diese Reglung umfasst unter anderem einen CD-Regelungsguide\footnote{\url{http://www.hs-emden-leer.de/fileadmin/user_upload/Einrichtungen/Praesidialbuero/Downloads/HS_CD_Manual_08_2013.pdf}}, sowie diverse Vorlagen für PowerPoint Präsentationen bis hin zum allgemeinen Logo und anepasste Logos für jeden Fachbereich (siehe Abbildung \ref{fig_logo_allgemein} und \ref{fig_logo_fb_technik}).

\begin{figure}[h!]
	\centering
	\includegraphics[width=8cm]{kapitel/gruppe2/bilder/hs_logo_allgemein}
	\caption{Allgemeines Logo der Hochschule Emden/Leer}
	\label{fig_logo_allgemein}
\end{figure}

\begin{figure}[h!]
	\centering
	\includegraphics[width=8cm]{kapitel/gruppe2/bilder/hs_logo_technik}
	\caption{Logo des Fachbereiches Technik}
	\label{fig_logo_fb_technik}
\end{figure}

\subsection{Handling von Bewerberdaten}
Durch das Interview mit Herrn Günter Müller konnte festgestellt werden, wie das allgemeine Handling von Bewerberdaten aus Sicht des Rechenzentrums stattfindet. 
Der Bewerber meldet sich an dem HIS der Hochschule mit seinen Daten an und dieser Account ist nur für die Dauer des Bewerbungszeitraumes aktiv. Wird der Bewerber nicht angenommen, so wird dieser Account im Anschluss wieder gelöscht. Wird der Bewerber jedoch akzeptiert, wird der vorhandene temporäre Account in einen permanenten Account mit erweiterten Informationseingaben, wie z.B. Krankenkassendaten, umgewandelt. Auf diese Weise ist sichergestellt, das nur aktive Studenten einen Account zur Verfügung gestellt bekommen.