\section{Repräsentation von Informationen (TK)}
Ein wichtiger Aspekt an Hochschulen ist die Repräsentation von Informationen. Hier spielt sowohl das 
Erscheinungsbild nach außen, als auch die Repräsentation von Informationen innerhalb der einzelnen Bereiche 
eine entscheidende Rolle. Für die interne und externe Kommunikation ist der Bereich "'Presse- und 
Öffentlichkeitsarbeit"' der Hochschule zuständig (siehe Kapitel \ref{praesidium_label}). Nachfolgend wird auf 
die Teilbereiche Studentengewinnung, Corporate Identiy und das Handling von Bewerberdaten näher 
eingegangen.

\subsection{Studentengewinnung}
Wie in Kapitel \ref{praesidium_label} bereits erwähnt, obliegt die Zuständigkeit des Marketings dem 
Präsidialbüro. Generell ist die Studentengewinnung wie folgt aufgeteilt: Zum einen existiert eine zentrale 
Studienberatung, bei der sich Interessenten direkt informieren können und zum anderen werden regelmäßig 
Besuche von Mitarbeitern der Hochschule an Schulen der Region durchgeführt. Mitglieder der Fachbereiche 
berichten vor Ort in den Schulen über die Inhalte der jeweiligen Studiengänge. Neben diesen beiden 
Maßnahmen zur Studentengewinnung verfügt die Hochschule ebenso über Onlinemedien, mit deren Hilfe sich 
die Interessenten über den gewünschten Studiengang und den generellen Ablauf des  Studiums informieren 
können.\footcite[Vgl.][]{gunter_muller_interview}


\subsection{Corporate Identity}
Die Hochschule hat eine Corporate Design-Regelung, die auf der Webseite öffentlich eingesehen werden 
kann. Diese wird vom Bereich Marketing zur Verfügung gestellt und 
gepflegt.\footcite[Vgl.][]{hsel_CD}

Die CD-Reglung umfasst unter anderem einen 
CD-Regelungsguide\footcite{hsel_CD_manual} sowie diverse Vorlagen für PowerPoint 
Präsentationen bis hin zu allgemeinen Logos sowie angepasste Logos für jeden Fachbereich (siehe Abbildung 
\ref{fig_logo_allgemein} und \ref{fig_logo_fb_technik}).

\begin{figure}[h!]
	\centering
	\includegraphics[width=8cm]{kapitel/gruppe2/bilder/hs_logo_allgemein}
	\caption{Allgemeines Logo der Hochschule Emden/Leer\protect\footnotemark}
	\label{fig_logo_allgemein}
\end{figure}\footnotetext{\cite{hsel_CD_manual}}

\begin{figure}[h!]
	\centering
	\includegraphics[width=8cm]{kapitel/gruppe2/bilder/hs_logo_technik}
	\caption{Logo des Fachbereiches Technik\protect\footnotemark}
	\label{fig_logo_fb_technik}
\end{figure}\footnotetext{\cite{hsel_CD_manual}}

\subsection{Handling von Bewerberdaten}
Durch das Interview mit Günter Müller wurde festgestellt, wie das allgemeine Handling von Bewerberdaten aus Sicht des Rechenzentrums stattfindet. 
Der Bewerber meldet sich mit seinen bei Registrierung erstellten Daten am Hochschulinformationssystem (HIS) der Hochschule an und der generierte Account ist nur für die Dauer des Bewerbungszeitraumes aktiv. Im Fall der Nichtannahme des Bewerbers, erfolgt im Anschluss die Löschung seines Accounts. Bei Immatrikulation des Bewerbers wird der vorhandene temporäre Account in einen permanenten Account mit erweiterten Informationseingaben umgewandelt (z.B. Krankenkassendaten). Auf diese Weise wird sichergestellt, dass nur aktive Studenten einen Account zur Verfügung gestellt bekommen.\footcite[Vgl.][]{gunter_muller_interview}