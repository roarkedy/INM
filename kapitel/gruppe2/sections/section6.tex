\section{Kooperations-Situation mit anderen Hochschulen}
\label{section_kooperations_situation}

Kooperationsverhältnisse mit anderen Hochschulen, Verbänden und Unternehmen spielt auch an der Hochschule Emden/Leer eine sehr entscheidende Rolle, da durch die Kooperation verschiedene Services zur Verfügung stehen.

\subsection{Regionaler Bezug zu Hochschulen (Mitgliedschaften)}
Die Hochschule Emden/Leer pflegt ein enges Kooperationsverhältnis mit dem Jade-Hochschulverbund. Über das Lokale Bibliothekssystem Ostfriesland/Wilhelmshaven (LBS) wird auf die gemeinsamen  Bibliotheksbestände zugegriffen.\footnote{\url{http://www.jade-hs.de/service-verwaltung/hochschulbibliothek/bestand/online-kataloge/}}

Durch ein gemeinsames Promotionskolleg findet ebenfalls eine intensive Zusammenarbeit mit der Universität Vechta statt. Aktuell baut die Hochschule Emden/Leer die Kooperation zur Hochschule Osnabrück aus.\footnote{\url{http://www.hs-emden-leer.de/hochschule/profil.html}}

Der Rechenzentrumsleiter Herr Günter Müller ist selbst Mitglied des Arbeitskreises LANIT / HRZ. Hier treffen die Leiter der Rechenzentren Niedersachsens aufeinander und tauschen Ihre Erfahrungen aus. Der Arbeitskreis befasst sich mit Themen der IT-Infrastruktur für Forschung, Lehre und Verwaltung an den Hochschulen Niedersachsens. Zu Schwerpunktthemen wurden Arbeitsgruppen eingerichtet.  Ebenfalls werden hochschulübergreifend Projekte durchgeführt.\footnote{\url{http://www.lanit-hrz.de/}}

Das ZKI (Zentren für Kommunikation und Informationsverarbeitung in Lehre und Forschung e.V.) spielt für die Hochschule eine wichtige Rolle. Ziel dieses Vereins ist es, die Kooperation zwischen den ZKI/Rechenzentren, Meinungs- und Erfahrungsaustausch, sowie die Beratung und Zusammenarbeit mit bildungs- und wirtschaftsfördernden Einrichtungen zu fördern. In den immer wiederkehrenden Tagungen erarbeitet der Arbeitskreis Lösungsvorschläge für aktuelle Probleme der Informationsverarbeitung. Aktuelle Themen sind z.B. eine Studie über das Thema: „CIOs und IT-Governance an deutschen Hochschulen“.\footnote{\url{https://www.zki.de/zki-nachrichten/einzelbeitrag/1215/}}

Durch den Verein DFN (Deutsches Forschungsnetz) wird der Hochschule eine Vielzahl von maßgeschneiderten Kommunikationsanwendungen (DFN-Diensten) zur Verfügung gestellt. Der DFN-Verein ist ein von der Wissenschaft selbst organisiertes Kommunikationsnetz für Wissenschaft und Forschung in Deutschland. Er verbindet Hochschulen und Forschungseinrichtungen miteinander und ist in den europäischen und weltweiten  Verbund der Forschungsnetze integriert.\footnote{\url{https://www.dfn.de/}}

\subsection{Kooperation zwischen Unternehmen}
Regional betrachtet arbeitet die Hochschule Emden/Leer mit 82\% der Unternehmen der Region zusammen. Der Vorteil dieser Kooperation ist es, dass zum einen die Studierenden die Möglichkeit haben ihre Fähigkeiten in der Praxis anzuwenden und zum anderen können die Unternehmen das Know-How  der Hochschule nutzen und sinnvoll einsetzen. Der Wissenstransfer der in der Hochschule erfolgt, bietet den Unternehmen einen erheblichen Mehrwert.\footnote{\url{http://www.hs-emden-leer.de/en/news-events/news/article/hochschule-weit-vorn-bei-kooperation-mit-unternehmen.html}}

\subsection{Eingesetzte IT-Systeme durch Mitgliedschaft im DFN-Verein}
Die angebotenen Dienste des Deutsches Forschungsnetzes sind für den Zweck von Wissenschaft und Forschung maßgeschneidert worden. Ein besonderes Augenmerk liegt hier auf die gute Integration der Dienste in der Prozesse der Hochschulen.  Auf die Dienste die der DFN-Verein der Hochschule Emden/Leer zur Verfügung stellt, wird fort folgend nun näher eingegangen.\footnote{\url{https://www.dfn.de/dienstleistungen/}}

\subsubsection{DFNRoaming/eduroam}
Die Hochschule Emden/Leer ist Mitglied des deutschen Forschungsnetzes (DFN). Durch diese Kooperation nutzt die Hochschule den durch das DFN zur Verfügung gestellten Dienst DFNRoaming/eduroam. Dieser ermöglich es, registrierten Nutzern über dienstkonforme WLANs Zugang zum Wissenschaftsnetz zur Verfügung zu stellen.  Der DFN-Verein betreibt und pflegt die eduroam Förderationsserver in Deutschland.\footnote{\url{https://www.dfn.de/dienstleistungen/dfnroaming/}}

\begin{figure}[h!]
	\centering
	\includegraphics[width=8cm]{kapitel/gruppe2/bilder/eduroam_map}
	\caption{http://weill.cornell.edu/its/images/eduroam-map.jpg}
	\label{fig_map_eduroam}
\end{figure}

\subsubsection{GigaMove der RWTH Aachen}
Die Rheinisch-Westfälische Technische Hochschule Aachen (RWTH Aachen) stellt eine einfach zu nutzende Möglichkeit  zum kurzfristigen Austausch großer Dateien zur Verfügung. Der Datenaustausch kann aus zwei Richtungen angestoßen werden. Zum einen kann ein Nutzer eine Datei hochladen und die Anwendung erzeugt einen Link zum Download, zum anderen kann eine Datei angefordert werden bei der die Anwendung einen Link generiert, der zu einem Formular zum Upload der Datei führt. Jeder Nutzer darf standardmäßig Dateien in der Gesamtgröße von 10 GB für einen Zeitraum von 14 Tagen auf den gehosteten Servern abspeichern. Der von der RWTH Aachen gehostete Dienst GigaMove wird den Nutzern der Hochschule Emden/Leer zur Verfügung gestellt.

\begin{figure}
	\centering
	\includegraphics[width=8cm]{kapitel/gruppe2/bilder/rwth_gigamove}
	\caption{https://gigamove.rz.rwth-aachen.de/instructions/wicket:pageMapName/wicket-0}
	\label{fig_rwth_gigamove}
\end{figure}

\subsubsection{DFNVideoConference (DFNVC)}
DFVNC bietet den Nutzern die Möglichkeit von einem PC, einem Raumsystem oder einem Telefon durch Nutzung des Wissenschaftsnetzes X-WiN mit einem oder mehreren Nutzern zu kommunizieren. Die Kommunikation findet multimedial statt. Das Wissenschaftsnetz X-WiN ist die technische Plattform des deutschen Forschungsnetzes. Über das X-WiN sind die Hochschulen und Forschungseinrichtungen in Deutschland untereinander und mit den Wissenschaftsnetzen in Europa auf anderen Kontinenten vernetzt.\footnote{\url{https://www.dfn.de/dienstleistungen/dfnvc/}} An der Hochschule wird dieser Dienst ebenfalls genutzt.

\subsection{Authentifizierung über das Shibboleth-Verfahren}
Das von der Hochschule eingesetzte Shibboleth-Verfahren ermöglicht den Studierenden und Mitarbeitern Ressourcen der Anbieter SpringerLink, WISO, video2brain und andere Dienste nutzen. Hierbei muss  bei der Anmeldung die Hochschule als Heimatinstitution und die Hochschulkennung angegeben werden.\footnote{\url{http://www.hs-emden-leer.de/no_cache/einrichtungen/bibliothek/medienangebot/elektronische-angebote/vpn-shibboleth.html}}

\subsubsection{Springer Link}
Studierende und Mitarbeiter der Hochschule Emden/Leer haben über die Springerlink-Kooperation Zugriff auf über 40.000 Bücher, 5 Millionen Artikel, 2200 Zeitschriften und 165 Nachschlagewerke. Des Weiteren kann jedes Dokument als PDF-Dokument heruntergeladen werden.\footnote{\url{http://link.springer.com/athens-shibboleth-login}}

\begin{figure}[h!]
	\centering
	\includegraphics[width=8cm]{kapitel/gruppe2/bilder/springerlink_startseite}
	\caption{Springer Link Startseite}
	\label{fig_springerlink_startseite}
\end{figure}

\subsubsection{WISO}
Durch das Kooperationsverhältnis mit der GBI-Genios Deutsche Wirtschaftsdatenbank GmbH können die Mitglieder der Hochschule Emden/Leer das komplette Angebot an Fachinformationen zu den Wirtschafts- und Sozialwissenschaften, zu technischen Studiengängen und zur Psychologie  nutzen. WISO bietet über 14 Mio. Literaturnachweise, 2100 elektronische Bücher, 130 Mio. Artikel, 700.000 Marktdaten.\footnote{\url{https://www.wiso-net.de/popup/ueber_wiso}}

\subsubsection{video2brain}
Seit 2015 ist für alle Mitarbeiter und Studierende der Zugang zum Videostreaming-Portal der video2brain GmbH möglich. Schwerpunkt sind IT- und Kreativ-Themen, Lehrvideos für Fotografen, Grafiker, Web- und Screendesigner. Das Verlagsangebot umfasst mehr als 1700 Video-Trainingskurse.\footnote{\url{http://de.wikipedia.org/wiki/Video2brain}}

\begin{figure}[h!]
	\centering
	\includegraphics[width=8cm]{kapitel/gruppe2/bilder/video2brain_suche}
	\caption{Suchergebnisse für Java bei video2brain, Quelle: \url{https://www.video2brain.com/de/search.htm?search_entry=Java}}
	\label{fig_video2brain_suchergebnis}
\end{figure}

\subsection{Support der Dienste}
Für die zentral angebotenen Dienste eduroam, Shibboleth, GigaMove übernimmt die Hochschule Emden/Leer den Endkundensupport. Die Mitarbeiter der Hochschule Emden/Leer bilden somit die zentrale Support-Schnittstelle und delegieren Anfragen, die vor Ort nicht gelöst werden können, an die entsprechenden Anbieter weiter.

Diese Informationen teile uns der Leiter des Hochschulrechenzentrums Emden/Leer, Herr Günter Müller in dem durchgeführten Interview mit.