
% Dies ist die Datei, in der alle Fäden der Arbeit zusammenlaufen.
%
\documentclass[a4paper, 12pt]{scrreprt}

\usepackage[utf8]{inputenc}
%\usepackage[applemac]{inputenc}
\usepackage[english,ngerman]{babel}
\usepackage[autostyle,german=guillemets]{csquotes}
\usepackage{parskip}
\usepackage{scrpage2}
\usepackage{graphicx}
\usepackage{tabularx}
\usepackage{todonotes}
\usepackage[nottoc]{tocbibind}
%\usepackage[tocflat]{tocstyle}
%\usepackage[tocgraduated]{tocstyle}
\setcounter{tocdepth}{4}
\setcounter{secnumdepth}{4}
\usepackage{listings}
\usepackage{caption}
%\usepackage{courier}
\usepackage{chngcntr}
\usepackage{hyperref}
\usepackage{url}
\urlstyle{rm}

\hypersetup{
	colorlinks,
	citecolor=black,
	filecolor=black,
	linkcolor=black,
	urlcolor=black
}
\usepackage[
backend=biber,
%style=authortitle-icomp,
style=authortitle,
isbn=true,
url=false,
doi=false
]{biblatex}
%\addbibresource{./struktur/literatur.bib}

\bibliography{struktur/literatur.bib}

\begin{document}
	% config for table of contents
	% ongoing enumeration of footnotes
	\counterwithout{footnote}{chapter}
		%deckblatt start
	\thispagestyle{empty}
	\begin{center}
		\Large{Hochschule Emden/Leer}\\
	\end{center}
	
	
	\begin{center}
		\Large{Fachbereich Technik}
	\end{center}
	\begin{verbatim}
	
	
	\end{verbatim}
	
	\begin{figure}[h!]
	    \centering
		\includegraphics[width=8cm]{struktur/grafiken/fb_technik_logo}
	\end{figure}
	
	\begin{verbatim}
	
	
	
	
	\end{verbatim}
	
	
	\begin{center}
		\textbf{Wissenschaftliche Arbeit}
		\textbf{im Studiengang Online-Medieninformatik,\\}
		\textbf{Modul: Informationsmanagement}
	\end{center}
	\begin{verbatim}
	
	
	
	
	
	
	
	
	
	
	\end{verbatim}
	
	\begin{flushleft}
		\begin{tabular}{lll}
			\textbf{Thema:} & & Potentielle Neuordnung des Informationsmanagements  \\ 
			& & einer kleineren Fachhochschule auf der Grundlage \\
			& & bestehender Lösungen an deutschen Hochschulen\\
			& & \\
			& & \\
			& & \\
			\textbf{eingereicht von:} & & (siehe Liste der Autoren)\\
			& & \\
			& & \\
			\textbf{eingereicht am:} & & 01.07.2015\\
			& & \\
			& & \\
			\textbf{Betreuer:} & & Prof. Maria Krüger-Basener
		\end{tabular}
	\end{flushleft}
	
	\newpage
	%deckblatt ende
	\listoftodos
    \begin{abstract}
	\paragraph*{Abstract}
	$\;$ \\
	$\;$ \\
	\textit{Autor: Andreas Willems}
	$\;$ \\
	$\;$ \\
	% Ziel der Arbeit
	Ziel dieses Gutachtens ist die Untersuchung des Informationsmanagements an der Hochschule Emden/Leer
	im Hinblick auf eine potenzielle Neuordnung.
	
	% Wiedergabe der Struktur
	% Grundlagen
	Der Grundlagenteil dieser Arbeit befasst sich zunächst mit dem gegenwärtigen Stand der Literatur in Bezug auf das Informationsmanagement
	speziell im Hinblick auf Hochschulen und betrachtet im Weiteren Trends des Informationsmanagements an Hochschulen und Ergebnisse der
	Einführung eines Informationsmanagements an anderen Hochschulen.
	% Ist-Zustand und Soll-Konzept
	Im weiteren Verlauf der Arbeit wird der gegenwärtige Stand des Informationsmanagements an der Hochschule Emden/Leer analysiert und
	daraus ein Konzept zur Neuordnung entwickelt.
	% Überführung und Kosten/Zeit
	Dieses Konzept wird zuletzt auf seine Umsetzbarkeit hin untersucht und Schätzungen hinsichtlich zeitlichem und finanziellem Aufwand 
	angestellt.
	
	% Wiedergabe der Ergebnisse
    % Ergebnis Gruppe 2
	Die Analyse der gegenwärtigen Situation kommt zu dem Ergebnis, dass die Hochschule Emden/Leer über kein
	Informationsmanagement verfügt.
	
	Das entwickelte Soll-Konzept
	
	
	
	
\end{abstract}
    \chapter*{Liste der Autoren}
Andreas Willems (Kapitel 1, 7, 9) \newline
Marco Beckmann (Kapitel 7) \newline
Christian Halfmann (Kapitel 7)
	\tableofcontents
	\chapter{Einleitung - AW}
\textit{Autor: Andreas Willems}

\ldots
	%\newpage
\chapter{Grundlegende Aufgaben und Organisation des Informationsmanagements und Besonderheiten an Hochschulen - AD, BH, MB}
\label{chapter_grundlagen_INM}

\todo{Update Gruppe 1.1}

%Autoren: Alina Düssmann, Boris Heiliger, Frank Holtmann, Miriam Boerger
\textit{Autoren: Alina Düssmann, Boris Heiliger, Miriam Boerger}

Einleitungstext zu diesem Kapitel...\\
\\
Im Folgenden soll das Informationsmanagement im Allgemeinen erklärt werden, auf die Besonderheiten von Hochschulen wird in den späteren Kapiteln genauer eingegangen.



\section{Begriffsdefinition des Wortes Informationsmanagement}
Das Informationsmanagement ist ein Bestandteil der Unternehmensführung und hat planende, kontrollierende und steuernde Aufgaben sowohl im strategischen als auch im operativen Bereich zu erfüllen. Zudem soll es die Entscheidungsprozesse in den Unternehmen oder Organisationen, in denen Informationsmanagement eingesetzt wird, mit den nötigen Informationen zu versorgen. Informationen sollten im Rahmen des Informationsmanagements als Ressource angesehen werden, die im Unternehmen gesammelt, verarbeitet und genutzt werden kann.\\

Das Informationsmanagement lässt sich im Wesentlichen in drei Aufgabenbereiche unterteilen. \\

Zum einen hat es die Klärung und Planung des \textbf{Informationsbedarfs} zur Aufgabe, in der abgewägt werden muss, welche Informationen (Qualität), wann (Dringlichkeit) und in welchem Umfang (Quantität) benötigt werden.\\


Ist der Informationsbedarf geklärt, muss die \textbf{Informationsbeschaffung} geplant und organisiert werden. Hier stellt sich die Frage, wo (Ort, Quelle, Medium), wie (Werkzeuge), wann (im günstigsten Moment) und durch wen (Qualifikation, Fähigkeiten) die Informationen beschafft werden können.\\


Sind die Informationen beschafft, folgt die \textbf{Informationssicherung}, \textbf{Nutzbarmachung} und \textbf{Nutzenmehrung}. Hier müssen die Informationen aufbereitet (Aus- und Bewerten), verarbeitet (Integrieren und Kombinieren), präsentiert (vor einer entsprechenden Zielgruppe) und dokumentiert (Archivieren) werden.\\

\subsection{Begriffsdefinition Information}
Information sind
\begin{itemize}
	\item immaterielle Güter
	\item keine freien Güter
	\item beliebig zu vervielfältigen
	\item nicht abnutzbar
	\item leicht erweiterbar und verdichtbar;
	\item leicht und schnell zu transportieren.
\end{itemize}

Informationen unterliegen des weiteren einem Informationslebenszyklus \emph{(Muss weiter ausgearbeitet werden)}.
\section{Informationsmanagementmodelle in der Literatur}
In der deutschsprachigen Literatur lassen sich viele verschiedene Arbeiten und Definitionen zum Thema Informationsmanagement finden, die sich zum Teil deutlich voneinander unterscheiden. Im folgenden werden die Modelle und Sichtweisen zum Informationsmanagement von Heinrichs\footnote{Wer und was bin ich?}, Wollnik\footnote{Wer und was bin ich?} und Krcmar\footnote{Wer und was bin ich?} vorgestellt.

\subsection{Informationsmanagement nach Heinrichs}
Lange Zeit stellte das 1987 erschienene Werk\footnote{Welches Werk?} von Heinrich das deutschsprachige Standardwerk im Bereich des Informationsmanagement dar. Entsprechend wurde es auch als Lehrbuch an Hochschulen eingesetzt.\\

Laut Heinrich wird unter Informationsmanagement das “Leitungshandeln (Management) in Unternehmen in Bezug auf Information und Kommunikation” verstanden. Es umfasst alle Führungsaufgaben, die sich mit Information und Kommunikation befassen. Diese Informations- und Kommunikationsaufgaben werden als Informationsfunktion bezeichnet, die den Schwerpunkt des Informationsmanagements darstellt.\\

Das Ziel des Informationsmanagements laut Heinrich ist es, eine Informationsinfrastruktur aufzubauen, die die Verteilung, Produktion und Nutzung vom Informationen zur Aufgabe hat. Die Informationsinfrastruktur dient dazu, das Leistungspotenzial der Informationsfunktion umzusetzen und somit einen optimaler Beitrag zum Unternehmenserfolg zu leisten.\\

Für die Umsetzung der Ziele werden die Aufgaben des Informationsmanagements in drei Ebenen strukturiert.
Die \textbf{strategische} Ebene plant, überwacht uns steuert die Informationsinfrastruktur.
Die \textbf{administrative} Ebene plant, überwacht und steuert die Komponenten der Informationsinfrastruktur (z.B. Anwendungssysteme, Mitarbeiter, Bestand an Daten).
Die \textbf{operative} Ebene umfasst Aufgaben und Nutzung der Informationsinfrastruktur. Mögliche Aktionsfelder für die operative Aufgabenebene stellen den laufenden Betrieb, die Nutzerunterstützung und die Störungsbeseitigung dar.\\

Auf jeder Aufgabenebene werden Methoden, Techniken und Werkzeuge eingesetzt, die die Durchführung der strategischen, administrativen und operativen Aufgaben durchführt und unterstützt. Die Gesamtheit dieser Methoden und Techniken wird von Heinrich als \emph{Information Engineering} bezeichnet.

\subsection{Informationsmanagement nach Wollnik}
Wollnik\footnote{In welchem Werk?} gliedert das Informationsmanagement in drei Ebenen.

Die \textbf{Ebene des Informationseinsatzes} und dessen Management befasst sich mit der Integration von Informationen in Produkte und Dienstleistungen. Des weiteren befasst es sich mit der Erschließung neuer Märkte durch den Einsatz von Informationstechnologie.\\

Die \textbf{Ebene der Informations- und Kommunikationssysteme} stellt die mittlere Managementebene dar. Laut Wollnik bestehen Informationssysteme aus folgenden Elementen/Komponenten: Aufgaben, Informationen, Personen, Geräte, Organisation und Programme. Diese bestimmen die Struktur eines Informationssystems. Die Aufgaben dieser Ebene sind die Festlegung, Erhaltung und Modifikation dieser Strukturen während des Lebenszyklus des Informationssystems.\\

Ein weiteres Handlungsobjekt dieser Ebene sind die Prozesse zur Gestaltung von Informationssystemen, die geplant, organisiert und kontrolliert werden müssen. Diese Ebene stellt das Verbindungsglied zwischen den betrieblichen Aufgaben (Ebene Eins) und der technischen Infrastruktur (Ebene Drei) dar.\\

Die \textbf{Ebene der Informations- und Kommunikationsinfrastruktur} ist die unterste der drei Ebenen und befasst sich mit der Informationstechnologie. Dazu zählt laut Wollnik die Hard- und Software sowie die inhaltlichen Strukturen (zentrale Informationsbestände, Zugriffsberechtigungen auf Informationen). Kernaufgabe dieser Ebene ist der Betrieb und die Entwicklung der Infrastrukturen.\\

Diese drei Ebenen sind hierarchisch strukturiert und stellen den jeweils übergeordneten Ebenen Dienstleistungen zur Verfügung bzw. Stellen Anforderungen an die jeweils untergeordneten Ebenen. Dieses einfache Ebenenmodell stellt auch die Grundlage für viele weitere Informationsmanagement- modelle dar, unter anderem das von Krcmar.

\subsection{Informationsmanagement nach Krcmar}
Krcmars\footnote{in welchem Werk?} Strukturierung des Informationsmanagement basiert auf dem Ebenenmodell von Wollnik, erweitert es jedoch um allgemeine Führungsaufgaben mit ebenenübergreifenden Funktionen (IT-Governance, Strategie, IT-Prozesse, IT-Personal, IT-Controlling).\\

Er gliedert das Informationsmanagement in die drei Teilbereiche Informationswirtschaft, Informationssysteme und Informations- und Kommunikationstechnik.\\

Die \textbf{Informationswirtschaft} beschäftigt sich mit dem Angebot, der Nachfrage und Verwendung von Informationen.
Die \textbf{Informationssysteme} haben das Management von Daten, Prozessen und dem Anwendungslebenszyklus zur Aufgabe.
Die \textbf{Informations- und Kommunikationstechnik} weisen die Speicherung, Verarbeitung und Kommunikation von Information als Basisfunktionalitäten auf.

\section{Aufbau des Informationsmanagements nach Krcmar}
gestrichen
\section{Ziele des Informationsmanagements}
Das Informationsmanagement verfolgt zwei grundlegende Zielsetzungen. Das erste Ziel ist die Koordination der Informationslogistik bzw. die Gewährleistung der adressatengerechten Informationsversorgung. Das zweite Ziel ist die Unterstützung der Unternehmensziele durch eine zielgerichtete und wirtschaftliche Steuerung der Informatik.

\subsection{Koordination der Informationslogistik}
In erster Linie ist das Ziel des Informationsmanagements, tatsächlich relevante Information von der Menge an verfügbaren und eventuell unnützen Informationen zu trennen, die für einen Entscheidungsprozess benötigt werden. Hierzu muss jedoch erst einmal ein Informationsbedarf vorliegen, der die Art, Menge und Beschaffenheit der Informationen bestimmt und auf dessen Grundlage eine Entscheidung getroffen werden kann.\\

Die Definition des Informationsbedarfs hängt einerseits vom Entscheider, andererseits von den Anforderungen der zu treffenden Entscheidung ab.
Der Informationsbedarf lässt sich grundsätzlich in zwei Kategorien einteilen: in den objektiven und den subjektiven Informationsbedarf.\\

Der \textbf{objektive} Informationsbedarf wird in erster Linie durch die Entscheidung festgelegt und baut auf der Aufgabenbeschreibung des Entscheiders und den jeweiligen Marktgegebenheiten auf.
Der \textbf{subjektive} Informationsbedarf wird primär durch den Entscheider festgelegt. Welche Informationen für die Entscheidung relevant sind, werden durch die Einschätzungen und Präferenzen des Entscheiders mitbestimmt.\\

Aus der Überschneidung des objektiven und subjektiven Informationsbedarfs entsteht die Informationsnachfrage, die wiederum maßgeblich vom Informationsangebot abhängt. Somit legt der Informationsbedarf

\begin{itemize}
	 \item die Beschaffenheit (Qualität),
	 \item den Zeitpunkt der Lieferung,
	 \item den Ort, an dem geliefert wird und
	 \item das Medium, über das geliefert wird		 
\end{itemize}
in Bezug auf die Information fest. Im Hinblick auf die Unternehmensziele sollten die Informationen als Ressource angesehen werden. 

\subsection{Informationsmanagement als Unterstützung der Unternehmensziele}
Das Informationsmanagement bildet einen Teil der Unternehmensführung ab, der die Steuerung der Informatik (d.h. Mitarbeiter, Prozesse, organisatorische Teilbereiche und die eingesetzten Informationstechnologien) zur Verantwortung hat. Diese Informatik und deren Leistungen sollte dabei auf die Unternehmensziele ausgerichtet sein.\\

\emph{Muss noch weiter ausgearbeitet werden}
\section{Qualitätsmanagement der Informationsprozesse}
Im Folgenden wird der Qualitätsmanagement-Prozess in seinen Grundzügen definiert, am konkreten Beispiel der Minimierung von Durchlaufzeiten genauer betrachtet und praktisch mit Hilfe der IT Balanced Scorecard durchexerziert. Abschließend wird erörtert, welche Besonderheiten hierbei an Hochschulen bestehen und Möglichkeiten aufgezeigt, diese Schwierigkeiten zu umgehen.

\subsection{Aufgaben des Qualitätsmanagements}
In einem Informationsmanagement bildet das Qualitätsmanagement der Informationsprozesse einen zentralen Aufgabenbereich. Es übernimmt die Planung, Koordination und Steuerung der Informationsflüsse und prüft fortwährend, inwieweit eine Nutzbarkeit und Effizienz der Prozesse in der Realität gewährleistet ist, um deren Qualität gegebenenfalls mit gezielten Maßnahmen zu optimieren.\footnote{Quelle 1}\\

Hierzu fungiert ein Team von Qualitätsmanagern als Vermittler zwischen den verschiedenen Parteien im Unternehmen und überbrückt potentiell auftretende Kommunikations- oder Kulturbarrieren, um eine zielorientierte und effiziente Informationsversorgung der beteiligten Parteien zu ermöglichen.\\

Zu Beginn des Qualitätsmanagement-Prozesses gilt es, eine Leitstrategie aufzustellen. Hierfür wird der aktuelle Ist-Zustand des Unternehmens in Bezug auf seine Organisation von Informationsflüssen analysiert. Dabei zum Vorschein kommende Schwachstellen werden erfasst und durch mögliche optimierende Handlungsoptionen ergänzt.\footnote{Quelle 2}\\

Während der Durchführung der neu erschaffenen Maßnahmen ist das Qualitätsmanagement-Team mit der stetigen Überwachung dieser betraut. Bereits bei kleinen Abweichungen vom Plan kann so mit gegensteuernden Maßnahmen eingegriffen werden. Eine im Voraus aufgestellte Zeitplanung ist hierbei ebenso wichtig wie eine klare Definition der Zuständigkeiten im Qualitätsmanagement-Team, um eine termingerechte Erreichung der gesetzten Ziele noch zu garantieren.\\

Nach Ablauf des gesetzten Zeitrahmens oder nach Beendigung der Maßnahmen ist es erforderlich, mittels einer sogenannten Feedback-Analyse festzustellen, inwieweit das gesteckte Ziel erreicht wurde und aus welchen Gründen es nicht zu 100\% zufriedenstellenden Ergebnissen kommen konnte. Die hieraus resultierenden Erkenntnisse bilden daraufhin die Grundlage für eine anschließende Feedforward-Analyse, die die weitergehend erforderlichen Maßnahmen feststeckt, um in einer weiteren Phase die Zielerreichung durch verbesserte Maßnahmen zu garantieren.\footnote{Quelle 3}

\subsection{Prozessoptimierung durch Minimierung der Durchlaufzeiten}
Essenzielles Ziel des Qualitätsmanagement-Teams ist es, anhand bewährter Vorgehensweisen die Durchlaufzeiten von Informationen zu minimieren. Hierdurch wird der Informationsfluss quantitativ und qualitativ verbessert, da bestehende Abhängigkeiten der Parteien in Bezug auf die Informationen schneller bedient werden können und somit durch minimierte Wartezeiten eine beträchtliche Budgetersparnis resultiert.

\begin{figure}[h!]
	\centering
	\includegraphics[width=10cm]{kapitel/gruppe1_1/abbildungen/abbildung1}
	\caption{Minimierung von Durchlaufzeiten nach Bleicher 1991, 196}
	\label{fig_gruppe11_abbildung_1}
\end{figure}

Wie in Abbildung \ref{fig_gruppe11_abbildung_1} erkennbar, existieren elementare Methoden zur Reduktion von Durchlaufzeiten nach Bleicher aus dem Jahre 1991, die noch heute ihre Gültigkeit in der Anwendung haben.

Insbesondere das Zusammenfassen von Aktivitäten hat den entscheidenden Vorteil, dass Abstimmungsprozesse und Abhängigkeiten zwischen mehreren Parteien entfallen und somit die Umsetzungsdauer auf ein Minimum reduziert wird.\\

Auch die Methode des Parallelisierens sollte in den Fokus gerückt werden. Wie in Abbildung \ref{fig_gruppe11_abbildung_1} ersichtlich, werden hierbei mehrere Parteien, die für eine darauffolgende Partei relevant sind, zeitgleich geschaltet, um Wartezeiten zu verhindern.\\

Zu guter Letzt sei das Ergänzen von Prozessschritten betont. Auf den ersten Blick scheint diese Methode paradox, da durch Ergänzung weiterer Parteien der Zeit- und Arbeitsaufwand vorerst erhöht wird. Durch einen globaleren Blick wird schnell deutlich, dass ohne diese Parteien zu einem späteren Zeitpunkt Problematiken entstehen können, die in ihrer Lösung viel zeit- und arbeitsintensiver sind und das Unternehmen in seiner Prozessqualität deutlich zurückwerfen könnte.\\

Die in Abbildung \ref{fig_gruppe11_abbildung_1} gezeigten Methoden zur Durchlaufzeit-Minimierung sollten also vom Qualitätsmanagement-Team von Beginn an in die Planung mit einbezogen werden, da mit minimalem Aufwand eine weitreichende, inhaltlich und finanziell positive Auswirkung auf die Qualität des Gesamtprozesses erzeugt wird.

\subsection{Anwendung des Qualitätsmanagements am Beispiel der IT Balanced Scorecard}
Das strategisch-operative Konzept für eine qualitative Unternehmenssteuerung aus den 90er Jahren von R. S. Kaplan und D. P. Norton hat sich im Laufe der Zeit zum Standardinstrument entwickelt.\footnote{Quelle 4}\\

Grundlegend für die IT Balanced Scorecard ist das Schema Eingabe $\to$ Verarbeitung $\to$ Ausgabe $\to$ Resultat. Die Kombination von Qualität der Mitarbeiter, Kundenorientierung und finanzielle Ziele ermöglicht die Generierung und Sicherung eines gelungenen Informationsmanagements.\footnote{Quelle 5}\\

\begin{figure}[h!]
	\centering
	\includegraphics[width=10cm]{kapitel/gruppe1_1/abbildungen/abbildung1}
	\caption{IT Balanced Scorecard Kreislauf nach Gadatsch 2012}
	\label{fig_gruppe11_abbildung_2}
\end{figure}

Die IT Balanced Scorecard zeichnet sich – wie in Abbildung \ref{fig_gruppe11_abbildung_2} deutlich wird – durch eine stetige Feedback- und Feedforward-Kommunikation aus.\\

Zu Beginn des Managementprozesses werden in der Phase „Planung und Vorgaben“ die grundlegenden Ziele des Unternehmens definiert. In einem nächsten Schritt werden in der Phase „Vision und Strategie“ Kernaussagen zur Strategiefindung erarbeitet, insbesondere im Hinblick auf den Zusammenhang von Ursache und Wirkung, und Optimierungsmöglichkeiten zusammengestellt. Das Handlungskonzept wird in „Feedback und Lernen“ final ausformuliert und in der vierten Phase „Kommunikation und Verbindung“ mit der Strategie und übergeordneten Zielen verknüpft. Eine detaillierte Dokumentation von Teilzielen erhöht in dieser Phase die Motivation der Mitarbeiter zur Zielerreichung.\footnote{Quelle 6}\\

Die Definition von klaren Zielen, Bedingungen und Kennzahlen generiert ein komplexes Kennzahlensystem, welches durch Herunterbrechen der Strategie auf operatives Handeln einen ganzheitlichen Überblick über die interne Organisation des Unternehmens liefert. Die Einbeziehung von Ursache und Wirkung vereinfacht die vorausschauende Unternehmensführung und ergänzt die Sichtweise auf das Unternehmen zu einem ausgewogenen (balanced) Bild.

Da die Möglichkeiten für Eintragungen in die Scorecard sehr vielseitig sind, sollte vermieden werden, sie mit zu vielen komplexen Zahlen zu überladen. Im Fokus stehen bei diesem Konzept vorrangig die Maßnahmenfindung unter Berücksichtigung von Ursache und Wirkung, was durch eine einseitige Betrachtung der Kennzahlen zu sehr in den Hintergrund rücken und den Lösungsprozess negativ belasten könnte.



\section{Anwendung des Informationsmanagements am Beispiel von Hochschulen}
Ein gut funktionierendes Qualitätsmanagement kann nur effektiv und reibungslos funktionieren, wenn es an zentraler Stelle nahe des Entscheidungsträgers positioniert und gelebt wird. Die Umsetzungsverantwortung eines ganzheitlichen Qualitätsmanagements liegt bei Hochschulen in der Regel bei der Hochschulleitung, die ihre Aufgaben im Prozess der Informationsflussoptimierung begreifen und verantworten muss.\\

Eine der grundlegendsten Besonderheiten an Hochschulen liegt in der internen Strukturierung von Verantwortlichkeiten. Die Hochschule ist unterteilt in Fachbereiche, welche geschlossen für sich arbeiten können, aber dennoch der Hochschulleitung unterstellt sind. Zusätzlich zu diesen beiden Bereichen ist noch das Präsidium zu nennen, welches insgesamt für eine effiziente Aufgabenerfüllung und Interessensvertretung der Hochschule verantwortlich ist.\footnote{Quelle 7}\\

Im Zuge der Einführung eines geordneten Qualitätsmanagements gilt es also, die Positionierung nahe der Hochschulleitung mit einer anwendungsbezogenen Platzierung innerhalb jedes Fachbereiches unter Einbeziehung des Präsidiums zu verknüpfen, um ganzheitliche Lösungen zur Realisierung eines Qualitätsmanagements zu finden und umsetzen zu können. Ein Außenvorlassen des Fachbereichs, in dem die Lösungen schließlich umgesetzt werden, ist faktisch unmöglich. Durch die Vielzahl an Entscheidungsträgern und Mitredern besteht an Hochschulen ein höherer Bedarf an Kommunkations- und Abstimmungsleistungen zwischen diesen als in anderen Institutionen und Unternehmen. Es besteht zudem die Gefahr, dass Zuständigkeiten der verschiedenen Rollen an der entsprechenden Hochschule nicht klar geregelt sind, was die Funktionsweise des Entscheidungsprozesses zwar bestenfalls nicht beeinträchtigt, dessen Ablauf allerdings sehr unsystematisch gestaltet und den Fluß des Prozesses ausbremst.\\

Neben der strukturellen Schwierigkeiten in der Aufstellung eines Qualitätsmanagements besteht eine weitere Besonderheit in der inhaltlichen Vereinheitlichung der Anforderungen der einzelnen Parteien, die im schlechtesten Fall sehr verschieden sind oder sich gar widersprechen, sodass diese für alle Bereiche zentral gültig ist.\\

Mithilfe renommierter Werkzeuge, wie z.B. der IT Balanced Scorecard, liegt es nun in der Hand des Qualitätsmanagement-Teams, die erarbeiteten Prozessstrategien und Maßnamen transparent für jeden Bereich der Hochschule einsehbar zu publizieren und alle betreffenden Personen über Änderungen zu informieren. Die Kontrolle in den Fachbereichen, ob und inwieweit die Maßnahmen zur Prozessoptimierung beitragen, darf hierbei nicht vernachlässigt werden.\footnote{Quelle 8}

\subsection{Immatrikulations- und Prüfungsamt}
In Hochschulen, bei denen ein Informationsmanagement Anwendung findet, bildet das Immatrikulations- und Prüfungsamt eine Art interne Informationszentrale, welche weitere Bereiche mit notwendigen Informationen versorgt. Betrachet an einem Beispiel bedeutet dies Folgendes: Bei Immatrikulation eines neuen Studierenden wird diesem vom Immatrikulationsamt eine Matrikelnummer zugewiesen und seine Stammdaten ins HIS eingepflegt. Nun ist es Aufgabe des Immatrikulationsamtes, das HIS zu einer Art Schnittstelle für alle wichtigen Hochschulbereiche, wie z.B. die Bibliothek, die Mensa oder auch die Verwaltung von Computerräumen, zu machen, sodass diese Bereiche via Eingabe der Matrikelnummer auf für sie wichtige Studierendendaten zugreifen können. Um den Datenschutz der Studierenden zu garantieren, wäre hierfür eine Lösung mittels individueller Rechtezuweisung für jeden Bereich denkbar.\\

Der absolut saubere und stets aktuelle Datensatz im HIS wäre nicht nur zentral für alle Hochschulbereiche verfügbar, sondern auch jederzeit auf aktuellstem Stand, sodass Redundanzen ausgeschlossen werden können. Zur Minimierung des Verwaltungsaufwandes, wäre es denkbar, bei Stammdatenänderung durch das Immatrikulationsamt eine automatisch generierte E-Mail an alle beteiligten Bereiche mit den aktualisierten Informationen über den Studierenden zu versenden, was einem ganzheitlichen Informationsmanagement entsprechen würde.\\

Auch nach außen hin stellt das HIS eine zentrale Anlaufstelle für alle wichtigen Informationen wie Raumpläne, Kontaktdaten der Lehrenden und Prüfungsmodalitäten dar. Bei Ausfall einer Veranstaltung kann dieses dort direkt publik gemacht werden. Nach der Prüfungsanmeldung im HIS kann schnell und komfortabel aus den Anmeldedaten der Studierenden ein zentraler Raumbelegungsplan erzeugt werden.\footnote{Quelle 9}\\

Bei der Notenvergabe meldet der Prüfer die Noten der Studierenden an das Prüfungsamt, welche diese in das HIS einpflegen. Die Studierenden haben nun die Möglichkeit zentral ihre Noten abzurufen. Auch die Fachbereiche, welche über die Leistungen ihrer Studierenden informiert werden sollten, können auf diese Daten zugreifen.

Die Sammlung und Bereitstellung an zentraler Stelle wie dem HIS minimiert Abstimmungsmodalitäten zwischen den verschiedenen Hochschulbereichen, reduziert den Arbeitsaufwand für die erneute Erfassung und Verwaltung der Studierendendaten in dem jeweiligen Bereich und garantiert einen stets konsistenten Datensatz.

\subsection{Bibliotheken}
Hochschulbibliotheken werden tagtäglich mit einer Menge an Informationen und Daten konfrontiert. Von deren Besitz eines EDV-Systems zur Erfassung der Ausleihe inkl. Ablauf der Fristen und Stammdaten des Studierenden kann an dieser Stelle ausgegangen werden, da die Grundfunktionalität des Bibliothekssystems ansonsten kaum gewährleistet wäre. Als weitere Basisfunktion sei die Autorisierung der Studierenden zu nennen. Bei der Ausleihe wird in Hochschulbibliotheken über das System geprüft, ob dieser Studierende durch Immatrikulation dazu berechtigt ist, an dieser Hochschule Bücher auszuleihen.\\

Im Zuge eines angewandten Informationsmanagements wäre es von Vorteil, die Stammdaten der Studierenden direkt aus dem HIS auszulesen.\\

Aufbauend auf dieses Grundsystem existieren Lösungen, die das Bibliothekswesen mittels Informationsvermittlung, -speicherung und -auswertung für zahlreiche Einsatzmöglichkeiten bereichert. Jede Hochschule sollte sich etwas Zeit nehmen, sich mit einer EDV-Lösung zu befassen, die neben der elektronischen Erschließung der Ausleihfaktoren auch Werkzeuge zur statistischen Erfassung, Messung und Bewertung der Bestandsentwicklung und des Leihverhaltens bietet. Aus diesen statistischen Daten können Rückschlüsse auf das Verhalten der Studenten gezogen und wichtige Erkenntnisse für den weiteren Bestandsaufbau gezogen werden.\footnote{Quelle 10}\\

Je nach Größe der Bibliothek ist es sinnvoll, sich grundlegend Gedanken darüber zu machen, welche Mitarbeiter für die Medienbestellung zuständig sind und wer die Entscheidungskompetenz besitzt. Eine kontinuierliche Abstimmung optimalerweise mittels zentralem Verwaltungssystem untereinander ist unumgänglich, um Doppelbestellungen zu vermeiden und das Budget möglichst gewinnbringend für die Studierenden einzusetzen.\\

Die Mitarbeiter, die für die Medienbestellungen zuständig sind, sollten sich stetig auf dem Laufenden halten, welche Neuerungen es auf dem Büchermarkt gibt, um diese Werke möglichst aktuell in den Bestand aufnehmen zu können und den Studierenden eine topaktuelle Ausleihe zu garantieren. Die Bibliotheksleitung könnte über Kooperationen mit anderen Hochschulen zum Austausch von Neuerungen oder auch zum Tausch von Dubletten nachdenken, um dem Gesamtkonzept eines gelebten Informationsmanagements gerecht zu werden.\\

Die Studierenden könnten via Newsletter oder Website der Bibliothek darüber informiert werden, welche Neuerungen in den Bücherbestand aufgenommen wurden. Ab einer gewissen Bibliotheksgröße könnte auch ein Online-Katalog angedacht werden, der das Repertoire der Bibliothek abbildet und wichtige Informationen nach außen trägt. Ohne diese zentralen Informationsplattformen wäre ein Informationsmanagement an der Hochschule überflüssig.\\

\subsection{Rechnerpools}
Die Organisation der Nutzung von Rechnerpools zieht ohne zentrales Informationsmanagement einige Probleme nach sich. Doppelbelegungen und unnötig leerstehende Computerräume sind die Folge eines fehlenden zentralen Belegungssystems.\\

Das bereits erläuterte HIS könnte um genau diese Funktion erweitert werden. Die Lehrenden können sich im HIS einen Computerraum für ihre Lehrveranstaltungen verbindlich reservieren und bei Ausfall der Veranstaltung wieder für die Allgemeinheit freigeben. Da die Raumbelegung an zentraler Stelle geschieht, ist auch hier der klare Vorteil, dass der Plan jederzeit auf aktuellem Stand ist und von jedem Lehrenden oder Studierenden eingesehen werden kann, was Verzögerungen, die bei der Suche eines geeigneten Computerraums auf herkömmlichem Wege, eliminiert.

\chapter{Trends des Informationsmanagements an Hochschulen - AH, OS}
\textit{Autoren: Aurelian Hermand, Oliver Seidel}

%\section{Einleitung}
%Ich bin eine Einleitung zu diesem Kapitel, die noch geschrieben werden muss.
\section{Orientierungen im Informationsmanagement - OS}
\textit{Autor: Oliver Seidel}

Hochschulen befinden sich bei der Bereitstellung von Informationssystemen in einem stetigen Entwicklungsprozess. Ihnen stehen dabei drei grundlegende Orientierungen zur Verfügung: 
Serviceorientierung, Prozessorientierung und 
Architekturorientierung.\footcite[Vgl.][32]{leitner_itil_2008} Kapitel 
\ref{subsection_serviceorientierung} geht dabei auf die Bedeutung und 
Umsetzungsmöglichkeiten der Serviceorientierung ein. Kontinuierlich verbesserte Prozesse 
und Gestaltungen von IT-Strukturen werden im Kapitel \ref{subsection_prozessorientierung} 
behandelt. Die Architekturorientierung wird hier nicht betrachtet, Kapitel 
\ref{subsubsection_gestaltung_IT_strukturen} geht allerdings grundlegend auf 
Architekturveränderungen in der IT-Infrastruktur ein. Mittels Konklusion werden die Serivce- und Prozessorientierung am Ende gegenübergestellt und miteinander verglichen.


\subsection{Serviceorientierung}
\label{subsection_serviceorientierung}
Die zentralen Ziele der Serviceorientierung liegen darin, die Dienstleistungen auf die 
Anforderungen der Kunden auszurichten und dabei gleichzeitig ihre Qualität kontinuierlich 
zu verbessern. Kundenorientierung bedeutet weiter, die bestehenden und zukünftigen 
Kundenbedürfnisse zu kennen, die Interessen zu berücksichtigen und in den Mittelpunkt zu 
stellen.\footcite[Vgl.][34]{leitner_itil_2008} Die Herausforderung dieser Ambition liegt bei 
erstmaligem Betrachten in den technologiefokussierten IT-Abteilungen der Hochschule, die 
es in kundenorientierte IT-Dienstleister zu verwandeln gilt. Hochschulrechenzentren 
verstehen sich dabei als zentrale, wissenschaftliche Dienstleistungseinrichtung für 
öffentliche Hochschulen.\footcite[Vgl.][10]{schroeder_2011} Zusätzlich hindern 
wissenschaftliche Ambitionen des IT-Personals die Realisierung kundenorientierter 
Serviceangebote. Andererseits verbindet die Forschung wieder die Hochschulrechenzentren 
mit Ihren Kunden. Erschwerend kommt hinzu, dass die Rollenverteilung in Hochschulen 
zwischen Kunden und Dienstleistern nicht klar definiert werden kann. Die unterschiedlichen 
Serviceorganisationen der Hochschule sind der Rolle der Dienstleister zuzuordnen. Auf 
Seiten der Kundensicht kommen Studierende, Hochschulen selbst, aber auch ihre Fakultäten, 
Fachbereiche, die Lehrenden und Verwaltungsmitarbeiter in Frage. Trotz alledem ist eine 
stärkere Serviceorientierung aufgrund steigenden Wettbewerbs um Studierende und 
potenziellen Forscher-Nachwuchs, veränderten Auswahlverhaltens und gestiegenem 
Anspruchsdenken der Studierenden notwendig.\footcite[Vgl.][14]{leitner_itil_2008}

\subsubsection{Realisierung der Serviceorientierung}
\label{realisierung_der_serviceorientierung}
Um das Wertversprechen gegenüber Studierenden weiter zu verbessern, ist die Optimierung der Serviceorientierung wichtig. Erreicht werden könnte dies beispielsweise durch eine Verbesserung der Bibliotheksangebote und weitreichendere Öffnungszeiten. Darüber hinaus können mittels verbesserter Lehrqualität neue berufs- und ergebnisorientierte Bedürfnisse der Studierenden befriedigt werden. Weiter lässt sich durch eine stärkere Einbindung von Praktikern als Gastdozierende und Mentoren praxisrelevantes Wissen vermitteln. Eine reibungslose Abwicklung und große Auswahl an institutionalisierten Austauschprogrammen und Auslandssemestern ist ebenfalls förderlich. Die individualisierte Karriereförderung sollte allerdings über die eigentliche Studienbetreuung hinaus gehen und personalisierte Bewerbungstrainings und Karrierecoachings, sowie persönliche Kontakte zu Arbeitgebern beinhalten. Um mit potentiellen Arbeitgebern frühzeitig in Kontakt treten zu können, sind Career Services und der Ausbau von Jobmessen wichtig.\footcite[Vgl.][13]{schroeder_2011}

Verbesserte Dienstleistungen werden von Studierenden hoch geschätzt. So ist besonders für berufstätige Studierende eine flexible Lehre wichtig. Dazu gehören ergänzend zum Präsenzunterricht E-Learning-Angebote (siehe Kapitel \ref{subsubsection_e_learning_plattformen}) aber auch Verfahrensweisen wie BYOD (siehe Kapitel \ref{netzinfrastruktur_consumerization_und_byod}). Serviceorientierung lässt sich weiter durch effiziente Gestaltung von Bewerbungsverfahren und bedienerfreundlichem Kursauswahlverfahren erzielen. \footcite[Vgl.][18]{deutsche_wissenschaft_2010}

Große Systemvielfalt beinhaltet viele Login-Prozesse und unterschiedliche Ansprechpartner. Hier ist das Ziel weniger Systeme  und mehr Serviceorientierung einzusetzen, um eine bessere Nutzerfreundlichkeit zu erreichen. Durch ein SSO (Single Sign-On) wird nach einer einmaligen Anmeldung an einem Portal wie in Abbildung \ref{fig_sso} dargestellt, ein genereller Zugriff auf alle Anwendungen gewährt. So können Tätigkeiten wie beispielsweise Prüfungsanmeldungen, Zugriffe auf Rechenzentren, Bibiliotheken oder die Verwaltung mit nur einem Login durchgeführt werden.\footcite[81]{deutsche_wissenschaft_2010}

\begin{figure}[h!]
	\centering
	\includegraphics[width=10cm]{kapitel/gruppe1_2/bilder/SSO}
	\caption{Single-Sign-On\protect\footnotemark}
	\label{fig_sso}
\end{figure}\footnotetext{\cite[81]{deutsche_wissenschaft_2010}}


\subsubsection{IT Infrastructure Library (ITIL)}
\label{subsubsection_ITIL}
Zur Fokussierung der IT-Dienste auf Kundenorientierung und für eine stärkere Ausrichtung des IT-Bereichs an strategischen Organisationszielen, stehen Hochschulen und anderen Institutionen verschiedene Referenzmodelle zur Verfügung. Diese Modelle unterstützen bei der Bereitstellung klar definierter IT-Services, einer kennzahlengestützten Steuerung und Bewertung des IT-Managements und Umstrukturierung der IT-Organisation. Die IT Infrastructure Library (kurz ITIL) ist das international am meisten genutzte Referenzmodell. Es ist aus einer Sammlung von Beispielen guter Praxis entstanden und wird stetig weiterentwickelt. In ITIL werden sämtliche Prozesse in Beziehung zueinander gesetzt und definiert. Dazu gehören beispielsweise Konfigurationsmanagement, Kapazitäts-, Verfügbarkeits- und Finanzplanung, der Umgang mit Katastrophen, Störungs- und Problembehandlung, aber auch Service Level Vereinbarungen. ITIL ist durch seine Skalierbarkeit und Prozessorientierung auf Gesamtorganisationen, einzelne Abteilungen oder übergreifende Dienstleistungen anwendbar. Die Prozesse können unabhängig von einer konkreten IT-Infrastruktur genutzt werden, wodurch der Einsatz in vielen Bereichen ermöglicht wird. \footcite[Vgl.][34]{leitner_itil_2008}

\paragraph{Servicedesk}\mbox{}\\	
\label{subsubsection_service_desk}
Die Schaffung eines Servicedesks resultiert aus dem Verständnis, die Studierenden und Lehrenden als „Kunde“ zu betrachten, denen man serviceorientierte Dienstleistungen anbieten möchte. Zum anderen wird eine effizientere Ressourceneinsatzplanung im Verwaltungsbereich ermöglicht. Der Servicedesk ist die zentrale Anlaufstelle für jegliche Belange. Hier wird im Zuge des 1st Level Supportes eine Lösung der Anfrage ohne Kontaktierung weiteren Fachpersonals versucht. Zusätzlich stellt diese Ebene eine schnelle Reaktionszeit bei Störungen sicher. Sollte eine Sofortlösung nicht möglich sein, werden die Anfragen über ein Ticketsystem sortiert, priorisiert und den entsprechenden Bearbeitern zugeteilt. Hier wird die Bearbeitung zeitversetzt durch Spezialisten im 2nd Level Support fortgeführt. In der Abbildung \ref{fig_service_desk} ist der beschriebene Ablauf visualisiert, es handelt sich hier um die Infrastruktur der Universität Freiburg. \footcite[Vgl.][5]{klug_2008}

\begin{figure}[h!]
	\centering
	\includegraphics[width=15cm]{kapitel/gruppe1_2/bilder/ServiceDesk}
	\caption{Servicedesk: Beispiel an der Universität Freiburg\protect\footnotemark}
	\label{fig_service_desk}
\end{figure}\footnotetext{\cite[15]{bode_2007}}



\paragraph{Change-Management}\mbox{}\\\\
Change-Management ist einer der ITIL-Prozesse und beschreibt wie auf Änderungsanfragen zu reagieren ist. Dabei durchläuft der Change (die Veränderung) nachfolgende feingranulare Aktivitäten:

\begin{enumerate}
	\item Change wird erfasst und klassifiziert
	\item Change wird bewertet und freigegeben
	\item Change wird bei Bedarf eskaliert
	\item Change wird implementiert
	\item Change wird getestet und abgenommen
	\item Change wird abgeschlossen
\end{enumerate}

Innerhalb des Prozesses gibt es die Rolle des Change Requestors, der die Anfrage stellt. Diese wird optimaler weise an einer einzigen Stelle wie den Servicedesk aufgegeben, um sicherzustellen, dass alle Anforderungen vollständig erfasst und zentral gebündelt sind. Der Change Manager klassifiziert, plant die Durchführung und gibt den Change frei. Eine Beurteilung der Änderungsanfrage erfolgt anhand einer hochschulweiten vereinbarten Einstufung, die eine Klassifizierung nach Typ, Risiko und Dringlichkeit ermöglicht. Ein Beispiel so einer Tabelle ist in Abbildung \ref{fig_typenvonchange} vorzufinden. Der Change Builder setzt die Veränderung letztendlich um und der Change Approver prüft und testet die Änderung. \footcite[Vgl.][48]{breiter_implementierung_2011}

\begin{tabular}{|c|c|c|} 
\hline 
\bf Change-Typ & \bf Beschreibung & \bf Genehmigung\\ \hline \hline
Normal Change & Normalfall & keine Genehmigung durch den Change Manager\\
\hline 
Security Change & Erforderlich für Änderungen, 
die nur einem bestimmten Anwenderkreis zugänglich gemacht werden sollen & keine Genehmigung durch den Change Manager\\
\hline
Notfall Change & Sofortige Freigabe und Bearbeitung & Genehmigung durch Change Manager nötig\\
\hline 
\end{tabular}


\begin{figure}[h!]
	\centering
	\includegraphics[width=15cm]{kapitel/gruppe1_2/bilder/typen_von_changes} 
	\caption{Typen von Changes\protect\footnotemark}
	\label{fig_typenvonchange}
\end{figure}\footnotetext{\cite[48]{breiter_implementierung_2011}}



\paragraph{Service Level Agreements}\mbox{}\\\\
„Service Level Agreements“ (SLAs) sind verbindliche Vereinbarungen zwischen dem Leistungsempfänger und dem Leistungserbringer. Die SLAs sichern die Bereitstellung von IT-Dienstleistungen, regeln die Dienstleistungsqualität und definieren die Preise für die Erbringung von Leistungen. Weiter werden auch Reaktionszeiten je nach Schweregrad definiert und Konventionalstrafen für den Fall von einer Überschreitung festgelegt. Betriebszeiten und Ausfallsicherheit wichtiger Infrastruktur sind ebenfalls Bestandteile solch einer Vereinbarung.
Zusammengefasst sind SLAs für Dienstleistungsempfänger ein wichtiges Instrument zur Sicherheit und Kenntnisnahme über den Leistungsumfang, die Leistungskosten, die minimale Leistungserbringung und der benötigten Reaktionszeit. Informationsmanager nehmen hierbei eine beratende Rolle ein und unterbreiten zusätzlich Vorschläge zur fachlichen Beschreibung von Zielvorgaben. \footcite[Vgl.][499]{heinrich_stelzer_2011}


\subsubsection{Chief-Information Officer (CIO)}
\label{cio_text}
Der Chief Information Officer (CIO) ist die Berufsbezeichnung für eine Person/Führungskraft, die verantwortlich für die Informationstechnik und Anwendungen einer Hochschule ist. \footcite[Vgl.][]{beuschel_2009}

\paragraph{Aufgaben und Funktionen des Informationsmanagers}\mbox{}\\\\
\label{aufgaben_funktionen_informationsmanager}
Die Aufgaben des CIO bestehen in der Entwicklung einer IT-Infrastruktur-Strategie und der Ausrichtung der IT auf die Unternehmensstrategie. Seine Tätigkeiten lassen sich im operativen Geschäft auf 3 Kernaufgaben festlegen: 
\begin{enumerate}
	\item Das Planen und Implementieren von Software- und Hardware-Architekturen 
	\item Priorisierung neuer Steuerungsprozessen, sowie neuer Anwendungen
	\item Bereichsübergreifende Koordination
\end{enumerate}

Im Rahmen des Plan-Do-Check-Act-Zyklusses (PDCA), siehe Kapitel  \ref{subsubsection_kontinuierlicher_verbesserungsprozess},  sollte er kontinuierliche strategische Vorschläge unterbreiten, wie Informationen zur Zielerreichung und Gewinnmaximierung innerhalb der Hochschule eingesetzt werden sollten. Außerdem hat er Informationen auf die Hochschulkultur und –praxis  abzustimmen und unter Berücksichtigung all dieser Aspekte individuell passende und benutzerfreundliche Werkzeuge auszuwählen. In seiner Verantwortung steht, dass erfolgsentscheidendes Wissen auch in schwierigen Situationen und bei hoher Fluktuation in der Hochschule erhalten bleibt. Organisationen, die über einen CIO mit solch einem Aufgabenprofil verfügen, sind selten. Sie sind eher in modernen Großkonzernen oder Unternehmen mit einer besonderen Affinität zu internetgestützter Kommunikation vorzufinden. \footcite[Vgl.][404]{becker_gora_uhrig_2012}
Aus dieser Aussage lässt sich für deutsche Hochschulen ableiten, dass der Einsatz eines CIOs gut überlegt sein muss. Kleine Hochschulen besitzen einen geringeren Kommunikationsbedarf, als große Hochschulen. Daher ist der Nutzen eines CIOs im Vorfeld gut zu prüfen. Das Kapitel \ref{section_einsatz_cio} wägt den Einsatz eines CIOs für die Hochschule Emden/Leer ab.


\paragraph{Anforderungsprofil}\mbox{}\\\\
\label{anforderungsprofil_informationsmanager}
Ein gutes Anforderungsprofil eines CIOs umfasst eine Mischung aus technischem Wissen, unternehmerischer Denkensweise und Managementfähigkeiten. Für die erfolgreiche Arbeit sind konzeptionelle und analytische Befähigungen aber auch Schlüsselqualifikationen wie Entscheidungsstärke, Organisationstalent, Teamfähigkeit, Führungsqualifikation, Kontaktfähigkeit, Kenntnisse im Projektmanagement und  in der strategischer Planung wichtig. In Abbildung \ref{efec} sind außerdem persönliche Merkmale wie Sensibilisierung und soziales Kompetenzwissen als Erfolgsfaktoren für einen CIO aufgeführt. Die Beziehungen zum CEO und ein Aufbau einer gemeinsamen Vision sind ausschlagegebend für die erfolgreiche Arbeit. Entscheidend ist auch das glaubwürdige Auftreten innerhalb der Hochschule.\footcite[Vgl.][150]{krcmar_einfuhrung_2015}

\begin{figure}[h!]
	\centering
	\includegraphics[width=15cm]{kapitel/gruppe1_2/bilder/erfolgsfaktoren_cio} 
	\caption{Erfolgsfaktoren für einen CIO\protect\footnotemark}
	\label{efec}
\end{figure}\footnotetext{\cite[150]{krcmar_einfuhrung_2015}}

\paragraph{Eingliederung in die Hochschulhierarchie}\mbox{}\\\\
Nachdem die grundsätzlichen Aufgaben eines Informationsmanagers erläutert wurden, ist noch die Ansiedlung dieses Postens innerhalb der Hochschulhierarchie zu klären. Das Spektrum reicht hier vom CIO mit Leitungsfunktion repräsentiert durch einen Vizepräsident bis hin zu einer kollektiven Teilung innerhalb des Lenkungsausschusses durch mehrere Personen. Im Detail wird hierauf in Kapitel \ref{subsubsection_zki} eingegangen, insgesamt werden 4 verschiedene Umsetzungstypen unterschieden. \footcite[Vgl.][10]{leitner_itil_2008}


\subsection{Prozessorientierung}
\label{subsection_prozessorientierung}
Die Prozessorientierung ist ein grundlegendes Konzept des Geschäftsprozessmanagements, worunter die Gestaltung, Ausführung und Beurteilung von Prozessen verstanden wird. Ein Prozess ist eine zusammenhängende Abfolge von Einzelfunktionen, zwischen denen logische Verbindungen bestehen, wie in Abbildung \ref{fig_aufgaben_vs_prozess} mit Pfeilen visualisiert wurde. \footcite[Vgl.][60]{krcmar_einfuhrung_2015} Weiter lässt sich aus der Abbildung anhand des Organigramms ablesen, dass die Gliederung der IT-Organisation an Hochschulen oft funktional aufgestellt ist, konkret zu erkennen an dem Netzwerk- und Systembetrieb, Nutzersupport (Servicedesk) oder dem Anwendungsmanagement. Diese Aufgabenorientierung erlaubt eine stärkere Spezialisierung in den jeweiligen Fachgebieten der Mitarbeiterinnen und Mitarbeiter, widerspricht aber dem Gedanken einer Prozessorientierung. Hier ist daher eine klare Definitionsabgrenzung durchzuführen, denn das Handeln in Prozessen erfordert eine Abkehr von aufgabenorientierten Verfahrensweisen. Prozessorientiert zu denken bedeutet, sich nicht nur auf eine Aufgabe zu konzentrieren, sondern den Gesamtkontext zu betrachten, sprich das Zusammenspiel und die Wechselwirkungen zwischen allen Einzelfunktionen eines Prozesses. Erst durch die Betrachtung der Verkettung einzelner Aufgaben werden nämlich komplexe und betriebswirtschaftliche Prozesse ersichtlich. \footcite[Vgl.][274]{heinrich_stelzer_2011}


\begin{figure}[h!]
	\centering
	\includegraphics[width=15cm]{kapitel/gruppe1_2/bilder/aufgaben-versus_prozessorientierung} 
	\caption{Aufgaben- versus Prozessorientierung\protect\footnotemark}
	\label{fig_aufgaben_vs_prozess}
\end{figure}\footnotetext{\cite[Vgl.][35]{leitner_itil_2008}}

Die Erreichung der Prozessorientierung kann auf unterschiedliche Weise erfolgen, zum Beispiel durch eine kontinuierliche Prozessverbesserung oder die Gestaltung und Anpassung von IT-Strukturen \footcite[Vgl.][45]{wissensmanagement_2010}. Im Rahmen des Kapitels \ref{subsubsection_kontinuierlicher_verbesserungsprozess}  wird der in der oberen Abbildung gezeigte kontinuierlicher Verbesserungsprozess beschrieben, evaluiert und letztendlich optimiert.


\subsubsection{Kontinuierlicher Verbesserungsprozess}
\label{subsubsection_kontinuierlicher_verbesserungsprozess}
Das Ziel des kontinuierlichen Verbesserungsprozesses ist die stetige Verbesserung von Zuständen in kleinen Schritten und die Wahrung der  Zustandsverbesserung, wie in Abbildung \ref{fig_kontinuierliche_verbesserung} gut veranschaulicht. Zur Umsetzung systematischer Verbesserungsmaßnahmen, wird ein in 4. Phasen aufgeteilter Regelkreis angewandt:

\begin{figure}[h!]
	\centering
	\includegraphics[width=5cm]{kapitel/gruppe1_2/bilder/kontinuierlicher_verbesserungsprozess} 
	\caption{Kontinuierlicher Verbesserungsprozess\protect\footnotemark}
	\label{fig_kontinuierliche_verbesserung}
\end{figure}\footnotetext{\cite{yasar_kvp_2015}}


In der Phase Plan wird sich die Frage gestellt, was und wie etwas zu tun ist. Auf die Prozessorientierung angewandt, lässt sich hier auf die Prozessdefinition und –analyse schließen. Die Phase Do beschäftigt sich mit der Frage was erreicht wurde und steht für die Ausführung, also sinnbildlich für die Prozesskonstruktion. Bei der Check-Phase geht man auf die Frage ein, was noch zu tun ist und ob die Aufgaben nach Plan erfüllt sind, ableitbar auf eine Prozessvalidierung. In der letzten Phase Act wird überprüft, welche Dinge verbessert werden können, das für eine Prozessoptimierung und –automatisierung spricht.\footcite{yasar_kvp_2015}



\paragraph{Prozessidentifizierung und -analyse}\mbox{}\\\\
\label{paragraph_prozessidenifizierung}
Aufgabe der Prozessidentifizierung ist es, Prozesse zu bestimmen und zu beschreiben, die mit hoher Priorität geplant, gesteuert und verbessert werden sollen. In der Prozessanalyse werden dann die einzelnen Elemente eines Prozesses und deren Beziehung untereinander bestimmt und beschrieben. \footcite[Vgl.][276]{heinrich_stelzer_2011}
Angewandt auf den Prozess in der Abbildung \ref{fig_aufgaben_vs_prozess} des Kapitel \ref{subsection_prozessorientierung} kann folgendes abgeleitet werden:
Der Anwender meldet eine Störung innerhalb einer Fachapplikation wie zum Beispiel der Studierendenverwaltung dem Servicedesk der Hochschule. Dieser analysiert, beschreibt und priorisiert den eingehenden Fall. Innerhalb des First-Level-Supportes und bestehender Fehlerprotokolle/-dokumentationen wird versucht, eine Sofortlösung zu erzielen. Ist dies nicht erfolgsversprechend, wird bei einer fehlerhaften Serverkonfiguration der Server-Administrator verständigt. Dieser entdeckt bei seiner Untersuchung ein fehlendes Update der Fachapplikation und beauftragt damit die Beschaffungsabteilung. Das Einspielen des Updates wird durch das Anwendungsmanagement auf einem Testsystem durchgeführt, die sich anschließend zwecks Qualitätssicherung mit der Testgruppe zur Validierung abstimmt. Nach Freigabe durch die Betriebsleitung kann die Aktualisierung auf dem Produktivsystem eingespielt werden.
Ein in der Abbildung nicht aufgeführter möglicher Rückweg wäre: Nach Freigabe des Updates wird durch das Anwendungsmanagement die Installation auf dem Produktivsystem veranlasst. Der Servicedesk wird hierrüber nach erfolgreichem Abschluss informiert, der die Fehlerbehandlung protokolliert und den Endanwender über die Lösung der gemeldeten Störung unterrichtet. 


\paragraph{Prozesskonstruktion und –sichtbarkeit}\mbox{}\\\\
\label{paragraph_prozesskonstruktion_und_sichtbarkeit}
Um die im vorherigen Kapitel bei der Prozessanalysendefinition erwähnten Elemente sichtbar zu machen, werden Prozessketten verwendet. Diese eigenen sich, um den Ablauf bestehender Prozesse und die Beziehung der einzelnen Elemente untereinander zu visualisieren. Aber nicht nur der Ist-Prozess, sondern auch der Soll-Prozess kann mittels Prozessketten modelliert werden. Zur Verfügung stehen unterschiedliche Modellierungselemente, beispielsweise ein Rechteck zur Symbolisierung einer Funktion, eine Ellipse als organisatorisches Element (Prozessstart, Prozessende) oder Pfeile, die einen Informationsfluss veranschaulichen. \footcite[Vgl.][64]{krcmar_einfuhrung_2015} Zur Steigerung der Prozesseffizienz wurden in Kapitel \ref{subsubsection_prozessoptimierung_durch_minierung_der_durchlaufzeiten} unterschiedliche Durchlaufoptimierungen aufgeführt: Weglassen, Auslagern, Zusammen fassen, Parallelisieren, Verlagern, Beschleunigen, Keine Schleifen und Ergänzen.

Im Sinne des kontinuierlichen Verbesserungsprozesses und der in diesem Kapitel verwiesenen Prozesskonstruktion wurde im Anhang in der Abbildung XXXX\todo[inline]{OS: Referenz aktualisieren} der im Kapitel \ref{paragraph_prozessidenifizierung} beschriebene Prozess der Fehlerbehandlung einer Fachapplikation mittels Prozesskette abgebildet. In der Zeilenbeschreibung sind die Zuständigen der jeweiligen Aufgaben aufgeführt. Der Prozess startet in der ersten Zeile bei „Start“ und ist der vorgegebenen Pfeilrichtung entsprechend zu lesen bis hin zum Prozessende.

\paragraph{Prozessevaluierung}\mbox{}\\\\
„Der wesentliche Zweck der Prozessevaluierung besteht darin, zu überprüfen, ob ein Geschäftsprozess gemäß den Vorgaben ausgeführt wird. Relevante Vorgaben können in Prozessentwürfen, Verfahrensanweisungen und Arbeitsanleitungen beschrieben sein“. \footcite[277]{heinrich_stelzer_2011} Es wird sich die Frage gestellt, ob die Prozesse so ausgeführt werden, wie sie in den Prozessmodellen beschrieben wurden und ob der Geschäftsprozess der kontinuierlichen Verbesserung unterliegt. \footcite[Vgl.][277]{heinrich_stelzer_2011} In unserem Beispiel wurde die Prozessmodellierung auf Basis der Prozessbeschreibung erstellt, wodurch die Prozessevaluierung positiv abschließt. Diese Evaluierung muss natürlich in regelmäßigen Abständen wiederholt werden, um zu überprüfen, ob der Gesamtprozess noch nach Plan läuft. Trotz positiver Bewertung kann auch unser Beispielprozess von einer Prozessoptimierung profitieren.

\paragraph{Prozessoptimierung}\mbox{}\\\\
Die Prozessoptimierung bezeichnet alle Maßnahmen zur Veränderung von Prozessen, um die Kosten zu senken, Durchlaufzeiten zu verkürzen, Innovationsfähigkeit zu erhöhen oder die Qualität zu steigern. \footcite[Vgl.][280]{heinrich_stelzer_2011}
Die im Kapitel \ref{paragraph_prozesskonstruktion_und_sichtbarkeit} gewonnenen Erkenntnisse zur Steigerung der Prozesseffizienz wurden auf unseren Fall angewandt und mündeten in einer optimierten Prozesskette. Dieses Kapitel konzentriert sich aufgrund der Komplexität auf den ersten Teilprozess, sprich die Meldung des Fehlers bis zur Freigabe des Updates. Der Rückweg in Form der Installation auf dem Produktivsystem und der Erfolgsmeldung an den Kunden ist der Vollständigkeit halber im Anhang in Abbildung XXXX\todo[inline]{OS: Referenz aktualisieren} aufgeführt, ist aber nicht Bestandteil dieser Ausarbeitung. 
Die erste Spalte der Abbildung \ref{fig_prozessoptimierung} zeichnet den aktuellen Ist-Zustand auf. Das Ergebnis einer Prozessoptimierung ist in der zweiten Spalte erkennbar, auf das nachfolgend weiter eingegangen wird. Damit die Innovation Einzug in der Hochschule hält, wäre eine übergreifende Service- und Programmüberwachung denkbar. Sie ermöglicht das entdecken und identifizieren von Fehlern vor der Meldung durch einen Endanwender. Parallel dazu wird ein Automatismus geschaffen, der neue Updates für alle Fachapplikationen sucht und bei Entdeckung an die übergreifende Überwachungssoftware meldet. Wird ein neues Update festgestellt, wird der Server-Administrator informiert, um die Serverkonfiguration zu prüfen. Es ist sinnvoll, seine Kompetenzen um das Einspielen von Updates für Applikationen zu erweitern. Nur spezifische Störungen innerhalb der Anwendung oder konkrete Nachfragen zur Bedienung sollten an das Anwendungsmanagement weitergeleitet werden.  Die Testgruppe prüft anhand vordefinierter Testfälle die Funktionsfähigkeit der Anwendung. Der Betriebsleiter gibt das Update nach positivem Testfeedback frei.
Um die einzelnen Maßnahmen zur Prozessoptimierung aufzugreifen, sind im Schaubild Kreise mit Zahlen aufgeführt:

\begin{enumerate}
	\item Maßnahme Weglassen: Im Optimalfall wird der Endanwender von einer Störung nichts mitbekommen
	\item Maßnahme Parallelisierung: Das Laden von Updates wird automatisiert und findet parallel zur Programmüberwachung statt. So wird eine verkürzte Durchlaufzeit erzielt, da der Server-Administrator sofort informiert und auf bereits heruntergeladene Updates zugreifen kann.
	\item Maßnahme Ergänzen: Durch die Einführung einer übergreifenden Überwachungssoftware wird ein neuer Teilprozess ergänzt.
	\item Maßnahme Zusammenführen: Der Serveradministrator prüft nicht nur Serverkonfigurationen, sondern spielt auch Updates ein
	\item Maßnahme Auslagern: Die Fachabteilung Beschaffung muss die Updates nicht mehr selbst herunterladen, ein Automatismus auf einem Server übernimmt diese Tätigkeit.
\end{enumerate}


\begin{figure}[h!]
	\centering
	\includegraphics[width=13cm]{kapitel/gruppe1_2/bilder/prozessoptimierung} 
	\caption{Prozessoptimierung für eine Fehlerbehandlung}
	\label{fig_prozessoptimierung}
\end{figure}

\todo[inline]{AW: Positionierung er Grafik Prozessoptimierung unter gleichnamigem Kapitel}


\subsubsection{Gestaltung und Anpassung von IT-Strukturen}
\label{subsubsection_gestaltung_IT_strukturen}
Die Gestaltung und Anpassung von IT-Strukturen ist zur Prozessoptimierung durch Zentralisierung, Standardisierung und Outsourcing erreichbar.


\paragraph{Zentralisierung}\mbox{}\\\\
Im Hochschulbereich haben sich einige Lehrstühle und Institute ihre eigene IT-Abteilung geschaffen. Dies gilt beispielsweise für viele Leiter von Forschungsprojekten, für die Verwaltung und die Bibliothek, die eigene IT-Dienstleistungen erbringen. Das hohe Maß an Dezentralisierung der IT-Betriebsorganisationen führt zu einer Redundanz der IT-Service-Erbringung. Es ließe sich ein Parallelaufbau von betriebsrelevanter Infrastruktur wie Netz- und Stromversorgung, Belüftung und Klimatisierung vermeiden.
\footcite[Vgl.][22]{stratmann_it_2013}. Auch das doppelte Bereitstellen von beispielsweise Mailservices oder Groupware ist nicht sinnvoll. Des Weiteren wird mit diesen Standard-IT-Dienstleistungen mehrfach Personal gebunden, das mit der zunehmenden Komplexität der Basisdienstleistungen oft überfordert ist. 
Die Institute können sich nicht selbst auf allen Ebenen mit hochwertiger IT-Dienst-Betreuung befassen. Die vielen Insellösungen sind zusätzlich unwirtschaftlich und für eine hochschulweite Integration des Informationsmanagements oft hinderlich. Die Institute müssen Strategien entwickeln, um gemeinsame Synergieeffekte zu nutzen und die begrenzten IT-Betreuungsressourcen sinnvoll einzusetzen.
Die Zentralisierung von Diensten ermöglicht eine einfach zu koordinierende Beschaffung von Hard- und Software. Alle Systeme sind durch die zentrale Planung und Einbettung gut aufeinander abgestimmt und ergeben größere Ausfallsicherheit mit hoher Verfügbarkeit. Das stärkt die Stabilität und Robustheit des IT-Gesamtsystems. Die Redundanz in dem Personaleinsatz und der Serviceerbringung entfällt. \footcite[Vgl.][22]{moenkediek_2006}


\paragraph{Standardisierung}\mbox{}\\\\
Unter der Standardisierung in Hochschulen wird die einheitliche Nutzung von Basisdiensten und Grundfunktionalitäten verstanden. Konkret soll die Vereinheitlichung von Anwendungsprogrammen, Prüfungsordnungen und IT-Infrastrukturen in den Fachbereichen erzielt werden. Über ein Softwareverteilungstool kann eine gleiche Version aller Applikationen sichergestellt werden. Zur Realisierung von einheitlicher IT-Infrastruktur wäre eine Zentralisierung der Serviceleistungen denkbar, wie im vorherigen Kapitel beschrieben. Die Einführung von ITIL-Standardprozessen wäre ein möglicher Weg der Umsetzung und mittels SLAs könnten auch die Reaktionszeiten auf Fehlermeldungen festgelegt werden. Ein Informationsmanager (siehe Kapitel \ref{subsection_cio}) würde für eine kontinuierliche Einführung und Einhaltung der Standards in allen Fachbereichen Sorge tragen. Eine Zertifizierung nach standardisierten Normen (ISO 200000), wird in den kommenden Jahren ebenfalls an Bedeutung gewinnen.\footcite[Vgl.][168]{breiter_implementierung_2011}


\paragraph{Outsourcing}\mbox{}\\\\
Outsourcing besteht aus den Wörtern \glqq Outside\grqq{}, \glqq Ressource\grqq{} und \glqq Using\grqq{}. Gemeint ist damit, dass einzelne Aufgaben der IT, wie bspw. Infrastruktur, Applikationen, Prozesse, Personal oder gesamte 
IT-Aufgaben, auf Basis einer vertraglichen Vereinbarung, für einen definierten Zeitraum an einen 
externen Anbieter ausgelagert werden.\footcite[Vgl.][164]{krcmar_einfuhrung_2015} Konkrete 
Beispiele im Bereich der Informationstechnologie für Hochschulen wären der Betrieb des 
Rechenzentrums, der Anwendungsentwicklung oder der Telekommunikationsnetzwerke an andere 
Unternehmen abzugeben. Der Informationsmanager verspricht sich einen besseren Zugriff auf 
notwendige Ressourcen, Verlagerung möglicher Risiken und transparentere Ausgaben durch eine 
Kooperation mit Outsourcing-Gebern. Dagegen stehen erhöhter Koordinationsaufwand, komplizierte 
Vertragsgestaltungen und räumliche / zeitliche Distanz und damit fehlendes Vertrauen in den neuen 
Kooperationspartner.\footcite[Vgl.][195 ff.]{barthelemy_2001}

\subsection{Konklusion Serviceorientierung und Prozessorientierung}
Schlussfolgend aus der Definition der Serviceorientierung und der Prozessorientierung lässt sich ableiten, dass sich beide Orientierungen nicht ausschließen, sondern ergänzen.  Dienstleistungsorientierung bedeutet, dass die Bereitstellung von Informationssystemen als Leistung und Dienst am Kunden selbst verstanden und gesteuert wird. Genau dies ergänzt die Prozessorientierung mit optimierten Prozessen zur Erbringung dieser Leistungen. 
Die Hochschule Emden/Leer setzt beide Orientierungen ein, möchte sich aber laut dem Hochschulrechenzentrumsleiter Günter Müller in Richtung Prozessorientierung weiterentwickeln. \footcite{gunter_muller_interview} So empfieht es sich auf eine effektive und effiziente Gestaltung der Dienstleistungen zu konzentrieren und durch Standardisierung einheitliche Prozesse zu schaffen.


\section{Neue Medien - AH}
\textit{Autor: Aurelian Hermand}


Hochschultrends in den neuen Medien umfassen den online Auftritt bezüglich der Erstellung, Verarbeitung und Bereitstellung von Informationen und ferner die ganzheitliche Außendarstellung mittels Marketing. Die Imagebildung der Hochschulen folgt dabei festgelegten Prinzipien und Zielen. Marketing kann nicht leicht abgegrenzt werden, denn man kann nicht nicht kommunizieren. Von der serviceorientierten und prozessorientierten Organisationsstruktur der Hochschule bis hin zum direkten Online-Marketing über den Webauftritt und soziale Plattformen werden Studenten und Hochschulangehörigen konfrontiert mit neuen Medien. Im Zuge der Consumerisation haben sich die Grenzen der Nutzung von privater und beruflicher Software und Geräte aufgelöst. Eine homogen gestaltete Infrastruktur ist damit hinfällig. Der Trend zum BYOD (Bring Your Own Device) hat veranlasst, dass eine Infrastruktur flexibel gestaltet wird und werden muss. Insbesondere auch die gestiegene Nutzung von Mobilgeräten wie Smartphones und Tablets dazu geführt, dass Software neuen Vorgaben gerecht werden muss.


\subsection{Infrastruktur und Management}
Im folgenden werden Trends im Bezug auf die Infrastruktur an Hochschulen und im speziellen an der Hochschule Emden / Leer betrachtet. 

\subsubsection{Netzinfrastruktur, Consumeration und BYOD}
\label{netzinfrastruktur_consumeration_und_byod}
Viele Hochschulen in Europa haben sich mittlerweile dem Projekt eduroam (education roaming) angeschlossen. Eduroam ermöglicht die Nutzung von WLAN und LAN an jeder teilnehmenden Hochschule durch vorherige Authentifizierung mit den erhaltenen Zugangsdaten.

Eduroam als standardisierte Lösung macht die Teilnahme verschiedenster Geräte und Standorte im Zuge der Consumeration und BYOD sehr einfach. Geringere Support-Aufwendungen auf der anderen Seite ermöglichen es die Serviceorientierung teilweise abzulösen und durch Prozessorientierung zu ersetzen. Der Aufwand erhöht sich jedoch für die Administratoren durch "die Offenheit der Gerätewahl"\footnote{\url{http://www.wickhill.de/theguardian/byod-viele-vorteile-bei-einhaltung-von-sicherheitsspielregeln/}}, wobei der Sicherheitsaspekt hier eine große Rolle spielt. An den Standorten selber kommt es in Folge der Offenheit für die Benutzer wiederum so zur Steigerung der Produktivität und Benutzerzufriedenheit. Die Arbeitsabläufe werden hierdurch flüssiger und effizienter.

\subsubsection{Software für Forschung und Lehre}
Das Netzwerk Dreamspark ermöglicht es Studenten und Bedienstete im Rahmen von Forschung und Lehre kostenlos eine Vielzahl von Microsoft Softwareprodukten zu erhalten und auf Ihren Geräten zu installieren.

Die Hochschulen Osnabrück und Hannover, wie auch die Hochschule Emden / Leer ermöglichen die Teilnahme.

Der Service kann mit Hinweisen auf kostenlose Software für Forschung und Lehre erweitert werden. So bietet JetBrain  ebenfalls ein Teil seiner Software kostenlos Studenten an \footnote{\url{https://www.jetbrains.com/student/}}. Ebenfalls bietet auch GitHub ein Paket für Studenten \footnote{\url{https://education.github.com/}}.


\subsubsection{Identitätsmanagement}
Ein umfassendes Identitätsmanagement setzt eine komplexe Architektur voraus. Die zwei Hauptfunktionen des Identitätsmanagement klassifiziert sich in:

\begin{itemize}
  \item Neuen Benutzern
  \item Entfernung von Benutzern \ldots
\end{itemize}


Verfolgt wird der Trend, die Einrichtung und Löschung schnell und zentral erledigen zu können. Dabei keinen Benutzer-Daten klar und nachvollziehbar zu hinterlegen. \footnote{\url{https://www.rrzn.uni-hannover.de/fileadmin/it_sicherheit/pdf/SiTaWS08-idm.pdf}}

Die Einmalanmeldung (Single-Sign-On) ermöglicht dem Benutzer alle angeschlossenen Dienste einer Hochschule zu nutzen, ohne sich mehrfach Authentifizieren zu müssen. An der HS Emden/Leer wird dabei auf Shibboleth gesetzt. 

\subsubsection{E-Mail}
Ein integraler Bestandteil an Hochschulen ist der E-Mail Service. Der Trend geht zu einem zentralen System für Mitarbeiter und Studenten. Im Sinne des Consumeration ist die Nutzung des E-Mail Service offen gestaltet für alle vorstellbaren Endgeräte und Programme. Zudem gibt es Webmailer um auch von Fremdrechnern an die E-Mails zu gelangen.

\subsubsection{E-Learning Plattform}
\label{subsubsection_e_learning_plattformen}
Das elektronische Lernen setzt auf den Einsatz von digitaler Medien.
Das Blended Learning vereint 3 Lerndomänen, das lernen online durch Kommunikation, Distanziert ohne Interaktion und in der Präsenz. Auch im Präsenzstudium wird zunehmend nicht nur auf offline Medien Skripte oder Bibliothek gesetzt, sondern auch auf online Abgaben, Aufzeichnung, Videochats und Foren.

Dazu gibt es zwei größere Plattformen ILIAS und Moodle. Die Hochschule setzt Moodle ein, dass jedoch weitgehend nicht verpflichtend im Präsenzstudium ist. Als Angebot könnten, wie im Online-Studium Skripte zur Verfügung gestellt und weiterentwickelt werden, welches die Lehrenden übernehmen verwenden können. Der Vorteil liegt darin, dass um ein Skript herum ein Ökosystem aus Übungsaufgaben, Videos und Beiträgen entsteht.


\subsection{Dokumentenverwaltung}
Die Dokumentenverwaltung umfasst verschiedene Services, da die Anforderungen sehr unterschiedlich sind.


\subsubsection{Wiki}
Das Wiki ist ein Dienst zur Erfassung ungeordneter, miteinander verknüpfbarer Texte. Sie sind sehr flexibel einsetzbar. Es lassen sich Informationen schnell, versionsbasiert gemeinsam zusammentragen.

Wikis stellen oft die Basis für Informationsverwaltungen, aus denen konzentriertere Informationssysteme entstehen können in Form von Websites oder auch FAQs, Anleitungen uvm.



\subsubsection{Clouds und Big Data}
Clouds ermöglichen den einfachen Datenaustausch großer Dateien mit verschiedenen Zugriffsrechten. Eingeteilt werden können die Clouds in


\begin{itemize}
  \item Öffentliche Clouds
  \item Private Clouds
  \item Hybride Clouds
  \item Community Clouds \ldots
\end{itemize}

Die Community Cloud stellt einem definierten Nutzerkreis von mehreren Standorten Zugriff auf die Cloud zur Verfügung. Hierbei wird gemeinsam oder von einem Anbieter die die Cloud verwaltet. Die Hochschulen in NRW als Verbund setzen auf diese Cloud-Form. Dahinter steckt die Software ownCloud. Die Lösung nennt sich Sciebo die CampusCloud.

Die Hochschule Emden / Leer führt derzeit mit Hilfe des Shibboleth-Dienstes eine Cloud namens „Gigamove“ zum Austausch großer Datenmengen ein. Gigamove wird von der RWTH Aachen zur Verfügung gestellt \footnote{\url{https://gigamove.rz.rwth-aachen.de}}. Die Spezifikationen dieser Cloud erlauben es Bediensteten der Hochschule 10 GByte mit Personen / Institutionen auszutauschen. Dabei ist es Möglich Dateien anderen zur Verfügung zu stellen, als auch anderen Platz einzuräumen. Nach 14 Tagen werden die Daten automatisch gelöscht und ist daher eine Austauschplattform und kein Massenspeicher.

\subsubsection{Versionsverwaltung}
Die Versionsverwaltung dient allgemein zur Dokumentenerstellung, -bearbeitung und -verwaltung. Dabei ist es jederzeit möglich auf einen vorhergehenden Stand zurückzusetzen oder Änderungen nachvollziehen zu können, damit auch ein gemeinsames Arbeiten an einem Dokument möglich ist. Die Uni Kassel setzt auf das DMS (Dokumentenmanagement) Alfresco. Alfresco ist ein bequemes und unkompliziertes System mit dem verschiedene Dokumente und Dateien zentral verwaltet werden können. Diese Software bietet Features wie Benutzerverwaltung, Integration in Moodle, Workflows zur Dokumentenüberprüfung, Aufgabenverteilung, Zusammenfassung, Versionierung von Office- oder PDF-Dokumenten. Außerdem steht eine App für Mobilgeräte bereit.\footnote{\url{http://www.uni-kassel.de/its-handbuch/kommunikation/dms/dokumentenmanagement-dms.html}}


\subsubsection{Zentrales Druckzentrum}
Das ZIV (Zentrum für Informationsverarbeitung) der Uni Münster zentralisiert u.a. die Rechnerräume, aber unterhält auch ein Druckzentrum. Das Druckzentrum bietet den Service von zu Hause zu drucken, dies schließt den Mobilgeräte ein. Die Drucke landen in einem zugeordneten Postfach mit einem farbigen Deckblatt und können von dort zu einem späteren Zeitpunkt aus dem Fach genommen werden können.
\footnote{\url{https://www.uni-muenster.de/imperia/md/content/ziv/pdf/printpay_flyer.pdf}}

\subsection{Außendarstellung und Marketing Instrumente}
Das Marketing und die Präsentation der Hochschule erfolgt breit gefächert und geht im Idealfall fließend ineinander über. Die Trends erfolgen oft in Organisatorischen Maßnahmen \footnote{\url{http://www.dfg.de/download/pdf/dfg_im_profil/gremien/hauptausschuss/it_infrastruktur/dfg_tum_bode.pdf, S.4f.}}. D.h. es wird am Ausbau und Vereinheitlichung gearbeitet, im Sinne von Corporate Identity bzw. Corporate Design der Webdienste, E-Learning Plattform, zentrale Datenspeicher, Verwaltungs EDV und sonstigen Angeboten.

\subsubsection{Website}
Die Website ist ein integraler Bestandteil der Hochschulen. Alle relevanten Informationen werden hierfür aufbereitet und dem Benutzer zugänglich gemacht. Der technische Fortschritt, verlangt zudem Beachtung neuer Designkriterien um die Sichtbarkeit im Internet zu gewährleisten.

\paragraph{Responsive Website}\mbox{}\\ % <-- bei Verwendung des Paragraph-tags bitte beachten.
Responsivität im Webdesign heißt, dass im Sinne des BYOD, der Zugriff auf die Hochschul-Website komfortabel und geräteunabhängig gestaltet ist. Die Fachhochschulen Köln und Münster sind dem Trend gefolgt, jedoch ohne auf einen etablierten Marktstandard zu setzen.

Es gibt zwei sehr verbreitete Frameworks, die meist aus Sammlungen von Modulen, Grids und Best-Practices bestehen, wie dem Prinzip des Mobile First. Mobile First bedeutet, dass aus Gründen des meist kleineren Bildschirms der Fokus auf den Inhalt liegt. Hiermit wird auch gleichzeitig das Prinzip des "Content First" bzw. "User First" umgesetzt. Sowohl Bootstrap von Twitter als auch Foundation von Zurb gelten als ausgereifte Frameworks. Die Verständigung auf ein ausgereiftes System, kann eine kostenintensive und proprietäre Selbstentwicklung verhindern. Twitter Bootstrap wird bspw. von der Hochschule Coburg und der TU München eingesetzt.

Die responsive Umsetzung mit Mobile First erhöht deutlich die Gebrauchstauglichkeit (engl. Usability), weil die Website auf dem Mobilgerät nicht gezoomt werden muss und so ausgeliefert wie der Designer es konzipiert hat. 

Die neueste Entwicklung im Bereich responsive Umsetzung erfolgte durch die Änderung des Such-Algorithmus von Google im April 2015. Die Änderung betrifft die Bewertung mobil optimierte Websites in den Suchergenissen, die fortan bevorzugt behandelt werden, sofern über ein Mobilgerät gesucht wird.

\paragraph{Sichtbarkeit und SEM}\mbox{}\\ % <-- bei Verwendung des Paragraph-tags bitte beachten.
Das Suchmaschinen-Marketing wird zusammengefasst unter dem Kürzel SEM (Search-Engine-Marketing). SEM umfasst die Konzepte SEO (Search-Engine-Optimization) und das SEA (Search-Engine-Advertising). \footnote{\url{B2B-Online-Marketing und Social Media, S.83}}

Die Suchmaschinen-Werbung bzw. SEA wird genutzt um gezielt bestimmte Suchbegriff gegen Bezahlung auf den ersten Seiten der Suchmaschinen als Werbung einzublenden.

Darunter SEO versteht man die allgemeine Suchmaschinen-Optimierungs-Maßnahmen, um im organischen Ranking weit vorne zu landen.

Der Sichtbarkeitsindex dient als Indikator für die Sichtbarkeit einer Website im Google Ranking. Dabei errechnet sich der Index aus:


\begin{itemize}
  \item dem Ranking der thematisch überwachten Keywords
  \item dem zu erwartenden Traffic aus der Positionierung und
  \item dem zu erwartenden Traffic aus dem Keyword. \ldots
\end{itemize}

Der Sichtbarkeitsindex wird als ein weiterer Messwert herangezogen für den Erfolg von SEO-Maßnahmen, neben u.a. Zugriffszahlen und Verweildauer der Besucher. \footnote{\url{https://de.onpage.org/wiki/Sichtbarkeitsindex}}

Die www.hs-emden-leer.de erreicht laut SISTRIX im Mai 2015 einen Sichtbarkeitsindex von 0,31. Im Vergleich dazu erreicht www.hs-coburg.de 0,62 und www.jade-hs.de 0,69. \footnote{\url{http://www.sichtbarkeitsindex.de/}} Die Hochschule Emden / Leer hat demnach noch Potential nach oben.

\paragraph{Inhaltsaufbereitung}\mbox{}\\ % <-- bei Verwendung des Paragraph-tags bitte beachten.
Der wichtigste Teil einer Website ist der Inhalt selber, der Fokus hierin liegt auf Vollständigkeit und einer verständlichen Sprache. Die Inhalte werden Aufbereitung und andere Ausgabemedien, wie zum Beispiel in sozialen Medien, PDFs, Drucklayouts, XML-Sitemaps und RSS ausgegeben.

RSS ist ein XML-Format zur Übertragung vorallem von News und Informationen. Die HS Emden/Leer setzt RSS an vielen Stellen ein, wie dem InfoSys und den nächsten Terminen.

\subsubsection{Social Media}
"Social-Media-Marketing (SMM) ist eine Form des Online-Marketings, die Branding- und Vertriebsziele durch ein Engagement in einem oder in verschiedenen sogenannten Social- Media-Angeboten erreichen will." \footnote{\url{Praxiswissen Online-Marketing - S.31}}

Newsletter Kampagnen sind ein trotz vieler neuer Medien weiterhin ein wichtiger Baustein im Online-Marketing-Mix. Es gibt einen klaren Trend in Richtung hin zu Mobilgeräten. Die Öffnungsraten auf mobilen Endgeräten sind seit 2010 bis 2013 um 300 Prozent angestiegen und übertreffen mittlerweile auch die Öffnungsraten der gewöhnlichen Desktop-Geräte. Betreiber des E-Mail-Marketing setzen umso mehr auf die Optimierung der Kampagnen auf mobile Endgeräte \footnote{\url{Praxiswissen Online-Marketing - S.35}}. Um den Wiedererkennungseffekt zu fördern und die eigene Marke zu etablieren, sollte beim Marketing auf das Corporate Design gesetzt werden.

Dem Social-Media-Marketing stehen unzählige weitere Vertriebskanäle zur Verfügung, wie Facebook, Twitter oder YouTube. Wichtig ist dabei vorher ein Leitbild zu entwickeln und beizubehalten. \footnote{\url{http://www.hs-merseburg.de/fileadmin/_migrated/content_uploads/090219_Marketingkonzept_Final.pdf S.8}}

\begin{figure}[h!]
	\centering
	\includegraphics[width=10cm]{kapitel/gruppe1_2/bilder/nutzungsklassen}
	\caption{Nutzungsklassen und Anwendungsbeispiele sozialen Medien (Quelle: B2B Online-Marketing und Social Media, S. 152)}
	\label{fig_nutzungsklassen}
\end{figure}

Die Soziale Medien können auf vielfältige Weise genutzt werden. Die Nutzungsklassen, siehe Abbildung \ref{fig_nutzungsklassen} der sozialen Medien können in drei Bereiche aufgeteilen werden:


\begin{itemize}
  \item Kommunikation: Blogs, Microblogs, Soziale Netzwerke, Social-Bookmarking-Plattformen, Foren/Communities
  \item Content-Sharing: Text, Foto, Video, Audio
  \item Kooperation: Wikis, Bewertungs-/Auskunftsprotale, Kreativportale \ldots
\end{itemize}

Die Nutzungsklasse \ref{fig_nutzungsklassen} "Kommunikation" zielt darauf ab, aufbereitete Informationen über private und professionelle Netzwerke bereitzustellen und zu diskutieren.
Ähnlich der Nutzungsklasse "Kommunikation" zielt auch das "Content-Sharing" darauf Inhalte zu teilen über spezifische Media-Sharing Plattformen.
Bei der Nutzungsklasse "Kooperation" steht vorallem die gemeinsame Aufbereitung von Informationen im Mittelpunkt.

Social Media spielt für die Rekrutierung neuer und Erreichbarkeit bestehender HS-Interessierter eine wichtige Rolle. Hierüber können Angebote, Stellenausschreibungen geschehen. Diese können dann verlinkt und geteilt werden. 

Eine Integration in die Website sollte Datenschutzrechtlich vorgenommen werden, bspw. mit der 2-KLick-Technik.


\subsubsection{App als Informationssystem}
Der Trend geht zu Informationssystemen in Apps. Ein allgemeiner Trend dabei ist das Prinzip die Apps sowohl offline als auch online Verfügbar zu gestalten (Offline First).

\begin{figure}[h!]
	\centering
	\includegraphics[width=10cm]{kapitel/gruppe1_2/bilder/hsel-androidapp}
	\caption{Android App der HS Emden / Leer}
	\label{fig_hselandroidapp}
\end{figure}

Die Hochschule Emden / Leer ist dem Trend gefolgt und hat Anfang 2014 eine Android App im Rahmen einer Projektarbeit vorgestellt, siehe Abbildung \ref{fig_hselandroidapp}. Das Prinzip "Offline First" wurde dabei berücksichtigt. Hauptaugenmerk wurde dabei auf die Integration von InfoSys und die Individualisierungssmöglichkeiten der Studentenden gelegt, um den Stundenplan anzupassen. \footnote{\url{http://www.hs-emden-leer.de/aktuelles-termine/news/article/immer-up-to-date-dank-neuem-smartphone-app.html}}

Da die Entwicklung im Rahmen einer Projektarbeit vonstatten ging, wird es sehr wahrscheinlich bei dieser einen Version und dem einzigen Gerätetyp bleiben.

\begin{figure}[h!]
	\centering
	\includegraphics[width=10cm]{kapitel/gruppe1_2/bilder/marktanteile}
	\caption{Marktanteile der Betriebssysteme an der Smartphone-Nutzung in Deutschland von 2011 bis 2014}
	\label{fig_marktanteile}
\end{figure}
\footnote{\url{http://de.statista.com/statistik/daten/studie/170408/umfrage/marktanteile-der-betriebssysteme-fuer-smartphones-in-deutschland/} Statista, 2014}

An Hand der Marktanteile \ref{fig_marktanteile} werden u.a. 20 Prozent iOS Nutzer nicht berücksichtigt und ist nicht im Sinne von BYOD, da eine Beschränkung vorliegt. Der Grund dafür liegt an den Unterschieden der Betriebssysteme. Für jedes System muss prinzipiell eine eigene App entwickelt werden. Ein kostengünstiger Lösungsansatz ist der Einsatz ausgereifter Javascript Webapp-Frameworks, wie beispielsweise Sencha Touch und AngularJS. Die Apps lassen sich so mit jedem Gerät zunächst einmal als Website auf dem Mobilgerät öffnen und mit Hilfe von Cordova/Phonegap ist es weiterhin möglich diese Webapps in den wichtigsten App Stores auszuliefern.

Nicht nur Flexibilität im Bezug auf Geräteunabhängigkeit wird geschaffen, auch Hürden der Weiterentwicklung werden verringert, da auf ausgereifte Software gesetzt wird.


\paragraph{InfoSys und News}\mbox{}\\ % <-- bei Verwendung des Paragraph-tags bitte beachten.
An einigen Hochschulen, wie der Hochschule Heidelberg werden u.a. Hochschulinformationen und Aktuelle Nachrichten direkt über eine App ausgeliefert. Die Integration des InfoSys und der aktuellen Nachrichten der Hochschule sind vorhanden, jedoch existieren diese Informationen nur für Android Benutzer.

\paragraph{HIS (Notenzugriff, Stundenpläne)}\mbox{}\\ % <-- bei Verwendung des Paragraph-tags bitte beachten.
Die HAW Hamburg und auch die Hochschule Heidelberg ermöglicht in der App den Zugriff auf Stundenpläne, Raumpläne, Prüfungen und Noten \footnote{\url{https://itunes.apple.com/de/app/haw-hamburg/id670347114?mt=8}}. Die Hochschule Emden / Leer hat in der Android App nur den Zugriff auf die Stundenpläne. Ableiten lässt sich daraus, dass geprüft werden muss, ob das HIS, den Zugriff über eine Schnittstelle ermöglicht.

\paragraph{Mensa}\mbox{}\\ % <-- bei Verwendung des Paragraph-tags bitte beachten.
Hochschulen haben nicht selten entweder eine spezielle App nur für die Speisepläne oder haben die Speisepläne in der Hochschul-App integriert. Die Hochschule Emden / Leer hat derzeit keine spezielle Speiseplan-App. Das Studentenwerk Oldenburg bietet jedoch eine App für iOS an, bei der auch die Hochschule Emden / Leer integriert ist. Derzeit wird laut dem Studentenwerk Oldenburg an einer neuen Webapp für die Speisepläne entwickelt. \footnote{\url{http://itunes.com/apps/MensaplanOL}}

\paragraph{Gelände-Wegweiser IPS}\mbox{}\\ % <-- bei Verwendung des Paragraph-tags bitte beachten.
Ein Indoor Positioning System mit beispielsweise Beacons bzw. Triangulation ermöglicht die Standortbestimmung innerhalb von Gebäuden. Das Auffinden eines Raumes in unbekannten Gebäuden mit Hilfe dieser Technologie und einem mobilen Endgerät, wäre damit problemlos möglich. Die Uni Hohenheim bietet dieses Feature als "Hörsaal-Finder mit Live-Navigation" \footnote{\url{https://itunes.apple.com/de/app/universitat-hohenheim-die/id490603166?mt=8}} siehe Abbildung \ref{fig_livenavi}.

\begin{figure}[h!]
	\centering
	\includegraphics[width=10cm]{kapitel/gruppe1_2/bilder/nutzungsklassen}
	\caption{Hörsaal-Finder der Uni Hohenheim}
	\label{fig_livenavi}
\end{figure}

\section{Qualitätsmanagement der Informationsprozesse - MiB}
\textit{Autor: Miriam Börger}

Im Folgenden wird der Qualitätsmanagement-Prozess von Informationsflüssen in seinen Grundzügen definiert, 
am konkreten Beispiel der Minimierung von Durchlaufzeiten genauer betrachtet und 
praktisch mit Hilfe der IT Balanced Scorecard durchexerziert. 
Abschließend wird erörtert, welche Besonderheiten hierbei an Hochschulen bestehen 
und Möglichkeiten aufgezeigt, diese Schwierigkeiten zu umgehen.

\subsection{Aufgaben des Qualitätsmanagements}
Aufbauend auf das in Kapitel \ref{subsection_management_schnittstellen_infoempfangern} beschriebene Qualitätsmanagement der Informationen 
wird nachfolgend das Qualitätsmanagement der Informationsprozesse genauer erörtert, was 
einen zentralen Aufgabenbereich in einem Informationsmanagement bildet.

Es übernimmt die Planung, Koordination und Steuerung der Informationsflüsse und prüft fortwährend, inwieweit eine Nutzbarkeit und Effizienz der Prozesse in der Realität gewährleistet ist, um deren Qualität gegebenenfalls mit gezielten Maßnahmen zu optimieren.\footcite[34 ff.]{schroder_wertorientiertes_2005}

Hierzu fungiert ein Team von Qualitätsmanagern als Vermittler zwischen den verschiedenen Parteien im Unternehmen und überbrückt potentiell auftretende Kommunikations- oder Kulturbarrieren, um eine zielorientierte und effiziente Informationsversorgung der beteiligten Parteien zu ermöglichen.

Zu Beginn des Qualitätsmanagement-Prozesses gilt es, eine Leitstrategie aufzustellen. Hierfür wird der aktuelle Ist-Zustand des Unternehmens in Bezug auf seine Organisation von Informationsflüssen analysiert. Dabei zum Vorschein kommende Schwachstellen werden erfasst und durch mögliche optimierende Handlungsoptionen ergänzt.\footcite{helmke_management_2013}

Während der Durchführung der neu erschaffenen Maßnahmen ist das 
Qualitätsmanagement-Team mit der stetigen Überwachung dieser betraut. 
Bereits bei kleinen Abweichungen vom Plan kann mit gegensteuernden Maßnahmen 
eingegriffen werden. Eine im Voraus aufgestellte Zeitplanung ist hierbei ebenso wichtig wie 
eine klare Definition der Zuständigkeiten im Qualitätsmanagement-Team, um eine 
termingerechte Erreichung der gesetzten Ziele noch zu garantieren.

Nach Ablauf des gesetzten Zeitrahmens oder nach Beendigung der Maßnahmen ist es erforderlich, mittels einer sogenannten Feedback-Analyse festzustellen, inwieweit das gesteckte Ziel erreicht wurde und aus welchen Gründen es nicht zu 100 Prozent zufriedenstellenden Ergebnissen kommen konnte.

Die hieraus resultierenden Erkenntnisse bilden daraufhin die Grundlage für eine anschließende Feedforward-Analyse, die die weitergehend erforderlichen Maßnahmen feststeckt, um in einer weiteren Phase die Zielerreichung durch verbesserte Maßnahmen zu garantieren.\footnote{\cite{gadatsch_it-controlling_2012}}

\subsection{Prozessoptimierung durch Minimierung der Durchlaufzeiten}
\label{subsubsection_prozessoptimierung_durch_minierung_der_durchlaufzeiten}
Essenzielles Ziel des Qualitätsmanagement-Teams ist es, anhand bewährter Vorgehensweisen die Durchlaufzeiten von Informationen zu minimieren. 
Hierdurch wird der Informationsfluss quantitativ und qualitativ verbessert, da bestehende Abhängigkeiten der Parteien in Bezug auf die Informationen schneller bedient werden können und somit durch minimierte Wartezeiten eine beträchtliche Budgetersparnis resultiert.

\begin{figure}[h!]
	\centering
	\includegraphics[width=\textwidth]{kapitel/gruppe1_1/bilder/minimierung_durchlaufzeiten}
	\caption{Minimierung von Durchlaufzeiten, nach Bleicher 1991, 196}
	\label{fig_minimierung_durchlaufzeiten}
\end{figure}
\newpage

Wie in Abbildung \ref{fig_minimierung_durchlaufzeiten} erkennbar, existieren elementare Methoden zur Reduktion von Durchlaufzeiten nach Bleicher aus dem Jahre 1991, die noch heute ihre Gültigkeit in der Anwendung haben.\footnote{\cite{bleicher_organisation_1991}}

Insbesondere das Zusammenfassen von Aktivitäten hat den entscheidenden Vorteil, dass Abstimmungsprozesse und Abhängigkeiten zwischen mehreren Parteien entfallen und somit die Umsetzungsdauer auf ein Minimum reduziert wird. 

Auch die Methode des Parallelisierens sollte in den Fokus gerückt werden. Wie in Abbildung \ref{fig_minimierung_durchlaufzeiten} ersichtlich, werden hierbei mehrere Parteien, die für eine darauffolgende Partei relevant sind, zeitgleich geschaltet, um Wartezeiten zu verhindern. 

Zu guter Letzt sei das Ergänzen von Prozessschritten betont. Auf den ersten Blick scheint diese Methode paradox, da durch Ergänzung weiterer Parteien der Zeit- und Arbeitsaufwand vorerst erhöht wird. Durch einen globaleren Blick wird schnell deutlich, dass ohne diese Parteien zu einem späteren Zeitpunkt Problematiken entstehen können, die in ihrer Lösung viel zeit- und arbeitsintensiver sind und das Unternehmen in seiner Prozessqualität deutlich zurückwerfen könnte.

Die in Abbildung \ref{fig_minimierung_durchlaufzeiten} gezeigten Methoden zur Durchlaufzeit-Minimierung werden in vollem Umfang in Kapitel 3.2.2.1.2 definiert.\todo[inline]{AW: Referenz setzen}

Sie sollten Qualitätsmanagement-Team von Beginn an in die Planung mit einbezogen werden, da mit minimalem Aufwand eine weitreichende, inhaltlich und finanziell positive Auswirkung auf die Qualität des Gesamtprozesses erzeugt wird. 

\subsection{Anwendung des Qualitätsmanagements am Beispiel der IT Balanced Scorecard}
Das strategisch-operative Konzept für eine qualitative Unternehmenssteuerung aus den 90er Jahren von R. S. 
Kaplan und D. P. Norton hat sich im Laufe der Zeit zum Standardinstrument 
entwickelt.\footcite{friedag_scorecard_2004}

Die Kombination von Qualität der Mitarbeiter, Kundenorientierung und finanzielle Ziele ermöglicht die 
Generierung und Sicherung eines gelungenen Informationsmanagements.\footcite{gabriel_inm_2003}

\begin{figure}[h!]
	\centering
	\includegraphics[width=\textwidth]{kapitel/gruppe1_1/bilder/balanced_scorecard}
	\caption{Balanced Scorecard Kreislauf, nach Gadatsch}
	\label{fig_balanced_scorecard_cycle}
\end{figure}
\newpage

Die IT Balanced Scorecard zeichnet sich – wie in Abbildung \ref{fig_balanced_scorecard_cycle} deutlich wird – durch eine stetige Feedback- und Feedforward-Kommunikation aus. 
Zu Beginn des Managementprozesses werden in der Phase \glqq Planung und Vorgaben\grqq{} die 
grundlegenden Ziele des Unternehmens erarbeitet. Hierbei wird der Fokus auf die Möglichkeiten der 
zukünftigen Verbesserung der Organisationsstruktur gelegt.

In einem nächsten Schritt werden in der Phase \glqq Vision und Strategie\grqq{} finanzielle Eckdaten 
Strategiefindung definiert. Als beispielhafte Kernfrage könnte hier: \glqq Wie können IT-Prozesskosten reduziert werden?\grqq{} genannt werden

Das Handlungskonzept wird in \glqq Feedback und Lernen\grqq{} final ausformuliert und in Prozesse 
unterteilt, die die Qualität und Geschwindigkeit der gewählten Maßnahme beschleunigen.

In der vierten Phase \glqq Kommunikation und Verbindung\grqq{} findet ein Abgleich statt, inwieweit die 
erarbeitete Strategie mit der Außendarstellung des Unternehmens in Einklang gebracht werden kann. Hierbei 
steht die Beurteilung von außen im Vordergrund.\footcite[35 ff.]{kaufmann_feinschliff_2002}

Die Definition von klaren Zielen, Bedingungen und Kennzahlen generiert ein komplexes Kennzahlensystem, 
welches durch Herunterbrechen der Strategie auf operatives Handeln einen ganzheitlichen Überblick über die 
interne Organisation des Unternehmens liefert. Die Einbeziehung von Ursache und Wirkung vereinfacht die 
vorausschauende Unternehmensführung und ergänzt die Sichtweise auf das Unternehmen zu einem 
ausgewogenen (balanced) Bild. 

Da die Möglichkeiten zur Befüllung der Scorecard sehr vielseitig sind, sollte vermieden werden, sie mit zu 
vielen komplexen Zahlen zu überladen. Im Fokus stehen bei diesem Konzept vorrangig die 
Maßnahmenfindung unter Berücksichtigung von Ursache und Wirkung, was durch eine einseitige Betrachtung 
der Kennzahlen zu sehr in den Hintergrund rücken und den Lösungsprozess negativ belasten könnte.

\subsection{Besonderheiten an Hochschulen}
Ein gut funktionierendes Qualitätsmanagement kann nur effektiv und reibungslos funktionieren, wenn es an zentraler Stelle nahe des Entscheidungsträgers positioniert und gelebt wird.
Die Umsetzungsverantwortung eines ganzheitlichen Qualitätsmanagements liegt bei Hochschulen in der 
Regel bei der Hochschulleitung, die ihre Aufgaben im Prozess der Informationsflussoptimierung begreifen 
und verantworten muss.

Eine der grundlegenden Besonderheiten an Hochschulen liegt in der internen Strukturierung von Verantwortlichkeiten. Die Hochschule ist unterteilt in Fachbereiche, welche geschlossen für sich arbeiten können, aber dennoch der Hochschulleitung unterstellt sind. Zusätzlich zu diesen beiden Bereichen ist noch das Präsidium zu nennen, welches insgesamt für eine effiziente Aufgabenerfüllung und Interessenvertretung der Hochschule verantwortlich ist.\footcite[353 ff.]{mintzberg_1992}

Im Zuge der Einführung eines geordneten Qualitätsmanagements gilt es also, die Positionierung nahe der 
Hochschulleitung mit einer anwendungsbezogenen Platzierung innerhalb jedes Fachbereiches unter 
Einbeziehung des Präsidiums zu verknüpfen, um ganzheitliche Lösungen zur Realisierung eines 
Qualitätsmanagements zu finden und umsetzen zu können.

Ein Außenvorlassen des Fachbereichs, in dem die Lösungen schließlich umgesetzt werden, ist faktisch 
unmöglich. Durch die Vielzahl an Entscheidungsträgern und Mitrednern besteht an Hochschulen ein höherer 
Bedarf an Kommunkations- und Abstimmungsleistungen zwischen diesen als in anderen Institutionen und 
Unternehmen. 

Es besteht zudem die Gefahr, dass Zuständigkeiten der verschiedenen Rollen an der entsprechenden 
Hochschule nicht klar geregelt sind, was die Funktionsweise des Entscheidungsprozesses zwar bestenfalls 
nicht beeinträchtigt, dessen Ablauf allerdings sehr unsystematisch gestaltet und den Fluss des Prozesses 
ausbremst.

Neben der strukturellen Schwierigkeiten in der Aufstellung eines Qualitätsmanagements besteht eine weitere Besonderheit in der inhaltlichen Vereinheitlichung der Anforderungen der einzelnen Parteien, die im schlechtesten Fall sehr verschieden sind oder sich gar widersprechen, sodass diese für alle Bereiche zentral gültig ist. 

Mithilfe renommierter Werkzeuge, wie z.B. der IT Balanced Scorecard, liegt es nun in der Hand des 
Qualitätsmanagement-Teams, die erarbeiteten Prozessstrategien und Maßnamen transparent für jeden 
Bereich der Hochschule einsehbar zu publizieren und alle betreffenden Personen über Änderungen zu 
informieren. 

Die Kontrolle in den Fachbereichen, ob und inwieweit die 
Maßnahmen zur Prozessoptimierung beitragen, darf hierbei nicht vernachlässigt werden.\footcite{evalag_eckpunkte_2012}


\chapter{Best-practice-Beispiel von INM an Hochschulen}
\chapter{Ist-Situation der Hochschule Emden/Leer hinsichtlich wichtiger Dimensionen}

Autoren: Marc Enders, Tina Koppermann

\section{Ziel (TiK)}
Mit Hilfe einer Analyse der Ist-Situation an der Hochschule Emden/Leer wird festgestellt, in wieweit ein Informationsmanagement besteht. Wenn dies nicht der Fall ist, wird recherchiert, welche Informationen bereits zentral gesammelt werden und welche Bereiche in das Projekt \textbf{„Potentielle Neuordnung des Informationsmanagements einer kleineren Fachhochschule auf der Grundlage bestehender Lösungen an deutschen Hochschulen“} mit einbezogen werden müssen.
\section{Aufgaben (TiK)}
Wesentliche Fragestellungen, die in diesem Kapitel gelöst werden sollen, sind auf der einen Seite, welche vorhandenen IT-Systeme bereits zentral Verwendung finden und auf der anderen Seite, wie Informationen aktuell repräsentiert werden. Des weiteren soll in dieser Analyse Aufschluss darüber gegeben werden, ob ein Informationsmanagement an der Hochschule betrieben wird und wie Informationen bereits zentral zur Verfügung gestellt werden.
  
Bei der Hochschule Emden-Leer handelt es sich um eine kleine Hochschule mit aktuell 4626 eingeschriebenen Studierenden. Den größten Anteil machen die 4303 Studenten vor Ort aus.\footnote{\url{http://www.hs-emden-leer.de/fileadmin/user_upload/Einrichtungen/ZDF/Studierende/JV_Stud_20142.pdf}} Es sind 396 Mitarbeiter beschäftigt, wobei 107 Professuren sind.\footnote{\url{https://www.hs-emden-leer.de/no_cache/hochschule/zahlen-daten-fakten.html}}

Ein Hauptbestandteil dieses Kapitels ist der Prozess der Sammlung, Selektion und Prüfung von Fragestellungen, welche die Grundlage für ein Experteninterview bilden. Im Rahmen dieser Ausarbeitung wurde sich für die Verwendung eines Experteninterviews entschieden, da hier die Zielgruppe ein Spezialist aus dem Fachbereich ist. In diesem Fall ist der Interviewte der Leiter des Hochschulrechenzentrums der Hochschule Emden/Leer, Herr Günter Müller. Das Experteninterview führten die Studierenden Tina Koppermann, Marc Enders, die betreuende Professorin Frau Prof. Dr. Krüger-Basener mit Herrn Günter Müller durch. 

Es wurde bei der Erstellung dieses Experteninterviews auf die Methodik des SPSS-Prinzipes verstärkt reflektiert. Dem SPSS-Prinzip nach Helfferich\footnote{\cite{helfferich_2009}} liegt folgendes Vorgehen zur Grunde:

\begin{enumerate}
	\item Sammeln
	\item Prüfen
	\item Selektieren
	\item Subsumieren		
\end{enumerate}

Mit Hilfe des Prinzips zur qualitativen Datenerhebung fand im ersten Schritt das Sammeln von Fragen statt. Diese konnten von allen Kursteilnehmern in einem zur Verfügung gestellten Online-Dokument eingesehen und editiert werden. Bei der Sammlung der Fragen sind insgesamt 62 Fragestellungen zu unterschiedlichen Schwerpunkten aufgenommen worden (siehe Abbildung \ref{fig_auszug_fragen_sammeln}).

\begin{figure}[h!]
	\centering
	\includegraphics[width=10cm]{kapitel/gruppe2/bilder/auszug_fragen}
	\caption{Auszug der gesammelten Fragen}
	\label{fig_auszug_fragen_sammeln}
\end{figure}

Nachdem die Sammlung aller Fragen abgeschlossen war, folgte im zweiten Schritt die Prüfung dieser. Hierbei wurden reine Informationsfragen aussortiert. Nach der erfolgreichen Prüfung der Fragen folgte im nächsten Schritt die Selektion dieser. Die Fragestellungen wurden nach Themengebieten kategorisiert. 

Im letzten Schritt, dem Subsumieren, wurde für jedes Themengebiet eine Erzählaufforderung gefunden und der Interviewleitfaden entsprechend dieser gegliedert. Mit Hilfe eines Farbcodes (siehe Abbildung \ref{fig_farbcode_SPSS}) wurden die Fragen entsprechend nach Erzählaufforderung, Checkliste, konkreter Frage und Aufrechterhaltungsfrage farblich markiert und anschließend einsortiert (siehe Abbildung \ref{fig_sortierung_fragentyp}).

\begin{figure}[h!]
	\centering
	\includegraphics[width=10cm]{kapitel/gruppe2/bilder/farbcode_spss}
	\caption{angewandter Farbcode für das SPSS-Prinzip}
	\label{fig_farbcode_SPSS}
\end{figure}

\begin{figure}[h!]
	\centering
	\includegraphics[width=10cm]{kapitel/gruppe2/bilder/sortierung_fragentyp}
	\caption{Sortierung der Fragen nach Fragentyp}
	\label{fig_sortierung_fragentyp}
\end{figure}

Die Visualisierung der Subsumtion fand mit Hilfe des Anwendungsprogramms Microsoft Excel statt. Als Endergebnis ist ein in acht unterschiedliche Themenbereiche gegliederter Interviewleitfaden entstanden (siehe Abbildung: \ref{fig_auszug_interviewleitfaden}).

\begin{figure}[h!]
	\centering
	\includegraphics[width=10cm]{kapitel/gruppe2/bilder/auszug_leitfaden}
	\caption{Auszug des Interviewleitfadens}
	\label{fig_auszug_interviewleitfaden}
\end{figure}

An einem festgelegtem Interviewtermin ist mit Hilfe dieses Leitfadens das Experteninterview mit Herrn Günter Müller durchgeführt worden. Dieses Interview fand über die Online Video Plattform „Adobe Connect“ statt. Herr Müller stimmte der digitalen Aufzeichnung zu. Im Anschluss an das Experteninterview wurde in der ersten Phase die Aufzeichnung auf wichtige inhaltliche Aspekte analysiert. 

Wie in Abbildung \ref{fig_E-Learning_Transkription} exemplarisch zu sehen ist, wurde in der zweiten Phase durch Transkription die zur Verfügung gestellte Aufzeichnung mit Hilfe der Applikation „Microsoft Word“ überführt, um die im Interview erhaltenen Informationen besser verarbeiten zu können.

\begin{figure}[h!]
	\centering
	\includegraphics[width=10cm]{kapitel/gruppe2/bilder/E-Learning_Transkription}
	\caption{Transkription: E-Learning}
	\label{fig_E-Learning_Transkription}
\end{figure}

In der dritten und letzten Phase wurden mit Hilfe des Tools „XMind6“ zu jedem Themenbereich ein entsprechendes Mindmap generiert, um somit bei der Recherche schneller auf Besonderheiten eingehen zu können  (siehe Abbildung \ref{fig_E-Learning_MM}).

\begin{figure}[h!]
	\centering
	\includegraphics[width=10cm]{kapitel/gruppe2/bilder/E-Learning_MM}
	\caption{Mindmap: E-Learning}
	\label{fig_E-Learning_MM}
\end{figure}

Die Ergebnisse dieser Analyse werden in den folgenden Kapiteln detaillierter beschrieben. 

\section{Zuständigkeiten}
\label{section_zustaendigkeiten}
In diesem Kapitel wird auf die Zuständigkeiten im Bezug auf Informationsbereitstellung an der Hochschule Emden/Leer eingegangen. Es wird dargestellt, welche Bereiche bereits zentral an der Informationsbereitstellung beteiligt sind und wo bereits Synergien vorliegen. Ebenso wird auf die Besonderheiten einzelner Fachbereiche, zentrale Einrichtungen und das Präsidium detaillierter eingegangen. 

\begin{figure}[h!]
	\centering
	\includegraphics[width=14cm]{kapitel/gruppe2/bilder/organigramm_HS}
	\caption{Organigramm der Hochschule Emden/Leer}
	\label{fig_organigramm_HS}
\end{figure}

\missingfigure{Schaubild über zentral genutzte Systeme unabhängig von Kooperation}

\subsection{Fachbereiche}
Die einzelnen Fachbereiche sind unter anderem durch die Mitgliedschaft in Arbeitsgruppen in den Informationsbeschaffungsprozess involviert (siehe Kapitel \ref{subsection_arbeitsgruppen_informationsaustausch}). 

Alle Fachbereiche verfügen über die Berechtigung relevante Informationen in dem Infosys darzustellen. Infosys ist eine zentrale Plattform, welche online auf der Webseite der Hochschule Emden/Leer öffentlich von jedem eingesehen werden kann oder vor Ort  in den Eingangsbereichen der jeweiligen Fachbereiche über Dashboards. Es werden, nach Fachbereich sortiert, die wichtigsten Neuigkeiten als Newsticker dargestellt und der Zugriff auf alle Vorlesungspläne der Fachbereiche ist gegeben um so zügig auf organisatorische Inhalte zugreifen zu können.

\missingfigure{Screenshot Info Sys}

In den nachfolgenden Kapiteln wird nur auf die Besonderheiten der einzelnen Fachbereiche eingegangen.

\subsubsection{Seefahrt}
Bei dem Fachbereich Seefahrt handelt es sich um einen relativ kleinen Fachbereich. Seefahrt ist  nur an dem Standort Leer vertreten. Dieser Fachbereich verwendet kein zentrales System zur Vorlesung und Raumplanung, sondern eine Eigenentwicklung.

\subsubsection{Technik}
Eine Besonderheit dieses Fachbereiches ist, dass für den Laborbetrieb ein Rechennetz neben dem zentralen Rechennetz der Hochschule Emden/Leer betrieben wird. Da unter anderem der Bereich „IT-Sicherheit“ ein wichtiger Aspekt in dem Studiengang Informatik ist, kommt es zu besonderen Konstellationen im Bereich der Forschung. Dieser Bereich verwaltet sein Netz selbst und ist somit autark vom allgemeinen Hochschulrechennetz.

\subsubsection{Wirtschaft}
\subsubsection{Soziale Arbeit und Gesundheit}

\subsection{Präsidium}
Das Präsidium insbesondere mit dem Bereich zentrale Verwaltung ist durch die Mitgliedschaft in Arbeitsgruppen in den Informationsbeschaffungsprozess involviert. Eine besondere Stelle, im Bezug auf die Repräsentation von Informationen besonders das Erscheinungsbild nach außen (siehe Kapitel \ref{section_kooperations_situation}), stellt eine Stabsstelle von dem Präsidium da. Das Präsidialbüro ist unter anderem für den Bereich Hochschulmarketing zuständig.

\subsection{Zentrale Verwaltung}
(\ldots)
\subsubsection{Studentenverwaltung}
\subsubsection{Mitarbeiterverwaltung}
\subsubsection{Rechenzentrum}
Das Hochschulrechenzentrum der Hochschule Emden/Leer ist stark in die Administration und Pflege der bestehenden Systeme zur Informationsbereitstellung involviert. Neben der Administration von bestehenden Systemen obliegt dem Hochschulrechenzentrum ebenfalls der Endkundensupport.

\subsection{Arbeitsgruppen zum Informationsaustausch und zur Informationsbereitstellung}
\label{subsection_arbeitsgruppen_informationsaustausch}
Die Zuständigkeiten an der Hochschule, in Bezug auf Informationssammlung, Beschaffung und Aufbereitung von Informationen, ist bereits durch Arbeitsgruppen in wichtigen Bereichen geregelt. Durch das Interview mit dem Leiter des Hochschulrechenzentrums der Hochschule Emden/Leer konnte ein Einblick in die bestehenden Gremien geschaffen werden. Derzeit existieren drei Arbeitsgruppen, welche für die Informationsverteilung in den jeweiligen Bereichen relevant sind:

\begin{itemize}
	\item Zahlen, Daten und Fakten (ZDF)
	\item WEB
	\item Moodle
\end{itemize}

\subsubsection{Zahlen, Daten und Fakten (ZDF)}
ZDF setzt sich zusammen aus den Verwaltungsabteilungen Finanzen, Personal, Presse und Rechenzentrum. Dieses Gremium ist zuständig für die Aufbereitung und zur Verfügung Stellung von Kennzahlen wie zum Beispiel aktuelle Kennzahlen zu eingeschriebenen Studierenden pro Studiengang.  ZDF ist für einen Unterbereich der offiziellen Webseite der Hochschule Emden/Leer zuständig. Die Kennzahlen und Zahlen werden gruppenbasiert erstellt. Es werden grobe Kennzahlen erzeugt, welche öffentlich zugänglich sind und detailliertere Zahlen für die Mitarbeiter, mit welchen sie arbeiten können. Dekane erhalten speziellere Zahlenwerte.

\subsubsection{WEB}
Es existiert eine Arbeitsgruppe, welche für die Gestaltung und den Inhalt der öffentlichen Webseite der Hochschule Emden/Leer verantwortlich ist. In dieser Arbeitsgruppe sind aus jedem Fachbereich Repräsentanten mit einbezogen. Die Leitung des Web-Teams obliegt dem Präsidialbüro.\footnote{\url{http://www.hs-emden-leer.de/fileadmin/user_upload/Einrichtungen/Praesidialbuero/Organigramm_Praesidialbuero_Juli2013_01.pdf}}

\subsubsection{Moodle}
In der Arbeitsgruppe „Moodle“ sind sowohl Repräsentanten aus jedem Fachbereich involviert sowie auch Repräsentanten aus der Verwaltungsebene. Da das Moodle E-learning System mittlerweile als ein zentrales Moodle für alle Bereiche eingeführt wurde, haben die Mitglieder aus den Fachbereichen unter anderem das Recht Kurse im Moodle freischalten zu können. 

\section{Definierte und bestehende Prozesse (Regelung und Handhabung von vorhandenen Informationen)}

\subsection{Wissensmanagement}
Wissensmanagement ist für die Hochschule Emden/Leer ein sehr wichtiger Aspekt, da Sie täglich mit dem Erwerb, der Entwicklung, dem Transfer sowie der Nutzung von Wissen konfrontiert wird. Für den Betrieb eines erfolgreichen Wissensmanagements ist an ein klares Regelwerk die Voraussetzung.

\subsection{E-Learning}
\todo{Quellenangabe?}
Laut Michael Kerres\footnote{Ich bin eine Quelle, bezeichne mich} ist E-Learning das Lehren und Lernen bei dem elektronische Medien für die Präsentation und Distribution von Lehrmaterialien und Kommunikation zum Einsatz kommen.

\subsubsection{E-Learning an der Hochschule Emden/Leer}
E-Learning ist an der Hochschule Emden/Leer ein sehr wichtiges Thema, da an der Hochschule der Studiengang Medieninformatik (Online), Wirtschaftsinformatik (Online) akkreditiert  wurde. Nun soll dargelegt werden ob in den Präsenzstudiengängen ebenfalls das Thema E-Learning Einzug gehalten hat.

\subsubsection{E-Learning in den Präsenzstudiengängen}
Es soll betrachtet werden in welchen Präsenzstudiengängen E-Learning eingesetzt wird.

\missingfigure{Grafik E-Learning Präsenzstudium}

Aus der Grafik geht hervor, dass der Fachbereich SAG (Soziale Arbeit und Gesundheit) der Vorreiter aller Fachbereiche mit der Einführung eines E-Learning Systems war. SAG setzt mehr als 200 Onlinekurse im Präsenzstudium ein. Auch für die Anmeldung an verschiedenen Kursen kommt ein Online-System zum Einsatz.

Es folgt dann der Fachbereich Technik in dem E-Learning ebenfalls sehr stark verbreitet ist, da die Studiengänge Medieninformatik und Wirtschaftsinformatik als reine Onlinestudiengänge in diesem etabliert sind.

Weniger stark wird E-Learning vom Fachbereich Wirtschaft betrieben. Das geringste Nutzungsverhalten ist im Fachbereich Seefahrt zu verzeichnen.

Trotz des unterschiedlichen Nutzungsverhaltens hat E-Learning in allen Fachbereich Einzug gehalten.

\subsubsection[Einsatz von E-Learning-Anwendungen]{Einsatz von E-Learning-Anwendungen in den Präsenzstudiengängen}
Im diesem Abschnitt wird erläutert, ob die Anwendungen Adobe Connect, Moodle im Präsenzstudiengang eingesetzt werden.

\paragraph{Adobe Connect}\mbox{} \\
Adobe Connect ist eine Kommunikationsplattform zur Bereitstellung von Webmeetings und E-Learning-Inhalten.

Ausschließlich der Fachbereich Technik (E+I) nutzt durch seine Onlinestudiengänge die Plattform Adobe Connect als Medium des visuellen Austausches von Bild und Sprache. In den Präsenzstudiengängen kommt die Plattform nicht zum Einsatz, da der persönliche Austausch von Studierenden und Dozenten in den täglichen Präsenzen stattfindet.

\paragraph{Moodle}\mbox{} \\
Die Hochschule Emden/Leer setzt Moodle als Lernplattform ein. Moodle ist ein freies objektorientiertes Kursmanagementsystem welches prädestiniert ist für den Einsatz von E-Learning Inhalten.\footnote{\url{https://moodle.hs-emden-leer.de/moodle/}}

\begin{figure}[h!]
	\centering
	\includegraphics[width=8cm]{kapitel/gruppe2/bilder/moodle}
	\caption{Übersicht Moodle für alle}
	\label{fig_moodle}
\end{figure}

Durch den Einsatz von Moodle in allen Fachbereichen, wird das volle Leistungsspektrum des Systems ausgenutzt. Folgende Funktionalitäten werden angeboten:
\begin{itemize}
	\item Lernvideos
	\item Vorlesungsskripte
	\item Forum
	\item Kalender
	\item Mail-Connect
\end{itemize}

Für jeden Fachbereich wird der volle Funktionsumfang der Plattform zur Verfügung gestellt, auch wenn nicht jeder Fachbereich jeden Service nutzt.

\subsubsection{Zentrale Informationsbereitstellung durch Datenlaufwerke}
Für alle beteiligten der Hochschule Emden/Leer werden spezielle Datenlaufwerke zur Verfügung gestellt. Es handelt sich um 3 Netzlaufwerke auf den Fileservern der Hochschule.\footnote{\url{https://connect.hs-emden-leer.de/cgi-bin/portal}}

\begin{figure}[h!]
	\centering
	\includegraphics[width=8cm]{kapitel/gruppe2/bilder/zugriff_auf_laufwerke_extern}
	\caption{Zugriff auf die Datenlaufwerke von extern}
	\label{fig_zugriff_datenlaufwerke_extern}
\end{figure}


\paragraph{Laufwerk Z}\mbox{} \\
Auf dem Laufwerk Z befinden sich die Daten des Home-Verzeichnisses jedes einzelnen Benutzers. Meldet sich dieser an beliebigen Rechnern des Rechnerpools an, werden die Inhalte Ihres Home-Verzeichnisses automatisch eingebunden. Der interne Zugriff auf die eigenen Dateien ist von jedem Rechner des Pools möglich, da servergespeicherte Profile zum Einsatz kommen. Auch von externe können die Studierenden problemlos auf die Ressourcen der Datenlaufwerke zugreifen.

\paragraph{Laufwerk Y}\mbox{} \\
Für den gemeinsamen Austausch der Daten wurde das Transferlaufwerk Y eingerichtet. Hier werden zentral Ressourcen für alle Studierenden und Lehrenden aus Emden zum Austausch zur Verfügung gestellt.

\paragraph{Verzeichnis Lehrende aus Leer}\mbox{} \\
Zusätzlich zum Transferlaufwerk steht das Verzeichnis der Lehrenden aus Leer  zur Verfügung. Hier stellen die Lehrenden Inhalte zur Verfügung.

\subsubsection{Moodle vs. Datenlaufwerke für Präsenzstudenten}
An der Moodle-Plattform melden sich die Nutzer über eine webbasierte Oberfläche am System an. Um Dateien zur Verfügung zu stellen, muss auf der Weboberfläche zu den gewünschten Reitern navigiert werden.

Stellt man die Datenlaufwerke (Y Laufwerk und Transferlaufwerk der Lehrenden) der Moodle-Plattform gegenüber  und betrachtet nur den Aspekt des Datenaustausches, so wird deutlich, dass der Dateiaustausch über   Datenlaufwerke in Bezug auf Komfort und Aufwand deutlich besser für die Präsenzstudierenden geeignet ist, als die Dateiablage über die Moodle-Plattform. Die über die Datenlaufwerke zur Verfügung gestellten Inhalte können von den Studierenden mit wenig Aufwand intuitiv erreicht werden.

Aus dem Interview mit dem Rechenzentrumsleiter Herrn Günter Müller kristallisierte sich heraus, dass der Fachbereich Wirtschaft die Datenlaufwerke am stärksten und der Fachbereich Technik und andere Fachbereiche diese weniger stark nutzen.

\subsection{Sicherheitsaspekte}
(\ldots)

\subsubsection{Sicherheitsrichtlinien an der Hochschule Emden / Leer}
Der Einsatz von Sicherheitsrichtlinien ist ein wichtiges Thema an Hochschulen. Sicherheitsrichtlinien beschreiben die Sicherstellung von Verfügbarkeit, Integrität, Vertraulichkeit und Authentizität von Informationen.

An der Hochschule Emden/Leer werden als Basis für die Informationssicherheit Teile des IT-Grundschutz-Kataloges umgesetzt. Nicht alle Empfehlungen des BSI sind an einer kleinen Hochschule, wie die Hochschule Emden/Leer es ist, umsetzbar.

An den Serverräumen der Hochschule ist die Umsetzung der IT-Grundschutzmaßnahmen deutlich zu erkennen.

Folgende physikalische Schutzmaßnahmen wurden an der Hochschule Emden/Leer in den Serverräumen umgesetzt:

\begin{itemize}
	\item einbruchsicher
	\item feuergemeldet
	\item videoüberwacht
	\item Lage der Serverräume im 1. OG (Wasserschutz)
\end{itemize}

\subsubsection{Einsatz von ITIL}
Die IT Infrastructure Libary (ITIL) ist eine Sammlung von Best Practises zur Umsetzung eines IT-Service-Managements (ITSM). In diesem Regelwerk werden die für den Betrieb einer IT-Infrastruktur notwendigen Prozesse und Werkzeuge beschrieben.
Der ITIL-Prozess ist an der Hochschule Emden/ Leer nicht etabliert, da die Personaldecke für die Umsetzung eines 1st und 2nd-Level Supports nicht gegeben ist.\footnote{\url{http://de.wikipedia.org/wiki/IT_Infrastructure_Library}}

\subsubsection{Umsetzung von ISO/IEC 27001}
Die ISO/IEC 27001 Zertifizierung wird auf Basis des IT-Grundschutzes vergeben. Durch die Zertifizierung des ISO/IEC 27001 Standards haben Unternehmen, Behörden, Organisationen die Möglichkeit, ihre Bemühungen um Informationssicherheit nach innen und außen zu dokumentieren.

Das Einsatzszenario ist an der Hochschule Emden/Leer nicht gegeben, da für Umsetzung die Personaldichte zu gering ist. Für die Erfüllung der Zertifizierung würde riesiger Personaloverhead entstehen.\footnote{\url{https://www.bsi.bund.de/DE/Themen/ZertifizierungundAnerkennung/Zertifizierung27001/GS_Zertifizierung_node.html}}

\subsubsection{Single Sign-On}
(\ldots)

\subsubsection{Fazit Sicherheitsrichtlinien}
Abschließend ist zu sagen, dass an der Hochschule Emden/Leer der IT-Sicherheitsaspekt ein sehr wichtiges Thema ist. Als kleine Hochschule ist es auf Grund der Personaldichte nicht möglich, alle Empfehlungen des BSI-Grundschutzes, ITIL und ISO/IEC 27001 umzusetzen. Jedoch sucht sich die Hochschule aus den Regelwerken die Empfehlungen heraus, die auf Grund der Personaldichte umsetzbar sind.  Dies bildet eine sehr gute Basis im Hinblick auf das sehr anspruchsvolle Thema IT-Sicherheit.



\chapter{Mögliche Soll-Situation im Hinblick auf die heutigen und zukünftigen Aufgaben - AE, JL, HS}
\label{chapter_sollsituation_INM}
\textit{Autoren: Andreas Ebling, Julia Lübke, Hannes Sprafke}

Ich bin ein Einleitungstext zu diesem Kapitel, der noch geschrieben werden muss.

\section{Marketing}
\label{section_marketing}
Das Marketing hat neben dem typischen Aufgabenbereich der Außenrepräsentation durch die Besonderheiten einer Hochschule auch einen Aufgabenbereich der Innenrepräsentation. Im folgenden werden diese Bereiche getrennt betrachtet.

\subsection{Externes Hochschulmarketing}
\label{subsection_externes_hochschulemarketing}
Das externe Marketing der Hochschule bezieht sich auf die klassischen Marketingaufgaben, das Produkt und die Marke vorteilhaft darzustellen. Im Falle einer Hochschule ist dies die attraktive Darstellung gegenüber zukünftige Studierenden, Forschungsinteressierten und Geldgebern.

\subsubsection{Webseite}
Zentrales Element bleibt die Hochschulwebseite, die mit aktuellen, offenen Möglichkeiten von HTML 5, CSS 3 und JavaScript den Funktionsumfang einer App erreichen kann, ohne auf spezielle oder spezifische Spezialtechnologien zu setzen. Besonders sei an dieser Stelle die Möglichkeit genannt, mittels CSS Größen und Darstellungsmöglichkeiten von Endgeräten unabhängig von konkreten Betriebssystemen und Hardwareplattformen abzudecken.

Dies ist vor dem Hintergrund wichtig, dass beispielsweise eine native App für iPhones zwingend durch den Appstore der Firma Apple installiert werden muss, dessen Nutzungsbedingungen sich für die Hochschule in Form von Kosten oder Inhaltseinschränkungen zu Ungunsten der Hochschule verändern könnten. Mit einer nativen App lässt sich trotzdem nur ein beschränkter Nutzerkreis ansprechen. Sollen mehrere Apps für verschiedene Plattformen gepflegt werden, so ist dies mit zusätzlichen Aufwand verbunden.

Eine Neuauflage der Hochschulwebseite mit aktuellen Möglichkeiten und per CSS an verschiedene Darstellungsgrößen angepasst erreicht dagegen jedes internetfähige Gerät mit Browser. Sollte ein neuer Formfaktor wichtig werden, zum Beispiel der einer Smartwatch, so lässt sich dies über eine Erweiterung des Stylesheets erreichen, ohne eine komplette Neuentwicklung in Auftrag zu geben.

\subsubsection{Soziale Netzwerke}
Soziale Netzwerke wie facebook oder twitter sind kritisch zu bewerten. Auftritte auf diesen Plattformen können nicht alleine stehen, benötigen aber durch ständigen Nutzerkontakt eigenständige Pflege und Aufsicht, was an einer kleinen Hochschule Personal bindet.

Der Nutzen ist weiterhin fraglich. Die Hochschule kann bestenfalls Informationen der Webseite duplizieren, während Kommunikation von Interessierten zur Hochschule wieder schnell in bestehende Kanäle geleitet wird.

Weiterhin ist die Reichweite zwar potentiell weltweit, was für eine nach Leitbild „Hochschule der Region“ aber an sich nicht relevant ist. In der Praxis wichtiger ist die Dichte der Nutzer, da es unrealistisch ist zu erwarten, dass jeder in sozialen Netzwerken organisiert ist, oder aber sich zur Nutzung in einem sozialen Netzwerk anmelden müsste.

Über interne Kommunikation und Daten in sozialen Netzwerken müsste im Einzelfall nach Grundlage geltender Datenschutzvorgaben entschieden werden, was als Prozess vorab bereits zu aufwendig, als dass eine Erwägung hier weiter Sinn machen würde.
Im Endeffekt verbliebe also die Verbreitung ohnehin öffentlicher Informationen, die bereits auf der Webseite zu finden wären.

\subsubsection{Verteilte Content-Erzeugung}
Im Kontext einer kleinen Hochschule ist die Personalsituation zu berücksichtigen. Es kann nicht davon ausgegangen werden, dass eine oder mehrere Personen die Redaktion aller zu veröffentlichenden Inhalte übernehmen. Vielmehr ist zu erwarten, dass relevante Neuigkeiten an mehreren unterschiedlichen Stellen auftreten, und am besten ohne Umweg veröffentlicht werden. Hierzu eignen sich Content-Management-Systeme.

\begin{figure}[h!]
	\centering
	\includegraphics[width=10cm]{kapitel/gruppe3/bilder/verteilter_publishing_workflow}
	\caption{Verteilter Publishing Workflow}
	\label{fig_publishing_workflow}
\end{figure}

Mit diesen kann geleistet werden, dass mehrere, unabhängige Autoren Inhalte beisteuern und veröffentlichen können, ohne im Einzelnen mit den technischen Einzelheiten des Hostings oder des Designs belangt zu werden. Auch können vielfach Content Management Systeme Inhalte für weitere Plattformen aufbereiten, zum Beispiel an soziale Netzwerke posten.

\subsection{Internes Hochschulmarketing}
\label{subsection_internes_hochschulemarketing}
Im Gegensatz zur Situation in einem Unternehmen genießen einzelne Fachbereiche und Personen in einer Hochschule einen hohen Grad an Freiheit und Autonomie. Daher können in Arbeitsgruppen und Gremien beschlossene Prozesse und Software nicht per Anordnung durchgesetzt werden, sondern müssen nach innen vermarktet werden, um akzeptiert zu werden.Herausfordernd ist hier besonders die Heterogenität, da der technische und fachliche Hintergrund sich unter Mitarbeitern und Fachbereichen erheblich unterscheiden dürften, etwa zwischen technischen und nichttechnischen Fachbereichen.

\subsubsection{Fokussierter Support}
Eine Möglichkeit, Benutzer hin zu einer präferierten Lösung zu leiten ist diese in Präferenzen, Anleitungen und FAQs an erster Stelle und in höherem Detailgrad zu präsentieren. Vielfach wird eine Voreinstellung einfach übernommen, und die erste Lösung zu einer Fragestellung als Referenz angesehen.

\subsubsection{Schulungen}
Eine weitere Maßnahme ist, Schulungen für die präferierten Lösungen anzubieten, die Vorteile der gefundenen Lösung gegenüber anderen herausstellt. Optimal ist eine solche Lösung transparent, oder aber bietet Alleinstellungsmerkmale, die eine Verwendung aus sich heraus attraktiv erscheinen lassen. Trotzdem kann es vorkommen, dass in Lern- und Umstellungsphasen Lernkurven in der Benutzung absolviert werden müssen. Soll eine Lösung akzeptiert werden, dann muss diese Lernkurve entsprechend begleitet werden.

\subsubsection{Integration}
Eine weitere starke, aber arbeitsintensive Maßnahme ist, die präferierte Lösung stark zu integrieren. Beispielsweise sei die Erstellung von hochwertigen Dokumentvorlagen entsprechend der Corporate Identity für die präferierte Textverarbeitung genannt.

Der Übergang zum fokussierten Support ist hier fließend. Wird die präferierte Lösung an das bestehende System angepasst, so kann von Integration gesprochen werden, wird das System an eine präferierte Lösung angepasst, ist dies fokussierter Support. Beides kann sehr gut gegenseitig ergänzend eingesetzt werden.
\section{Support und Fortentwicklung - AE}
Support und Fortentwicklung hängen hier eng zusammen, da die Fortentwicklung 
haupt-sächlich durch die im Support gewonnenen Einsichten über Defizite in Prozessen 
getrieben werden soll. So sollen Diskrepanzen zwischen Erwartungen an das System und 
dessen tatsächlichen Fähigkeiten und Nutzung aufgedeckt und behoben werden.

\subsection{Support}
Supportleistung an einer kleinen Hochschule geschieht häufig direkt und 
unbürokratisch.\footcite{gunter_muller_interview} Dieser ad-hoc-Ansatz bringt 
zwar vielfach schnelle Hilfe, aber nur wenig zuverlässige Informationen über Prozessdefizite.

\subsubsection{Zentrale Dokumentation}
Die Vorteile dieser Art der Hilfeleistung sind für eine kleine Hochschule allerdings evident. 
Der Overhead mehr reglementierter Supportsysteme würde einen unverhältnismäßigen 
Personalaufwand mit sich bringen, und Hilfeleistung verzögern. Die Qualität des 
Supportprozesses selber würde damit sinken.\footcite{gunter_muller_interview}

Notwendig zur besseren Identifizierung von Prozessdefiziten ist allerdings keine 
zentralisierte Supportleistung an sich, sondern lediglich eine zentralisierte Dokumentation 
des geleisteten Supports.
\newpage

\begin{figure}[h!]
	\centering
	\includegraphics[width=10cm]{kapitel/gruppe3/bilder/grafik_supportlog}
	\caption{Unabhängige Supportleister dokumentieren in zentralem Log}
	\label{fig_zentraler_supportlog}
\end{figure}

Die Abbildung \ref{fig_zentraler_supportlog} verdeutlicht, dass geleisteter Support von vollkommen unabhängigen Stellen zentral dokumentiert werden kann.

Es ist dabei unerheblich, ob die Supportleister die Unterstützung als Kern ihrer Aufgabe 
leisten, oder ob es sich um kollegiale Unterstützung bei einem Problem handelt. Gerade 
letztere Information aufzufangen ist wichtig, da diese sonst nur eine sehr schwer 
einzuschätzende Größe bleibt.

Dieser Dokumentationsoverhead ist gering gegenüber dem Overhead eines stark 
reglementierten Supportsystems, erhält alle Vorteile unbürokratischer, schneller 
Hilfeleistung und fängt zusätzlich Informationen über Art und Umfang gelisteten Supports 
auf.

\subsubsection{Knowledge Base}
Aus dem Supportlog kann eine durchsuchbare Knowledge Base aufgebaut werden, die nicht 
nur die allgemeinen Fehlerquellen und Schwierigkeiten von Software im Einsatz beleuchtet, 
sondern ganz speziell die an der Hochschule Emden/Leer in dieser Zusammenstellung 
einmaligen Konfiguration vorliegenden Probleme.

Dadurch kann sehr viel schneller auf spezifische Fehlerszenarien reagiert werden, als dies 
mit allgemeinen Informationen möglich ist, die erst auf die Verhältnisse vor Ort bezogen 
werden müssen.

Auch können aus dem Supportlog FAQs abgeleitet werden, die tatsächlich dem Wortsinn 
nach Listen häufig gestellter Fragen und Antworten darstellen, und nicht was mehr oder 
minder begründet vermutet wird. Eine Diskrepanz mag sich hier durch die Besonderheit von 
Hochschulen ergeben, ein sehr heterogenes Benutzerfeld abzudecken. So mag es 
Nutzergruppen geben, die ein ähnliches Maß an technischer Kompetenz aufweisen wie 
Personal, das ein bestimmtes System betreut, bis hion zu Benutzergruppen, die weit weniger 
oder deutlich andere technische Kompetenz aufweisen.

Auch zeigt sich in den Häufigkeiten bestimmter Probleme, wo spezielle Dokumentation und 
Hilfetexte notwendig sind, die ebenfalls hinterlegt werden können.

Hierzu muss das Supportlog allerdings von einer geeigneten Stelle regelmäßig gesichtet 
werden.

\subsection{Fortentwicklung}
Eine Konzeption kann nur aktuelle Trends und Entwicklungen berücksichtigen. Es ist 
schwierig vorauszuschauen, was die Zukunft danach bringen wird, welche Trends mehr oder 
weniger wichtig sind, und welche Trends darauf folgen werden.

Allerdings ist es keine Frage, dass eine Hochschule länger Bestand hat, und es damit 
sinnvoll ist, Prozesse zu hinterlegen, die neue Trends und Entwicklungen zwar nicht 
vorwegnehmen können, aber deren zeitnahe Entdeckung und Integration ermöglichen.

Auch zeigt sich in der Praxis, dass unvorhergesehene Bedingungen und Ereignisse 
theoretisch gut ausgearbeitete Prozesse und Infrastrukturen übermäßig blockieren können, 
und eine Anpassung geschehen muss. Beispielhaft sei hier für die Hochschule Emden/Leer 
der Trend angeführt, dass Mitarbeiter und Studierende eigene, WLAN-fähige Geräte 
mitbringen und im Netzwerk der Hochschule anzumelden. Das vormals ausreichend 
dimensionierte Netz wurde durch einen Trend unter- oder zumindest 
fehldimensioniert.\footcite{gunter_muller_interview}

\subsubsection{Feedback}
Zur effektiven Begegnung neuer Trends muss an jedem Punkt des Gesamtsystems dem 
Benutzer möglich sein, Feedback zu geben. Mehr noch muss gerade bei neuen oder 
überarbeiteten Prozessen dieses Feedback eingefordert werden, um die Qualität des neuen 
Prozesses oder Tools einschätzen zu können.

Das Feedback gelangt an die zuständige Stelle, muss aber auch zentral gesammelt werden, 
ähnlich wie das Supportlog. Diese Sammlung wird zentral ausgewertet, um verdeckte, 
verteilte Probleme aufzudecken, die sich in Feedback an unterschiedliche Stellen verbergen 
können.

Auf die Auswertung muss, wo sich Probleme zeigen, eine Information der zuständigen Stelle 
folgen, damit eine Verbesserung erarbeitet werden kann. Entsprechend ist die zuständige 
Stelle berechtigt, ein Meeting einzuberufen, damit ihre Eingaben nicht einfach verloren gehen 
können, sondern zwangsläufig mindestens einmal besprochen werden.

\subsubsection{Innovationseingabe}
In den Feedbackprozess eingebettet muss die Möglichkeit für jede Person sein, Innovationen 
aus beliebiger Quelle zu beschreiben, so dass Entwicklungen nicht erst von bestimmter 
Stelle wahrgenommen werden müssen, um erwägt zu werden. Damit kann von beliebiger 
Stelle aus eine Verbesserung in Diskussion gebracht werden.

Damit diese Möglichkeit von Benutzern angenommen wird, muss auf Eingaben angemessen 
schnell reagiert werden. Um eine ernsthafte Reaktion zu gewährleisten, müssen diese 
Vorschläge auch diskutiert worden sein. Daraus ergibt sich ein angemessen kurzer Turnus 
der Auswertung von Support- und Feedbacklog.

\subsubsection{Erfahrungsgetriebene Fortentwicklung}
Aus den Erkenntnissen über Schwachstellen aus dem Supportlog, den Benutzerberichten und 
-bewertungen aus dem Feedbacklog und den Innovationseingaben können nicht nur 
Schwachstellen und Fehler in Prozessen identifiziert werden, sondern auch Trends in der 
Benutzung des Systems erkannt. Da Support und Feedback andauernde Prozesse sind, ergibt 
sich daraus ein selbstregulierendes System, das, wenn die Messgrößen Supportlog und 
Feedbacklog angemessen berücksichtigt werden, evolutionär verbessert wird.

\begin{figure}[h!]
	\centering
	\includegraphics[width=\textwidth]{kapitel/gruppe3/bilder/zyklus_prozessverbesserung}
	\caption{Zyklus der Verbesserung eines Prozesses}
	\label{fig_zyklus_prozessverbesserung}
\end{figure}

Abbildung \ref{fig_zyklus_prozessverbesserung} illustriert den Zyklus folgendermaßen: Ein Prozess existiert und läuft im 
normalen Betrieb. Bei Problemen wird Support geleistet, der im Supportlog vermerkt wird. Zu 
dem Prozess wird zusätzlich Feedback gegeben, das im Feedbacklog festgehalten wird. Beide 
Logs werden ausgewertet und zur Verbesserung des Prozesses herangezogen. Der 
verbesserte Prozess tritt an die Stelle des ursprünglichen Prozesses, der Zyklus beginnt auf 
Basis des verbesserten Prozesses erneut.





















\section{Hard- und Software}
Zur Integration eines hochschulweiten Informationsmanagements können bezüglich der IT mehrere Ansätze gefahren werden.

Zum einen kann eine ganzheitliche integrierte Lösung verwendet werden. Die Universität Hamburg hat einen vollständigen Neuanfang bezüglich der Campussoftware gewagt mit der integrierten Gesamtlösung „CampusNet“ der Datenlotsen Informationssysteme GmbH. Die Entscheidung dazu resultierte aus der Zusammenlegung mehrerer Fachbereiche mit sehr unterschiedlichen Teillösungen zu einzelnen Fakultäten. Die verschiedenen Teillösungen waren größtenteils inkompatibel oder aufgrund von Eigenentwicklung schwer wartbar.\footnote{\cite{dini_webportale_2007}}

Laut Günter Müller, Leiter des Rechenzentrums, existieren an der Hochschule Emden Leer derart verschiedene Teillösungen nicht. Auch würden sich Eigenentwicklungen auf vernachlässigbare Systeme beschränken. Software würde grundsätzlich für die gesamte Hochschule eingesetzt.\footnote{Interview}

Der Einsatz einer integrierten Gesamtlösung zur Beseitigung von Inkompatibilitäten und schwer wartbaren Eigenentwicklungen kann somit keine Argumentationsgrundlage sein.
Des Weiteren reicht die in dieser Ausarbeitung getätigte Analyse des Ist-Zustands und der Anforderungen nicht aus, um einen vollständigen Anforderungskatalog zu bilden, auf dessen Grundlage eine integrierte Gesamtlösung gefunden werden kann.

Stattdessen wird auf eine flexible Lösung gesetzt, welche den Einsatz einzelner Fachanwendungen zur Lösung bestimmter Probleme vorsieht. Personelle und finanzielle Ressourcen sind dadurch flexibler einsetzbar, auf veränderte Anforderungen an eine Lösung kann flexibler reagiert werden und die Abhängigkeit von einem Anbieter für alle Anwendungen wird aufgelöst.

\subsection{Kernanforderungen}
Bei Core-Systemen wird weiterhin auf Appliance Lösungen gesetzt. Das minimiert Fehlerpotenzial und den operativen Betrieb.\footnote{Interview}

Softwaresysteme laufen auf virtuellen Maschinen. Die bessere Hardwareauslastung und Möglichkeit der automatisierten Administration kann finanzielle und personelle Ressourcen sparen.\footnote{\cite{baun_servervirtualisierung_2009}}

Die Systeme sind weniger abhängig von der Hardware, was dessen Austausch erleichtert. Netzwerkanbindung, Rechenleistung und Speicherkapazität sind somit flexibler an sich verändernde Anforderungen anpassbar.

Die eingesetzte Software soll in die Systemlandschaft integrierbar, lösungsorientiert und möglichst barrierefrei sein, sowie möglichst unabhängig von Client-seitig eingesetzten Systemen. Letzteres unterstützt auch den Ansatz der Freiheit in Forschung und Lehre.

Die Systemunabhängigkeit kann durch Webanwendungen im Sinne des Ansatzes Software as a Service (SaaS) erreicht werden. Um möglichst alle gängigen Browser und Endgeräte zu unterstützen, sollten die Anwendungen den Standards des World Wide Web Consortiums (W3C) und, soweit möglich, dem Ansatz responsive design gerecht werden. Dadurch kann auch dem Trend bring your own device Rechnung getragen werden.

\subsection{Bereichsübergreifende Basissysteme}
In den Bereichen Forschung, Lehre und Verwaltung fallen informationstechnologische Aufgaben an, für die eine zentrale IT-gestützte Lösung geschaffen werden kann. Das verringert redundante Daten und Systeme sowie administrative Aufwände. Im folgenden werden Lösungen für einzelne Aspekte des Informationsmanagements aus IT-Sicht vorgestellt, die in allen drei Bereichen genutzt werden können. Weiterhin dienen sie teilweise als Grundlage für spezialisierte Systeme.

\subsubsection{Identity Management}
Um die Anzahl an verschiedenen Accounts zu minimieren, sollten die Benutzer zentral gepflegt werden. Dies kann in einem Verzeichnisdienst wie dem bereits eingeführten Active Directory geschehen. Die Authentifizierung an einem System findet dann nicht am System selbst statt, sondern mit Hilfe des Verzeichnisdienstes. Der Anwender muss sich nur einen Anmeldenamen zzgl. Kennwort merken, um sich an den verschiedenen Systemen anzumelden. Weiterhin gilt eine Aktualisierung von Informationen global, wodurch Inkonsistenzen aufgelöst werden.

Davon ausgenommen dürfen Systeme sein, deren spezielle Sicherheitsanforderungen nicht mit diesem Konzept vereinbar wären.

In allen anderen Fällen ist zuzüglich zum zentralen Verzeichnisdienst auch ein SingleSignOn (SSO) Mechanismus empfehlenswert\footnote{\cite{zahn_itmanagement}}, wie es an der Universität Augsburg durch das System Webauth umgesetzt ist.

Die sich im Einsatz befindlichen Websysteme sollen auf SSO umgestellt werden, um dem Benutzer eine möglichst integrierte Landschaft zu bieten. Weiterhin sollte jedes System aus Sicherheitsgründen insofern angepasst werden, dass auch ein SingleSignOff möglich ist. Die zentrale Abmeldung soll gewährleisten, dass die Benutzenden auf allen Systemen, auf denen sie sich bewegt haben, mit einem Klick abgemeldet sind.

Die Benutzerdatenpflege sollte auf den einzelnen Systemen ausgeschaltet sein und ausschließlich über ein zentrales Formular geschehen. So wird ein konsistenter Datenbestand gesichert. Insofern die Informationen in anderen Systemen benötigt werden, müssen diese vom zentralen System direkt angefordert oder, wenn die Daten im System persistiert sein müssen, automatisiert und über gesicherte Verbindungen verteilt werden. Ein weiterer Vorteil ist, dass die Informationen, da zentral gesammelt, auch zentral ausgewertet werden können.

Diese zentrale Informationsbasis ermöglicht zentrale persönlichen Informationen mit Informationen anderer Art aus anderen Systemen anzureichern und weiterzuverwenden. So können automatisierte Reports über Forschungsprojekte erstellt werden, Expertisen zu bestimmten Themen identifiziert werden oder für die Verwaltung Verknüpfungen von Personen zu verwaltungstechnischen Aufgaben wie zum Beispiel Exmatrikulation. Die Umsetzung ist dabei individuell an die Gegebenheiten und Informationsbedarfe der Hochschule anzupassen.\footnote{\cite{vogl_fortschritte_2012}} Aus diesem Grund wird die technische Lösung eine Individuallösung werden.

\subsubsection{Geschäftsprozesse}
Neben der Forschung gibt es an Hochschulen gerade im Verwaltungsbereich viele Geschäftsprozesse, die in der Regel immer gleich ablaufen. Problem ist, dass Personen unterschiedlichster Bereiche involviert sind und die Prozesse nicht ausreichend definiert sind.\footnote{\cite{becker_prozesse_2010}}

Hier empfiehlt sich der Einsatz von Business Process Management. Nach dem Identifizieren möglicher Prozesse werden diese modelliert, konkretisiert und zuletzt digitalisiert.

Die Modellierung der Prozesse im ersten Schritt sollte auf abstraktem Niveau stattfinden. Dies erleichtert den Einstieg und macht Verbesserungspotenziale sichtbarer. In der WWU Münster wurde dafür die PICTURE Methode verwendet.
Im zweiten Schritt kann der ggf. angepasste Prozess konkretisiert und in Form des Industriestandards Business Process Model Notation (BPMN) digital notiert werden. Ein Client-Tool zur Erstellung von BPMN ist das Activiti BPMN 2.0 Eclipse Plugin.\footnote{\url{http://docs.codehaus.org/display/ACT/Activiti+BPMN+2.0+Eclipse+Plugin}}

Mit Hilfe der zentralen Business Process Management (BPM) Platform activiti können die Prozesse aktiv den Workflow verbessern, Konsistenz wahren und zeitliche Ressourcen sparen.  Die Plattform ermöglicht REST Anfragen, wodurch die Prozessinformationen auch in andere Applikation integriert werden können. Der Activiti Explorer ermöglicht den voll funktionalen Zugriff via Weboberfläche. Somit wird der Kernanforderung SaaS Rechnung getragen. Weiterhin ist Activiti Open Source und somit ausbau- und anpassungsfähig.\footnote{\url{http://activiti.org/index.html}}

Durch activiti wird es möglich sein die Automatisierung von einzelnen Prozessen Stück für Stück voranzutreiben indem manuelle Aufgaben gegen Automatismen ersetzt werden. Einzelne Teile des Workflows können dann automatisiert Scripte starten, E-Mails verschicken und ähnliches und somit stückweise die manuelle Bearbeitung reduzieren. Außerdem können Serviceanfragen mit diesem Tool zentralisiert verwaltet werden. Durch Definition von Pflichtfeldern für einzelne Prozessschritte können vorab benötigte Informationen festgelegt werden, sodass Nachfragen vermieden werden.

\subsubsection{Content Management}
Um den wachsenden Anforderungen in Bezug auf Content Management genüge zu tun wurde an der WWU Münster das Enterprise Content Management System alfresco eingeführt. Auch an einer kleinen Hochschule kann ein solches System eingesetzt werden. Alfresco bietet diverse Vorteile. Die für dieses Konzept Relevanten werden hier kurz aufgelistet:\footnote{\cite{kloetgen_2012}}

\begin{itemize}
	\item Unterstützung für mobile Endgeräte
	\item Anpassungs- und ausbaufähig
	\item diverse Zugriffsmöglichkeiten zur Nutzung innerhalb bekannter Standardanwendungen
	\item Publikation in sozialen Netzwerken
	\item Unterstützung verschiedener Standardschnittstellen
	\item Activiti Workflow Engine
	\item Metadaten
\end{itemize}

Alfresco bietet damit die ideale Grundlage verschiedenste Informationen zu verwalten sowie die Unterstützung des vollständigen Dokumenten-Lifecycles - von der Erstellung über die Bereitstellung bis zur Archivierung. 

Die Art des Zugriffs auf Dokumente ist dynamisch dank der Unterstützung zahlreicher Standards. Somit kann die Integration der Dokumente angepasst an die jeweiligen Anforderungen geschehen. Ein Dokument, welches an verschiedenen Orten auf verschiedene Arten bereitgestellt werden soll, kann dank Alfresco zentral aktualisiert werden und an allen Zugriffsstellen die aktuellste Version bereitgestellt wird.

Dennoch ist das System flexibel genug auch bestimmte Versionen eines Dokuments bereitzustellen. Dank der integrierten Versionierung entfällt außerdem der aufwendige Wiederherstellungsprozess. Durch die Möglichkeit Metadaten anzugeben, wird der Weg für eine brauchbare Dokumentensuche geebnet.

\subsection{Spezialsysteme}
Unter Spezialsystemen sind hier Softwarelösungen zu verstehen, die bei speziellen Aufgaben im Hochschulalltag unterstützen sollen.

\subsubsection{Verwaltungssoftware}
Software, die speziell in der Verwaltung genutzt wird, ist in dieser Ausarbeitung ausgenommen, da die Anforderungen sehr speziell sein können und die Wissensbasis innerhalb dieser Ausarbeitung um die Anforderungen der Hochschule an eine solche Software nicht ausreicht, um eine Empfehlung zu geben.

Erwähnt werden sollte dennoch, dass die Universität Karlsruhe erfolgreich Schnittstellen zu den HIS-Systemen integriert hat.\footnote{\cite{dini_webportale_2007}}

Da das Konzept dieser Ausarbeitung für die Hochschule Emden-Leer auf einem ähnlich flexiblen Ansatz basiert, wie der bereits umgesetzte Ansatz der Universität Karlsruhe, besteht die Möglichkeit, dass auch Verwaltungssoftware integriert werden kann, insofern diese die genannten Kernanforderungen erfüllt.

\subsubsection{Lernplattform}
Die Hochschule setzt bereits erfolgreich und in vielen Bereichen das System moodle ein. Die Nutzung der Funktionen variiert dabei zwischen den einzelnen Fachbereichen.
Solange moodle die Anforderungen der Hochschule an eine Lernplattform erfüllt, besteht kein Grund das System auszutauschen.

Neben den bereits genutzten Standardfunktionen ist moodle ausbaufähig.

Beim Aufruf von moodle soll die Authentifizierung durch einen SSO Mechanismus geschehen. Hat sich ein Benutzer bereits an einem anderen System authentifiziert, ist dieser beim Aufruf sofort angemeldet. Für das Lernraumsystem moodle gibt es bereits ein SSO Plugin.\footnote{\url{http://sourceforge.net/projects/moodleldapsso}} Integriert werden sollte auch der SingleSignOff Mechanismus.
Statt der Stammdatenänderung via moodle wird der entsprechende Menüpunkt ausgeblendet oder auf ein zentrales Formular weitergeleiten, um persönliche Informationen zentral und damit konsistent zu halten.

Die in den Kursen zur Verfügung gestellten Dateien jeglicher Art werden in alfresco gepflegt und von moodle angebunden. Die entsprechenden Schnittstellen und Plugins müssen nicht neu entwickelt werden.

Vorteil ist, dass die Dokumente in alfresco verwaltet werden. In moodle kann dann eine bestimmte Version oder die jeweils aktuellste referenziert werden. Bei den Dokumenten kann es sich um Textdokumente, Audio- oder auch Videodateien handeln. Durch alfrescos Unterstützung für mobile Endgeräte außerdem den Zugreifenden auch die Möglichkeit gegeben ein Dokument auf den verschiedensten Endgeräten anzuzeigen bzw. wiederzugeben.

Dank alfresco können Dokumente nicht nur innerhalb moodle via Weboberfläche aufgerufen werden, sondern auch bequem via Filesystem. Eine Datei kann somit auf verschiedene Art und Weise abgerufen werden – je nach dem welchen Weg der Anwender für den aktuell praktikabelsten hält.

\subsubsection{Publikationen}
Um Wissenschaftler bei der Publikation von Zeitschriften zu unterstützen, kann die Plattform Open Journal System (OJS) eingesetzt werden, wie es auch in der WWU Münster getan wird. Es bietet die Möglichkeit elektronische Zeitschriften zu verwalten und den gesamten Publikationsworkflow abzubilden.\footnote{\cite{kloetgen_2012}}

Grundsätzlich sollte auch ein Workflow in activiti implementiert werden, der bei der Publikation unterstützt. So können wichtige Metadaten aufgenommen und an relevante Systeme weitergegeben werden. Ändern sich Systeme oder kommen neue hinzu, müssen sich Wissenschaftler nicht umgewöhnen, sondern nutzen weiterhin den in activiti hinterlegten, für die neuen Systeme jedoch angepassten, Prozess. Dadurch besteht auch die Möglichkeit ein publiziertes Dokument zusätzlich in alfresco abzulegen, wenn ein Anwendungsfall dies benötigt.
OJS bietet die Möglichkeit der Authentifizierung via Single Sign On. Dies geschieht via Shibboleth.\footnote{\url{https://pkp.sfu.ca/wiki/index.php?title=Setting_up_authentication}}

Neben der Konfiguration von Single Sign On sollte auch hier den Benutzenden die Möglichkeit des Single Sign Off gegeben werden.

\subsubsection{Evaluation}
Wie auch die Universität Münster\footnote{\url{https://www.wiwi.uni-muenster.de/fakultaet/de/studium/lehrevaluation}} und die TU Dortmund\footnote{\url{https://www.itmc.uni-dortmund.de/dienste/e-learning/umfragewerkzeuge.html}} setzt die Hochschule Emden Leer bereits die Software EvaSys zu Evaluationszwecken ein. Sie ist webbasiert und entspricht damit dem Software as a Service Gedanken.
Seit Version 5 unterstützt EvaSys auch die SingleSignOn Authentifizierung\footnote{\url{http://www.ku.de/fileadmin/190304/Feature_Function_Benefit_EvaSys_V5.0_DE.pdf}}, welche auch an der Hochschule Emden Leer eingesetzt werden soll.

\subsubsection{Campus Portal}
Ein Campus Portal dient als zentrale Anlaufstelle für alle Hochschulangehörigen und ist ein personalisiertes Webportal. Es soll die Verwaltung persönlicher Daten ermöglichen, eine Übersicht über informationstechnische Funktionen inklusive Weiterleitung zum entsprechenden System integrieren, sowie aktuell relevante Informationen in Form einer Agenda anzeigen. Das Campus Portal soll also als Startpunkt dienen.

Unter dem Begriff informationstechnischer Funktionen sind hier alle Werkzeuge und Spezialsysteme(siehe oben) zu verstehen, die einem bestimmten Zweck dienen. Eine Selbstimplementierung für die bereitzustellenden Funktionen soll dabei vermieden werden.

Stattdessen soll, wie in den Kernanforderungen aufgenommen, auf existierende Systeme gesetzt werden, sofern dies möglich ist. Die Kernanforderungen an solche Systeme, nämlich integrierbar und systemabhängig (SaaS) zu sein, wird nun deutlich, da solche Systeme im Campus Portal integriert werden sollen.

Ein solches Campus Portal ist vor allem der Informationsübersicht dienlich. Durch personalisierte und dynamisch generierte Inhalte gewinnen die Benutzenden einen Überblick über die Informationen, die sonst in verschiedenen Systemen verteilt sind. Der Aufbau eines Campus Portals und die Integration der Spezialsysteme kann schrittweise erfolgen, sollte jedoch zur Akzeptanzgewinnung bei Veröffentlichung eine gewisse Menge externer Systeme integrieren.

Die Universität Karlsruhe hat im ersten Schritt das Veranstaltungsmanagement und die Prüfungsverwaltung der Software-Systeme der HIS GmbH in das Portal integriert.

Orientiert am Anforderungskatalog der WWU Münster an ein solches Portal und unter der Voraussetzung, dass die in diesem Konzept genannten Systeme umgesetzt werden, kann ein zukünftiges CampusPortal folgende Informationen konsolidieren:

\begin{itemize}
	\item Kalender
	\begin{itemize}
		\item Abonnement-Prinzip
		\item inklusive Detailinformationen, zum Beispiel Kontoinformationen für Rückmeldegebühren
		\item Agenda aus moodle
	\end{itemize}
	\item Referenz zu gewählten Kursen (moodle)
	\item Referenz zum E-Mail-Portal (Outlook Web App)
	\item Suchmaschine
	\item offene Tasks im BPM System
	\item Start möglicher Tasks im BPM System (zum Beispiel Workflow für die Publikation)
	\item offene Evaluationen
\end{itemize}

Neben den in diesem Konzept genannten Systemen, könnten weitere Spezialsysteme in das Campus Portal integriert werden. Der Anforderungskatalog der WWU Münster enthält außerdem:

\begin{itemize}
	\item Stundenplan
	\item Vorlesungsverzeichnis inkl. Details
	\item Leistungsübersicht
	\item Hochschulleben
	\begin{itemize}
		\item Mensapläne
		\item Hochschulsport
		\item Veranstaltungen
	\end{itemize}
	\item Einführung in die Benutzung
\end{itemize}

Alternativ zur Einführung in die Benutzung kann auch das Konzept eines Hilfesystems integriert werden, das via Sprechblasen Hilfestellungen oder Erläuterungen anzeigt zur jener Funktion, auf der sich der Mauszeiger gerade befindet.

Die technische Struktur kann analog zu der des Portals myWWU der Universität Münster aufgebaut sein, wie im Folgenden abgebildet.
\begin{figure}[h!]
	\centering
	\includegraphics[width=10cm]{kapitel/gruppe3/bilder/struktur_mywwu}
	\caption{technische Struktur des Portals myWWU der Universität Münster}
	\label{fig_struktur_mywwu}
\end{figure}

\subsubsection{Integrierte Gesamtsuche}
Wissenschaftliche Informationen sind häufig in verschiedenen Systemen angesiedelt. Durch die Einführung eines Systems zur integrierten Gesamtsuche würde die Suche an zentraler Stelle ausgeführt. Einzelne Systeme werden beim Suchen nicht vergessen und die Integration neuer Systeme als Informationsbasis des wissenschaftlichen Umfelds wird dauerhaft kommuniziert statt einmalig, wie es beispielsweise beim Versand einer Info-E-Mail der Fall wäre.

Die WWU Münster nutzt dafür einen „Suchmaschinen-basierte[n] Ansatz auf Basis der Software Primo von der Firma Ex Libris“.\footnote{\cite{vogl_fortschritte_2012}} Dafür nötig ist eine Normalisierung der Datenformate interner und externer Quellen, welche „insbesondere auf der Detailebene […] aufwändige Anpassungen und Eigenentwicklungen notwendig“ machen.

Quellsysteme ausgehend von diesem Konzept können sein:
\begin{itemize}
	\item alfresco
	\item Identity Management System
	\item ForschungsDB Niedersachsen
	\item moodle
	\item Open Journal System
	\item Hochschulexterne Informationssysteme wie zum Beispiel video2brain
	\item ggf. Bibliothekssuche
\end{itemize}

Wichtig ist neben korrekten und vollständigen Ergebnissen auch die Benutzbarkeit. Bekannte Funktionalitäten aus anderen Suchmaschinen, wie Gruppierungen, sollten integriert sein, wie auch eine übersichtliche und funktionale Benutzeroberfläche.

\subsection{Ausblick bei Integration der Bibliothek}
Auch wenn die Bibliothek in dieser Ausarbeitung ausgenommen ist, muss erwähnt werden, dass bei Informationsmanagement-Projekten anderer Hochschulen und Universitäten auch die Bibliotheken integriert werden. Die über die Bibliothek zur Verfügung gestellten Informationen werden vor allem für Forschung und Lehre genutzt, welches die Kernaufgaben von Hochschulen sind.

Dementsprechend kann die Integration der Bibliothekssuche in die integrierte Gesamtsuche zur Aufwertung der Suchergebnissen beitragen. Um auch nicht digital verfügbare bzw. archivierte Zeitschriften und Bücher integrieren zu können, kann ein Digitalisierungssystem eingeführt werden. Die WWU Münster setzt dafür vor Ort frei verfügbare Scanner zuzüglich der Software scantoweb ein.

\section{Abwägung des Einsatzes eines Informationsmanagers an der Hochschule Emden/Leer - JL}
\label{section_einsatz_cio}

Das Soll-Konzept analysiert die Ist-Situation um den derweilen Zustand der  Hochschule Emden/Leer zu ermitteln. Daraus lässt sich erkennen, ob es generell einer Verbesserung des Informationsmanagements in der Zukunft bedarf und wo diese anzusetzen sind oder ob noch kein Informationsmanagement besteht und aufgebaut werden muss. Dazu sind verschiedene Aspekte zu beleuchten. Neben der Anforderung des zukünftigen Marketings, den technischen Neuerungen und der darauf folgenden Umsetzung ist zu klären, ob die Hochschule Emden/Leer eine Führung im Bereich des strategischen und operativen Informationsmanagements benötigt.

Im klassischen Informationsmanagement ist dies die Aufgabe eines Informationsmanager, dem sogenannten Chief Information Officer. Wie auf der Abbildung \ref{fig_def_inm} zu erkennen, arbeitet der Informationsmanager dabei als zentrale Schnittstelle zwischen technischen, organisatorischen und wirtschaftlichen Teilbereichen und dient dort als sogenannter Mittler zwischen den verschiedenen Bereichen und untersucht dabei die Informations- und Kommunikationstechniken in allen unterschiedlichen Bereichen um diese sinnvoll einzusetzen. 
\footcite[86]{definition_informationsmanager}

\begin{figure}[h!]
	\centering
	\includegraphics[width=10cm]{kapitel/gruppe3/bilder/definition_informationsmanager}
	\caption{Definition Informationsmanager}
	\label{fig_def_inm}
\end{figure}

\subsection{Analyse des Ist-Zustandes}
\label{subsection_analyse_ist_zustand}

Bezug nehmend auf das Organigramm aus Abbildung \ref{fig_organigramm_HS} der Hochschule Emden/Leer und der Bewertung aus  \ref{section_bewertung_gewichtung} ist festzuhalten, dass der Hochschule kein klassisches Informationsmanagement zugrunde liegt, sondern ein zentrales Informationssystem. Es werden bereits Informationen verwaltet und weitergegeben, jedoch nicht an zentraler Stelle. Zentrale Systeme, siehe Abbildung \ref{fig_zentrale_systeme}, Kapitel \ref{section_zustaendigkeiten}, sowie unterschiedliche Möglichkeiten werden für alle zur Verfügung gestellt und in Anspruch genommen. 

Es gibt keine Verwaltung, sondern verschiedene Bereiche, die unterteilt sind in Arbeitsgruppen, Abteilungen sowie Rechenstelle und Pressestelle. Weiterhin beinhaltet das Informationssystem verschiedene Prozesse zum Datenaustausch bzw. Datenfluss und Back-up-Transfer aus verschiedenen Systemen.  Die Nutzung des gegenwärtigen Informationssystems wird unterschiedlich stark genutzt oder ausgelastet. 

Von den zentralen Einrichtungen nehmen das Hochschulrechenzentrum und die Bibliothek einen wichtigen Platz in der Hochschule Emden/Leer ein. Das Hochschulrechenzentrum übernimmt derweil viele Aufgaben der Informationsverwaltung und Planung. Doch nicht nur da werden Informationen gesammelt und ausgewertet. Die Hochschule in Emden definiert eine ganze Reihe von Arbeitsgruppen, beispielsweise die Arbeitsgruppe Zahlen, Daten, Fakten, die Kennzahlen der Hochschule und der einzelnen Fachbereiche sammelt und diese auswertet und an gewünschte Stellen weitergibt.  

Aktuell besteht keine erweiterte Vernetzung unterschiedlicher Intranetzsysteme zwischen verschiedenen Hochschulen. Lediglich im Bereich der Bibliothek werden Inhalte an mehreren Standorten gemeinsam genutzt. Abschließend ist zu erwähnen, dass die Hochschule Emden/Leer auch keine Einzelperson oder ein Gremium als zentrale Informationsverwaltung nutzt.


\subsection{Betrachtung des zu erwartenden Soll-Zustandes }
\label{subsection_betrachtung_soll}

Nach Betrachtung der Best-Practice-Beispiele aus Kapitel \ref{chapter_best_practice_beispiele} lässt sich erkennen, dass jede Hochschule und auch Universität den Umgang des Informationsmanagements anders angeht. So spielen verschiedene Faktoren eine Rolle, die an jeder Hochschule/Universität unterschiedlich ausgelegt sind. Ein Vergleich der betrachteten Universitäten mit der Hochschule Emden/Leer zeigt, dass Emden eine wesentlich kleinere Institution ist und somit andere Ansprüche hat und weniger komplexe Strukturen besitzt, als beispielsweise die WWU Münster, die über 40.000 Studierende pflegt. 

Trotz unterschiedlich integrierter Möglichkeiten der unterschiedlichen Universitäten zur Umsetzung des jeweiligen Informationsmanagements gibt es doch Bereiche, die gleich oder zumindest ähnlich sind. So sind Bibliotheken, Gremien, Ausschüsse, ebenso wie Fachbereiche, das Rechenzentrum und auch das Präsidium Teil einer jeden Hochschule oder Universität. 

Es ist also zu schauen, wo sich das Informationsmanagement als zentrale Informationsquelle ansetzen lässt, um mehrere Bereiche und Bestandteile untereinander zu verbinden. Fakt ist, dass es in Emden bereits drei Arbeitsgruppen gibt, die bestimmte Informationen gewinnen und filtern.  So wäre zu betrachten, wie die Zentralisierung einer übergeordneten Informationsquelle und -weitergabe zu bewerkstelligen wäre und wie der Aufbau einer neuen Struktur die Möglichkeit zur Verbesserung des Informationsaustausches aussehen könnte. 

In Kapitel \ref{subsection_zentralisierung_integration} wird beschrieben, dass die meisten Hochschulen und Universitäten unter einer neu geschaffenen Organisation arbeiten. Dabei spielen das Rechenzentrum, die Bibliothek und die Verwaltung immer eine Rolle in einer solchen Organisation. Kein Konzept ist maßgeschneidert und nicht auf jede Hochschule oder Universität anwendbar.

\subsubsection{Empfehlungen der ZKI bezüglich des Informationsmanagers}
\label{subsubsection_zki}

Neben den verschiedenen Projekten und Einrichtungen, die im Kapitel \ref{chapter_best_practice_beispiele} beleuchtet werden, und aufzeigen, wie mit dem Informationsmanagement umgegangen wird, gibt es noch die Zentren für Kommunikation und Informationsverarbeitung (ZKI) in Lehre und Forschung, die Empfehlungen für Hochschulen bezüglich des Informationsmanagements und besonders Empfehlungen für den Informationsmanager aussprechen.

Blickend auf die Publikation der ZKI basierend auf einer Studie der CIOs und IT-Governance an deutschen Hochschulen aus dem Jahre 2014 wurden über mehrere Jahre von der Kommission für IT-Infrastruktur der Deutschen Forschungsgemeinschaft (KfR) hinweg folgende Empfehlungen für Hochschulen ausgesprochen.

Zwischen 2001 - 2005 gab die KfR folgende Empfehlung:

\textit{" Aufgrund der Relevanz der Informationsverarbeitung für alle Bereiche der  Hochschule wird empfohlen, 
	einen Generalverantwortlichen für Information und  Kommunikation (CIO, Chief Information Officer) 
	in der Hochschulleitung oder ein geeignetes Leitungsgremium mit entsprechenden 
	Entscheidungskompetenzen mit der Entwicklung und  Koordinierung aller IuK-Aufgaben 
	zu betrauen."}\footcite[3]{zki_studie_cio_2014}

Zwischen 2006 - 2010 werden weitere Ausführungen genannt:

\textit{" Integriertes Informationsmanagement ist daher zur wesentlichen Aufgabe bei der Planung des 
	Einsatzes moderner Techniken von Information und Kommunikation für die Hochschulen geworden. 
	Eine solche Planung setzt die Position eines Verantwortlichen für Information und Kommunikation 
	als Mitglied der Hochschulleitung (CIO: Chief Information Officer) voraus, wie er in der Wirtschaft 
	und an verschiedenen Hochschulen bereits etabliert ist."}\footcite[16]{zki_studie_cio_2014}

Die KfR Empfehlungen zwischen 2011-2015 werden noch weiter ausgebaut:

\textit{"In der Hochschulpraxis lassen sich vier unterschiedliche Umsetzungstypen beobachten:  Strategischer CIO mit Leitungsfunktion: Ein Vizepräsident - oder eine Vizepräsidentin - ist explizit für das Informationsmanagement zuständig. Teilweise übernimmt auch der Kanzler die Zuständigkeit für das Informationsmanagement.}

\begin{itemize}
	\item \textit{Strategischer CIO mit Stabsfunktion: Ein Hochschullehrer oder IT-Manager - 
		bzw. Hochschullehrerin/IT-Managerin - im Präsidialstab koordiniert das Informationsmanagement.} 
	\item \textit{Operativer CIO: Der Leiter - bzw. die Leiterin - einer zentralen 
		Informationsinfrastruktureinrichtung fungiert gleichzeitig als CIO der Hochschule.}
	\item \textit{Kollektiver CIO: Die CIO-Funktion wird von einem Lenkungsausschuss mit zwei bis 
		drei Personen ausgeübt, der allerdings - anders als die traditionelle Senatskommission - über 
		unmittelbare Entscheidungsbefugnisse verfügt.}
\end{itemize}
\textit{Jede dieser CIO-Umsetzungsvarianten hat ihre Vor- und Nachteile. Es hängt von den 
	Gegebenheiten an den Hochschulen und insbesondere auch von Personen ab, welche 
	Umsetzung die am besten geeignete ist. Wichtig ist, dass der CIO - in welcher Form 
	auch immer - einen unmittelbaren Zugang zur Hochschulleitung hat und die IT-Belange 
	der gesamten Hochschule strategisch - mit unmittelbarer Richtlinien- und 
	Entscheidungskompetenz - fährt und verantwortet."} \footcite[17]{zki_studie_cio_2014}

Abschließend ist zu sagen, dass die ZKI/KfR einer Hochschule eine zentrale Informationsschnittstelle in Form eines CIOs oder eines CIO-Gremiums empfiehlt.

\subsubsection{Informationsmanager oder Gremium als zentrale Informationsschnittstelle}
\label{subsubsection_cio_gremium}

Der Aufbau eines Informationsmanagements bedarf vieler Schritte und Überlegungen (siehe Abschnitt \ref{begriffsdefintion_inm}). Neben den Veränderungen und deren Umsetzung ist zu klären, ob der bisherige Austausch der Informationen der Hochschule Emden/Leer durch eine zentrale Einrichtung oder einer Einzelperson und entsprechenden Verantwortlichkeiten geregelt werden soll. Um dies in ein reales Szenario zu bekommen, sind die Möglichkeiten aufzuzeigen und ein entsprechend passendes Modell für die Hochschule Emden/Leer zu entwickeln. Dazu werden in Abschnitt  \ref{section_betr_hochschule}, ebenso wie in der Studie der ZKI verschiedene Konzepte des Chief Information Officers (CIO) aufgezeigt. 

Ein einheitliches Konzept ist nicht gegeben, sodass nicht jede Lösung auch passend für die Hochschule Emden/Leer ist. Die betrachteten Universitäten haben ein anderes Anforderungsprofil an ein Informationsmanagement und deren zentrale Leitung als Emden, die wesentlich kleinere und weniger komplexe Strukturen besitzt. Zu den betrachteten Best-Practice-Beispielen lassen sich zusätzlich die Empfehlungen der KfR heranziehen. 

Alle haben gemeinsam, dass das Verwalten der Informationen aus einer Schnittstelle heraus geschieht. Auch dieses Konzept ist für die Hochschule Emden/Leer zu überlegen. Nun stellt sich die Frage, wo diese Schnittstelle anzusetzen ist und wer die Aufgaben übernehmen soll. Verschiedene Möglichkeiten sind hier zu betrachten. Zum einen gibt es das Personenmodell, den CIO, beschrieben in \ref{anforderungsprofil_informationsmanager} und \ref{eingliederung_informationsmanager} der die Schnittstelle bildet, zum anderen gibt es die Möglichkeit eines CIO-Gremiums. 


\textbf{Personenmodell:\newline}
Eine Person wird als Informationsmanager herangezogen und übernimmt die in den Abschnitten \ref{aufgaben_funktionen_informationsmanager}, \ref{anforderungsprofil_informationsmanager} und \ref{eingliederung_informationsmanager} angegebenen Aufgaben, die hochschulangepasst sind. Dabei ist zu betrachten, wer diese Aufgabe übernehmen soll. Der CIO kann aus der Privatwirtschaft geordert werden. Dabei ist zu bedenken, dass dafür eine Menge Ressourcen benötigt werden. Neben dem aufwendigen Bewerbungsprozess und der Einstellung erfolgt die Einarbeitungszeit und die Einführung des Informationsmanagements durch den CIO. Als weiterer Punkt sind noch die erhöhten Personalkosten in dieser Zeit zu nennen.

Neben der Möglichkeit einen CIO aus der Privatwirtschaft zu holen, besteht auch die Möglichkeit einen hochschulinternen Mitarbeiter zu involvieren. Der Bewerbungszeitraum und die Einarbeitung verringern sich, da ein bestehender Mitarbeiter die Hierarchie und die Arbeitsweise der Hochschule Emden/Leer bereits versteht und kennt. Allerdings ist nicht zu verachten, dass diese Person, entweder eine Mehrbelastung durch die zusätzlichen Aufgaben des Informationsmanagers tragen muss oder für die vorherige Stelle ein neuer Mitarbeiter gesucht werden müsste, was auch in diesem Fall mit einem erhöhten Kosten- und Zeitaufwand verbunden wäre.


\textbf{Gremiummodell:\newline }
Soll das Informationsmanagement allerdings nicht nur von einer einzelnen Person betrieben werden, ist zu klären, wer diese Aufgabe übernehmen soll. Dazu ist immer in Vergleich zu setzen, welche Parameter greifen. Die Studie der ZKI besagt, Bezug nehmend auf die Abbildung \ref{fig_herkunft_cio_hochschulen}, dass die Gremienmitglieder aus ganz unterschiedlichen Bereichen der Hochschule kommen. Ist dies der Fall und ein Gremium wird ernannt, ist ein Arbeitsaufwand der anfallenden CIO Tätigkeiten auf alle Mitglieder aufgeteilt. So ist der Gesamtaufwand pro Person prozentual geringer als bei einer einzelnen Person, die mindestens 50\% ihrer Zeit in CIO-Aufgaben investiert. 



\begin{figure}[h!]
	\centering
	\includegraphics[width=\textwidth]
	{kapitel/gruppe3/bilder/herkunft_cio_hochschulen}
	\caption{Herkunft des CIO an verschiedenen Hochschulen, nach ZKI CIO-Studie}
	\label{fig_herkunft_cio_hochschulen}\footnote{\cite[8]{zki_studie_cio_2014}}
\end{figure}



\subsection{Empfehlung für die Hochschule Emden/Leer}
\label{empfehlung_cio}

Durch stetig wachsende Anforderungen besonders im Bereich der technischen Neuerungen und deren Umsetzung sowie Weitergabe und Verarbeitung von Informationen spricht die KfR seit Jahren Empfehlungen bezüglich eines Informationsmanagers an Hochschulen aus. Durch zusätzliches Betrachten der Best-Practice-Beispiele wird gezeigt, dass jede Hochschule andere Anforderungen besitzt und bezüglich ihrer Größe, Lage und Ansprüche anders mit einem Informationsmanagement umgeht, jedoch alle gemeinsam haben, dass eine zentrale Schnittstelle gebildet wird, die zusammenlaufende Informationen verarbeitet, auswertet und weitergibt.

Nicht jede Lösung eignet sich dabei für jede Hochschule. In einer Studie der ZKI geht dies ebenfalls hervor. Die Studie befasst sich mit dem Informationsmanager und spricht dabei Empfehlungen für Hochschulen aus. Dabei ist festzuhalten, dass es neben dem Einzelpersonen-Modell auch ein CIO-Gremium-Model geben kann, je nach Bedürfnis der Hochschule. Eine Einzelperson kann hierbei vorteilhafter sein als ein Gremium, dennoch ist zu betrachten, dass ein enormer personeller Kosten- und Zeitfaktor entstehen wird, da nicht zu verachten ist, dass das Aufbauen einer solchen Struktur Jahre in Anspruch nimmt. Es ist daher abzuwägen, ob sich dieser finanzielle Aufwand für die Hochschule Emden/Leer lohnt.

Da Emden bereits die drei Arbeitsgruppen Zahlen, Daten und Fakten, Web und Moodle besitzt, detaillierter beschrieben in \ref{subsection_arbeitsgruppen_informationsaustausch}, die wichtige Informationen sammeln und verarbeiten, wäre der Hochschule Emden/Leer eine Empfehlung zu einem CIO-Gremium auszusprechen. Durch die bereits existierenden Arbeitsgruppen ist aus jedem Bereich bereits ein Vertreter vorhanden. 

Die Hochschule Emden/Leer ist diesbezüglich schon gut aufgestellt, um weitere Schritte beim Einführen dieses Konzeptes einleiten zu können. Die anfallenden Aufgaben werden auf das gesamte Gremium aufgeteilt, sodass eine geringere Mehrbelastung entsteht. Abb. \ref{fig_moegliches_gremium} zeigt die Umstellung des Organigramms der Hochschule Emden/Leer, als mögliche Organisation. Das Gremium unterliegt dabei der Hochschulleitung. Da das Rechenzentrum bereits wichtige und zentrale Aufgaben besonders im technischen Bereich übernimmt, wäre es sinnvoll das Gremium aus dem Bereich heraus zu gründen. 

\begin{figure}[h!]
	\centering
	\includegraphics[width=\textwidth]
	{kapitel/gruppe3/bilder/moegliches_cio_gremium}
	\caption{Umstellung/Änderung des Organigramms der Hochschule Emden 	hinsichtlich eines CIO Gremiums}	
	\label{fig_moegliches_gremium}
\end{figure}

In Verbindung mit den drei bereits existierenden Arbeitsgruppen würde sich eine Mischform zwischen strategischem und kollektivem CIO für die Hochschule Emden/Leer anbieten. Das bietet die Möglichkeit sich den Gegebenheiten der Hochschule anzupassen und ein auf Emden/Leer wäre, dass das CIO-Gremium eng mit dem Marketing zusammenarbeitet und somit von der vorgestellten Möglichkeit Feedback zu sammeln, beschrieben in \ref{feedback} profitieren würde. So können anfallende Probleme direkt diskutiert und Lösungen gefunden werden. 




\chapter{Umsetzungsplanung: Change Management und Migration - MB, CH}
\label{chapter_changemanagement_migration}
\textit{Autoren: Marco Beckmann (MB), Christian Halfmann (CH)}

Im Folgenden soll im Rahmen der Umsetzungsplanung erläutert werden, wie zum  einen Veränderungen unter Berücksichtigung der Betroffenen mittels Change Management initiiert und durchgeführt werden können, und wie zum anderen mittels geeigneter Migrationskonzepte die technische Umsetzung realisiert werden kann.

\section{Umsetzungsplanung}
Ich bin ein einleitender Text, der noch geschrieben werden muss.

\subsection{Positionsbestimmung}
Für einen erfolgreichen Umsetzungsplan mit der Zielsetzung einer Neuordnung des Informationsmanagements an einer Hochschule ist eine Positionsbestimmung der aktuellen Situation von elementarer Bedeutung. Hierzu muss der Ist-Zustand des aktuellen Informationsmanagements mit der Zielformulierung des avisierten Informationsmanagements an der Hochschule erfasst und abgeglichen werden. 

Da solche Veränderungen in der Regel einen langwierigen Prozess darstellen, ist es ratsam, Prioritäten zu definieren und die einzelnen Teilbereiche anhand der Dringlichkeit umzusetzen.

Ist die Position bestimmt, kann davon ausgehend ein entsprechender Migrationsplan (vgl. Kapitel \ref{section_migrationskonzepte}) und, wenn noch nicht geschehen, ein Changeplan (vgl. Kapitel \ref{subsection_change_management}) erstellt werden. Je nach Art und Umfang der Veränderungen sollte allerdings das Change Management nicht erst nach der Positionsbestimmung angewandt werden, sondern schon bei der Zielbestimmung – also mit in die Erarbeitung des möglichen Soll-Zustands einfliessen.

Diese Ausarbeitung wird sich aus Gründen der Komplexität im praxisbezogenen Teil nicht auf das gesamte Informationsmanagement der Hochschule Emden/Leer beziehen können. Exemplarisch wird daher eine Umsetzungsplanung an den Beispielen des Dokumentenmanagements Alfresco und der Erstellung eines responsive Designs der Webpräsenz der Hochschule erarbeitet.

Alfresco wird derzeit noch nicht an der Hochschule eingesetzt. Zur Zeit werden Dokumente in verschieden Systemen verwaltet und zugäglich gemacht. Für die Webpräsenz wird derzeit ein TYPO3-System in der Version 4.5 LTS (Long Term Support) genutzt, welches noch nicht für mobile Endgeräte optimiert ist.

\subsection{Change Management}
\label{subsection_change_management}

Ich bin ein einleitender Text, der noch geschrieben werden muss.

\subsubsection{Grundlagen des Change Managements}
Die Umsetzung einer Neuordnung des Informationsmanagements an einer Hochschule bedeutet auch Wandel und Veränderungen. Um das optimal zu steuern, bedarf es spezieller Managementtechniken, welche sich unter dem Begriff Change Management zusammenfassen lassen. Im Vordergrund aller Betrachtungen steht der Faktor Mensch, denn für eine erfolgreiche Umsetzung von Veränderungen ist die aktive Unterstützung der Betroffenen von erheblicher Bedeutung.\footnote{\cite{lauer_change_2014}}

Nach Thomas Lauer sollte das Change Management grundsätzlich an drei Punkten ansetzen:
\begin{itemize}
	\item Individuum: Das Individuum beschreibt jeden Einzelnen. Ohne die Mitarbeit der Betroffenen ist ein Wandel unmöglich. Das Change Management soll nicht nur die Fähigkeiten des Einzelnen an neue Herausforderungen anpassen, sondern auch die positive Einstellung gegenüber der Ziele des Wandels aller Betroffenen fördern.
	\item Unternehmensstruktur (bzw. Hochschulstruktur): Die Unternehmensstruktur umfasst Aufbau- und Ablauforganisation sowie Strategien und Ressourcen. Veränderungen in diesen Bereichen sind auf dem Papier grundsätzlich einfach.
	\item Unternehmenskultur (bzw. Hochschulkultur): Die Unternehmenskultur beschreibt dauerhafte, über lange Zeit gewachsene Strukturen die für Einstellung, Werte und Regeln des Umgangs verantwortlich sind. 
\end{itemize}

In den meisten Fällen bringt ein Wandel in den oben genannten Bereichen Veränderungen in allen Dimensionen mit sich, die sich wechselseitig beeinflussen.\footnote{\cite{fisch_veraenderungen_2008}} 

So ist z. B. ein Wandel ohne die Einbeziehung der Unternehmenskultur oftmals zum scheitern verurteilt.\footnote{\cite{lauer_change_2014}} Das heißt, für ein erfolgreiches Change Management sollten grundsätzlich die Abhängigkeiten der Bereiche untereinander berücksichtigt werden.

Veränderungen bedeuten Neues und Ungewohntes für alle Betroffenen. 
Betroffene müssen veränderte Aufgaben erledigen, neue Technologien und Methoden erlernen, erneut soziale Beziehungen zu Kollegen, Vorgesetzten oder Kunden aufbauen, mit Problemen in der Implementierungsphase umgehen und ggf. ihre Werte in Einklang mit neuen Standards und Zielen der Organisation bringen.\footnote{\cite{fisch_veraenderungen_2008}}

Dies kann zu Zweifeln und Widerständen führen, was im schlimmsten Fall ein Scheitern des gesamten Vorhabens bedeuten kann. Als Ursache eines gescheiterten Wandels steht der Widerstand an oberster Stelle. Mangelhafte Prozessteuerung, zu schnelles Veränderungstempo und unklare Zielsetzungen spielen dabei ein wichtige Rolle und können Gründe für eben diesen Widerstand sein.\footnote{\cite{lauer_change_2014}}

Die Bereitschaft zum Wandel nimmt zu, wenn die Betroffenen überzeugt sind, das die Veränderungen ihnen persönlich nutzen, ihre Identität nicht bedroht ist und ihre Werte und Ziele mit dem Wandel in Einklang gebracht werden können.

Des weiteren wird die Bereitschaft zum Wandel gefördert wenn die Betroffenen über Fähigkeiten verfügen, die den veränderten Anforderungen gerecht werden. Die Aufgabe des Change Management ist es, durch Information, Partizipation, Unterstützung (z. B. Coaching) und Anreizgestaltung den Betroffenen die Zweifel und Unsicherheit zu nehmen.\footnote{\cite{fisch_veraenderungen_2008}} 

\begin{itemize}
\item \textbf{Kommunikation}
Nach Thomas Lauer ist einer der entscheidendsten Erfolgsfaktoren des Change Managements die Kommunikation. Kommunikation schafft Transparenz und damit Orientierung und dient damit auch der Beilegung von Widerständen. Damit ist aber auch ein potenzieller Misserfolg eines Change Managements auf die Kommunikation zurückzuführen. Fehlinterpretationen und Missverständnisse können schnell zu Konflikten führen.\footnote{\cite{lauer_change_2014}}

Es sollten also entsprechende Kommunikationsstrategien und Kommunikationspläne erarbeitet werden, um die Betroffenen für die Veränderungen zu gewinnen und Missverständnissen aus dem Weg zu gehen. In der Startphase sollten die Betroffenen über Gründe, Ziele, Notwendigkeit, Nutzen und den zeitlichen Ablauf informiert werden. Aber auch potentielle Risiken und Schwierigkeiten sollten von Anfang an offen kommuniziert werden.\footnote{\cite{fisch_veraenderungen_2008}}

In der Durchführungsphase ist es wichtig die Kommunikation aufrecht zu erhalten. Beispielsweise können Projektfortschritte regelmäßig an alle Betroffenen weitergegeben werden. Durch das Aufrechterhalten der Kommunikation können frühzeitig Widerstände erkannt und überwunden werden.\footnote{\cite{lauer_change_2014}} 

\item \textbf{Partizipation}
Ein weiterer wichtiger Erfolgsfaktor ist die Partizipation. Das Einbinden möglichst vieler Betroffenen in den Change Prozess erhöht die Motivation und hilft den Betroffenen sich mit den Veränderungen zu identifizieren. Haben Betroffene andere Positionen oder Sichtweisen gegenüber des Wandels als die Organisation müssen diese Widerstände nicht gleich negativ ausgelegt werden. Die Ideen und Vorschläge der Betroffenen können in den Change Prozess einfliessen und weiterhin Veränderungen optimieren.

\item \textbf{Unterstützung}:
Besonders wenn es um neue Technologien, Werkzeuge oder Verfahren geht, ist Unterstützung für die Betroffenen gefordert. Die Unterstützung hat zum Ziel, die Betroffenen des Wandels auf die zusätzlichen oder neuen Anforderungen vorzubereiten. In den Meisten Fällen geschieht das in Form von Weiterbildung oder Coaching.

Des weiteren fördert eine vom Unternehmen ausgehende Weiterbildung nicht nur den Aufbau von Qualifikationen und die Erweiterung des Wissens der Betroffenen, sondern auch die Motivation. Den Betroffenen wird das Gefühl gegeben, dass in sie investiert wird und damit auf eine langfristige Partnerschaft gesetzt wird.
\end{itemize}

In dem Sinne ist die Aufgabe des Change Managements also nicht die Definition von Zielen, es soll den Weg des gesamten Vorhabens vom Ausgangspunkt bis zum Ziel gestalten\footnote{\cite{lauer_change_2014}} und den Betroffenen des Wandels ihre Zweifel nehmen.

Auch bei perfekt geplanten Change-Projekten können Widerstände nicht ausgeschlossen werden. Das Change Management sollte in der Lage sein auf diese Widerstände zu reagieren. Sollte sich in dem laufenden Change Prozess herausstellen, dass bestimmte Bedingungen nicht mehr aktuell sind, sollten Ziele und Veränderungen angepasst und neu formuliert werden.\footnote{\cite{fisch_veraenderungen_2008}}

Der Wandel kann nur gelingen wenn die Betroffenen hinter den Plänen stehen und die Veränderungen unterstützen. Im Fall der Hochschule treten einige Besonderheiten auf, auf welche im nächsten Kapitel genauer eingegangen wird.

\subsubsection{Change Management an Hochschulen}
Im vorherigen Kapitel wurden die Adressaten des Change Managements als Betroffene betitelt. Diese sind im klassischen Fall Mitarbeiter eines Unternehmens, in dem Veränderungen vorangetrieben werden sollen. Diese Mitarbeiter sind  meist Bestandteil einer klaren Hierarchie, an dessen oberster Stelle das Management steht, von welchem der Wandel initiiert wird. 

Im speziellen Fall von Hochschulen setzen sich die Betroffenen aus Professoren, wissenschaftlichen Mitarbeitern, Verwaltungsmitarbeitern und Studierenden zusammen, welche autonome Endscheidungen treffen. Studenten entscheiden, was sie lernen, Dozenten entscheiden welche Inhalte sie lehren.\footnote{\cite{hoelscher_wissenschaft_2011}}

Hinzu kommen Fakultäten, Fachbereiche und Institute, welche sich selbst organisieren und nahezu autonom und unabhängig von einander agieren.\footnote{\cite{fisch_veraenderungen_2008}} 
Dies erschwert die Kommunikation untereinander sowie das Erschaffen von Synergien und das Entwickeln übergeordneter Ziele und Strategien.

Ein erfolgreiches Change Management muss also die besonderen Gegebenheiten der Organisation Hochschule bei der Gestaltung und Auswahl entsprechender Maßnahmen berücksichtigen und auf sie eingehen.

Hierzu sollten die Betroffenen innerhalb der Hochschule frühzeitig in die Zielformulierung von Change Prozessen eingebunden werden. So kann Raum für Diskussionen geschaffen werden, denn die unterschiedlichen Bereiche der Hochschule vertreten oft unterschiedliche Interessen was zur Verschleppung oder Verzögerungen von Entscheidungen führen kann. In solchen Fällen kann ein Austausch mit externen Experten oder internen Stäben helfen, Entscheidungen voranzutreiben.\footnote{\cite{fisch_veraenderungen_2008}}

Studien zu Change Management an Hochschulen haben herausgefunden, dass auch hier Information und Partizipation wichtige Elemente des Change Managements sind:
\begin{itemize}
	\item Eine Studie zur Evaluation der Strategieumsetzung an der Universität Heidelberg könnte belegen, dass Partizipation und die Qualität der Information positive Auswirkungen gegenüber Veränderungen bei den wissenschaftlichen Mitarbeitern und Studierenden hatte. Je besser die Betroffenen über die Ziele der Veränderungen informiert wurden und desto mehr eigene Ideen sie in die Veränderungen einbringen konnten, desto eher wurde der Wandel positiv bewertet und die Bereitschaft gesteigert, aktiv an der Umsetzung mitzuwirken.\footnote{Quelle fehlt}
	
	\item Bei Veränderung des Curriculums und der Einführung neuer Prozesse und Strukturen des Qualitätsmanagements an einem amerikanischen Collage zeigte sich, dass es nicht nur darum geht, Partizipation zu erhöhen, sondern auch darum, Lehrende so anzuleiten, dass Entscheidungen nicht zu autoritär getroffen werden, noch durch zu starke Gleichberechtigung in die Länge gezogen oder gar verhindert werden.\footnote{Quelle fehlt}
	
	\item Eine weitere Studie zur Implementierung von E-Learning konnte belegen, dass integratives Change Management erforderlich ist, um Veränderungen nachhaltig zu implementieren. Wurden Maßnahmen wie z. B. Training, Beratung oder didaktische Szenarien aufeinander abgestimmt, wirkt sich das positiv auf die Nutzung von E-Learning aus.\footnote{Quelle fehlt} 
\end{itemize}

Aus den Studien wird ersichtlich, dass Information und Partizipation wichtige Element des Change Managements darstellen. Aber auch Schulungen, Trainings, oder Coaching spielen eine große Rolle. Allerdings kann es zur Herausforderung werden, potenzielle Teilnehmer für Weiterbildungen aus dem Kreise der Professoren oder der Hochschulführung zu gewinnen, da diese auf ihrem Fachgebiet als Experten gelten und eine Teilnahme an solchen Weiterbildungsangeboten als Ausdruck persönlicher Defizite werten könnten. Dennoch bietet es sich an, bei komplexen Veränderungen zusätzliche Kompetenzen durch Training oder Coaching zu erschließen.\footnote{\cite{fisch_veraenderungen_2008}}

Durch die besonderen Strukturen und Gegebenheiten einer Hochschule, muss  ein potentielles Change Management möglichst sensibel agieren und alle relevanten Akteure informieren und partizipieren lassen, um am Ende auch den gewünschten Erfolg und somit die avisierten Ziele zu erreichen.

\subsubsection{Changeplan}
Wie Eingangs erwähnt, kann hier nicht auf das gesamte Informationsmanagement der Hochschule eingegangen werden. Daher wird der Changeplan sich exemplarisch auf die Beispiele Alfresco und Responsive Design beziehen, wobei es sich in beiden Fällen um Veränderungsprozesse im IT-Bereich handelt.

Der Changeplan soll unter Betrachtung der beschrieben Grundlagen des Change Managements sowie der besonderen Rahmenbedingungen an Hochschulen erstellt werden.

\ldots

\paragraph{Responsive Website}
Im Rahmen der Erstellung eines responsive Designs der Webpräsenz der Hochschule soll gleichzeitig eine aktuelle TYPO3 Version migriert werden.

Beide Vorhaben stellen eine technische Migrationen dar. Inhalte und Funktionen  der Webpräsenz werden von Veränderungen nicht betroffen sein. Lediglich im Layout, welches durch die responsive Implementierung für alle Medien optimal dargestellt wird, werden leichte Veränderungen wahrzunehmen sein.

Bei den Anwendern wird es dadurch keine Veränderungen bei Prozessen, Arbeitsweisen oder dem benötigtem Wissen geben. Daher ist auf psychologischer Ebene also kein umfangreiches Change Management von Nöten, da hier auch nicht mit Widerständen zu rechnen ist.

Jedoch bietet es sich an, bei einer Neu-Implementierung auch eventuelle Verbesserungen, sei es von Funktionen, Layout oder Usability, gleich mit zu implementieren. Dafür sollten alle relevanten Akteure (hier die Verantwortlichen der Internetauftritte der verschiedenen Bereiche) in das Vorhaben einbezogen werden und die Möglichkeit haben Vorschläge und Wüsche zu äußeren und über diese zu diskutieren.

Das eigentliche Change Management richtet sich in diesem Fall an die IT-Mitarbeiter welche die neuen Systeme aufsetzen und pflegen. Aber auch hier werden die Betroffen nicht vor neue Aufgaben, Prozesse oder Arbeitsweisen gestellt, da die Migration neuer Systeme im Aufgabenfeld eines IT-Mitarbeiters verankert ist.

\paragraph{Alfresco}
Das Change Management für die Umstellung auf das Dateimanagement Alfresco ist dabei etwas Umfangreicher als bei der Erstellung eines responsive Designs für die Webpräsenz der Hochschule. 

Hier handelt sich um ein grundlegend neues System an der Hochschule. 
\ldots
\section{Migrationskonzepte - MB}
\label{section_migrationskonzepte}
Die Ziele einer Migration sind in der Regel betriebswirtschaftlicher oder strategischer Natur. Im Rahmen des hier untersuchten Rahmengebietes einer kleinen Hochschule ist die Migration hin zu einem ganzheitlichen Informationsmanagement eine strategische Entscheidung. Diese Entscheidung beinhaltet einen verbesserten Anwendernutzen, eine Erweiterung des Funktionsumfanges, bessere Integration und Verzahnung verschiedener Softwaresysteme sowie möglichst einer Erhöhung der Produktivität bei möglichst verringerten Kosten. Zur Erstellung des Migrationskonzeptes bedarf es der Betrachtung der Kriterien für eine erfolgreiche Migration und der möglichen Migrationsstrategien.

\subsection{Kriterien für eine erfolgreiche Migration}
Im Rahmen der Migrationsplanung werden die verschiedenen Phasen der Migration geplant. Im Rahmen der Betrachtung einer kleinen Hochschule wurden in der gesamten Ausarbeitung beispielsweise die Ist-Analyse vorgenommen und eine Soll-Konzeption erstellt.

\begin{figure}[h!]
	\centering
	\includegraphics[width=\textwidth]
	{kapitel/gruppe4_1/bilder/vorgehensmodell_softwaremigration}
	\caption{Vorgehensmodell für Software-Migrationen nach \cite{migrationsleitfaden_2012}}
	\label{fig_vorgehensmodell_softwaremigration}	
\end{figure}

Das in Abbildung \ref{fig_vorgehensmodell_softwaremigration} ersichtliche Vorgehensmodell beschreibt die verschiedenen, notwendigen Phasen, die einer Migration vorausgehen. Die Genauigkeit dieser Planung ist hierbei maßgeblich für den späteren Erfolg der Migration. Im Rahmen dieser Ausarbeitung wurde beispielsweise die in der Abbildung ersichtliche Methodik des Experteninterviews (vgl. Kapitel 5.2) angewandt, um Grundlagen für die Ist-Analyse zu erhalten.

In der Auswahlphase sind hierbei strategische, rechtliche und wirtschaftliche Aspekte zu berücksichtigen, ebenso wie der spätere Systembetrieb, die notwendigen organisatorischen Aspekte und Anforderungen an die Sicherheit der Systeme (vgl. Kapitel 5.4.3). Nach der Entscheidungsempfehlung in Kapitel 6.4 werden dann eine oder mehrere Migrationsstrategien für die Einführung der neuen und die Ablösung der alten Software festgelegt.

\subsection{Migrationsstrategien}
Die Wahl der Migrationsstrategie ist jeweils fallbezogen zu prüfen. Es ist auch denkbar, für verschiedene Systeme verschiedene Strategien zu nutzen. Nachfolgend werden auszugweise durch Prof. Dr. Markus Nüttgens\footcite{nuettgens_abloesung_2014} beschriebene Migrationsstrategien aufgeführt, welche in Abschnitt \ref{subsection_migrationsbeispiele} hinsichtlich der Verwendung durch die Migrationsbeispiele der Hochschule beleuchtet werden.

\begin{itemize}
	\item \textbf{Big Bang Approach (Cold Turkey Strategy):} Hierbei wird das Altsystem von Grund auf neu entwickelt und zu einem bestimmten Zeitpunkt zur Verfügung gestellt.	
	
	\item \textbf{Database First Approach / Database Last Approach:} Bei dieser Strategie wird erst das Datenbankmanagementsystem (Database First) migriert und anschließend alle Applikationen und Schnittstellen in ein neues System überführt. Database Last beschreibt hierbei den genau umgekehrten Vorgang.
	
	\item \textbf{Composite Database Approach:} Das neue Anwendungssystem wird schrittweise implementiert, während das Altsystem noch in Betrieb ist.
	
	\item \textbf{Chicken-Little Strategy:} Als Erweiterung des Composite Datebase Approach werden im Rahmen dieser Strategie zusätzliche Gateways entwickelt, welche für die Überführung der Daten aus dem Altsystem in das Zielsystem verantwortlich zeichnen.
	
	\item \textbf{Butterfly Methodology:} Hierbei geht es um eine reine Datenmigration, bei der eine Kooperation zwischen Alt- und Neusystem nicht notwendig ist.  Die Entwicklung des neuen Systems wird also von der Migration der Daten separiert.
\end{itemize}


\subsection{Migrationsbeispiele}
\label{subsection_migrationsbeispiele}

Die Hochschule Emden/Leer nutzt derzeit für Ihren Internetauftritt das Enterprise Content Management System TYPO3 in der Version 4.5. Die Dokumentenverwaltungssoftware Alfresco wird derzeit noch nicht genutzt. 

Nachfolgend wird exemplarisch eine mögliche Migration von TYPO3 auf eine aktuelle Version inkl. Erstellung eines responsive Designs beleuchtet. Im Rahmen der Kostenersparnis wird nicht von einer kompletten Neuentwicklung ausgegangen, sondern von einer schrittweisen Migration des derzeitigen Systems in eine aktuelle Version. Dies bietet den Vorteil, dass eine aufwändige Datenübernahme hinfällig wird. Ferner wird die Neueinführung von Alfresco als zentraler Bestandteil für ein Dokumentenmanagement untersucht. Da die aktuelle Version von Alfresco auch die Möglichkeit bietet Web Content zu verwalten, wäre theoretisch auch eine Migration des derzeitigen Internetauftritts in ein neu eingeführtes Alfresco-System denkbar.

\subsubsection{Responsive Website mit TYPO3}
TYPO3\footcite{typo3_overview_url} ist ein Open Source Enterprise Content Management System (ECMS oder kurz CMS) zur Verwaltung webbasierter Inhalte. Es ist multilingual, hoch skalierbar und bietet ein aktives Sicherheitsmanagement.

\paragraph{Ist-Zustand}\mbox{}\\\\
Die Hochschule Emden/Leer nutzt derzeit ein TYPO3-Sytem in der Version 4.5 LTS (Long Term Support). Das System ist derzeit noch nicht für die Anforderungen mobiler Endgeräte (responsive Design) gerüstet. Es werden verschiedene Extensions von TYPO3 genutzt, möglicherweise auch eigens für die Hochschule entwickelte Extensions. Mitarbeiter und Studenten sind als Benutzer innerhalb des CMS angelegt und können sich in einen geschützten Bereich über die Extension FE-Login anmelden.

Für die derzeit eingesetzte Version von TYPO3 gibt es keinen Support mehr, so dass – weder für den TYPO3-Kern, noch für die Extensions – neue Sicherheitspatches zur Verfügung gestellt werden. Dies stellt ein potentielles Sicherheitsrisiko für die Hochschule dar. Allein aus diesem Grund sollte eine Migration auf ein aktuelles System erwogen werden. Ferner nutzt ein Großteil der Besucher mobile Endgeräte, die aktuell nicht unterstützt werden.

\paragraph{Soll-Zustand}\mbox{}\\\\
Ein neues System sollte über Merkmale verfügen, die sowohl dem aktuellen Stand der Technik, als auch den Anforderungen an das Informationsmanagement genügen. Hierbei ist es notwendig, darauf zu achten, dass das neue System möglichst langen Support seitens der TYPO3 Association aufweist. Dadurch ist es möglich im Rahmen der Supportzeit Sicherheitsupdates zu erhalten. 

Um den die vermehrte Nutzung von mobilen Endgeräten seitens der Benutzer abzudecken soll das neue System eine Auslieferung des Contents für mobile Endgeräte unterstützen. 

Bisher genutzte Extensions sollten – falls technisch realisierbar – erhalten bleiben, ansonsten ist das Vorhandensein von Alternativen zu prüfen. 

Um auch Benutzern mit Handicap die Nutzung des Internetauftritts zu ermöglichen ist es sinnvoll Barrierefreiheit zu implementieren. 

Im Rahmen des Informationsmanagements stellt der Internetauftritt die Außenwirkung der Hochschule dar und transportiert Information zu Benutzern und Interessenten. Eine Auffindung dieser Information bereits über Suchmaschinenanfragen kann einen wirtschaftlichen Vorteil durch Gewinnung neuer Interessenten nach sich ziehen. Die Optimierung des neuen Internetauftritts für Suchmaschinen (SEO - Search Engine Optimization) ist deshalb von Vorteil. Ferner ist eine Anbindung an die Benutzerverwaltung (Single Sign On) für einen einfachen Informationsaustausch aus Benutzersicht hilfreich. Im Interview mit dem Rechenzentrumsleiter der Hochschule Emden/Leer, Herrn Günter Müller (vgl. Kapitel 5.2), bestätigte dieser, dass Single Sign On bereits für den derzeitigen Internetauftritt realisiert ist. 

Um eine Migration durchführen zu können, wird zunächst ein Migrationsplan erstellt.

\paragraph{Migrationsplan}\mbox{}\\\\
Um einen möglichst langen Supportzeitraum zu gewährleisten ist die Verwendung einer LTS-Version (Long Term Support) anzuraten. Die derzeit aktuelle Version ist 6.2.13 LTS (Stand 10.06.2015), welche noch bis Ende März 2017 supportet wird.

Derzeit ist bereits die Version 7.2.0 verfügbar, allerdings noch nicht als LTS-Version. Diese ist für Herbst 2015 avisiert. 

Da die Migration einige Zeit in Anspruch nehmen wird, ist es sinnvoll, direkt auf die Version 7.x LTS zu migrieren, da diese dann verfügbar sein wird. Hierfür sind allerdings Zwischenschritte vorzusehen, da eine direkte Migration von Version 4.5 auf 7.x nicht möglich ist.\footcite{typo3_upgrade_url} Es muss zunächst eine Migration auf die Version 6.2 LTS und von dort auf die Version 7.x erfolgen. Nachfolgend wird somit von einer Migration auf die Version 7.x LTS ausgegangen.

Vor der Migration ist eine Überprüfung aller derzeit genutzten Extensions erforderlich. Dabei muss geprüft werden, ob diese in der neuen Version noch gültig und lauffähig sind. Ist dies nicht der Fall, müssen Alternativen gesucht werden und deren Realisierung in die Planung einfließen. Insbesondere selbst geschriebene Extensions müssen hinsichtlich der Lauffähigkeit überprüft und ggf. ein Konzept zur Anpassung erstellt werden.

\subparagraph{Hardwareanforderungen}\mbox{}\\\\
Für eine erfolgreiche Migration sind bestimmte Hardwareanforderungen Voraussetzung. Unter anderem muss mindestens PHP 5.5, MySQL 5.5 und mehr als 200 MB freier Plattenplatz zur Verfügung stehen. Die genauen Konfigurationseinstellungen inkl. allen benötigten Module sind den Installationsvorgaben\footcite{typo3_installing_url} der TYPO3 Association zu entnehmen.

\subparagraph{Entwicklungssystem}\mbox{}\\\\
Zur Realisierung des neuen Systems wird ein Entwicklungssystem mit den oben beschriebenen Hardwareanforderungen aufgesetzt. Über einen Dump der Datenbank werden die Daten des Produkivsystems in die Datenbank des Entwicklungssystems übertragen. Das gesamte Dateisystem des TYPO3-Produktivsystems wird ebenfalls auf das Entwicklungssystem übertragen. Dort werden dann die Konfigurationseinstellungen von TYPO3 angepasst, damit ein identisches, lauffähiges System entsteht.

Innerhalb dieses Systems erfolgt die Migration auf die verschiedenen Versionen, die Anpassung der Extensions und die im Rahmen der Migration notwendige Softwareentwicklung.

\subparagraph{Migration}\mbox{}\\\\
Im Rahmen der Migration müssen Softwaretechnisch folgende Punkte berücksichtigt werden:
\begin{itemize}
\item Migration des TYPO3-Kerns
\item Migration aller eingesetzten Extensions
\item Anpassung selbstgeschriebener Extensions
\item Umstellung des Layout-Konzeptes von TYPO3 (von derzeit wahrscheinlich Templa-Voilà) auf Fluid-Templating
\item Schaffung einer Basis für responsive Design, beispielsweise auf Basis des Frameworks Bootstrap
\item Erweiterung / Anpassung der TypoScript-Programmierung
\item Anpassung Menüprogrammierung (TypoScript und Template)
\item Neuerstellung benötigter Fluid-Templates auf Basis von Haupttemplates und Partials
\item Programmierung eigener Extensions, falls notwendig
\end{itemize}

Der Datenbestand wird nach der Migration noch einmal mit dem Datenbestand des derzeitigen Systems abgeglichen. Alternativ ist auch eine Übernahme neuer Daten während der Migrationsphase, beispielsweise durch Gateways denkbar. 

\subparagraph{Produktivsetzung}\mbox{}\\\\
Die Ablösung des derzeitigen Systems erfolgt anhand der Migrationsstrategie Big Bang Approach (oder der Chicken-Little Strategy, falls die im vorigen Kapitel angesprochene Alternative mit Gateways genutzt wird), da mit dem Entwicklungssystem ein fertig entwickeltes und hinsichtlich des Datenbestandes aktuelles System zur Verfügung steht. Die Produktivsetzung erfolgt in umgekehrter Reihenfolge wie die Einrichtung des Entwicklungssystems, also mit Datenbank-Dump, Datei-Migration und ggf. TYPO3-Konfigurations-anpassungen. Hierdurch ist die Downtime für den Internetauftritt der Hochschule Emden minimal.

\subsubsection{Alfresco}
\label{subsubsection_migration_alfresco}
Alfresco\footcite{alfresco_dm_url} ist ein Dokumenten-Management-System welches als Open-Source-Plattform offene Standards unterstützt. Hiermit lässt sich der gesamte Content auf einer einzelnen Plattform konsolidieren und damit die Benutzerfreundlichkeit erhöhen und die Kosten senken.

Luis Cabaceira\footcite{cabaceira_alfresco_2015} hat die nachfolgende Grafik erstellt, die eine Übersicht über die Funktionen von Alfresco gibt:

\begin{figure}[h!]
	\centering
	\includegraphics[width=\textwidth]
	{kapitel/gruppe4_1/bilder/uebersicht_alfresco}
	\caption{Übersicht Alfresco nach Luis Cabaceira}
	\label{fig_uebersicht_alfresco}
\end{figure}

Peter Franke – Leiter des Rechenzentrums der Hochschule Braunschweig/Wolfenbüttel – berichtet von positiven Erfahrungen seit der Einführung von Alfresco.\footcite{franke_alfresco_2011}

\paragraph{Ist-Zustand}\mbox{}\\\\
Alfresco wird derzeit von der Hochschule Emden/Leer noch nicht eingesetzt. Derzeit werden Dokumente in verschiedensten Systemen verwaltet und zugänglich gemacht. Auszugsweise sind hier zu nennen:

\begin{itemize}
	\item Austauschlaufwerke für Dozenten
	\item Webseiten mit offenen und geschlossenen Bereichen (Kennzahlen, Daten, Fakten für Mitarbeiter und Dekane)
	\item Eigene Software Vorlesungsverzeichnis im Fachbereich Seefahrt
	\item Software EvaSys für die Evaluierung
	\item Gigamove zum Austausch große Datenmengen (vgl. Kapitel 5.6.3.2)
	\item Eigene Systeme in den Fachbereichen (Labor)	
\end{itemize}

Derzeit gibt es also viele gewachsene Systeme und Strukturen.

\paragraph{Soll-Zustand}\mbox{}\\\\
Ein neues System soll Information bündeln und zentral verwalten. Hierfür werden alle vorhandenen Dokumente in das neue System migriert, unabhängig vom Datentyp. Ein Versionsmanagement sorgt für die Versionierung der Dokumente, mit dem Vorteil, dass auch auf ältere Versionen zugegriffen werden kann. Ein schneller und ortsunabhängiger Zugriff auf die Information ist für die Usability des Systems wichtig und bedingt unter anderem, dass keine Client-Installation notwendig wird.

Die Software Alfresco bietet alle diese Merkmale. In der aktuellen Version wird auch die Auslieferung von Web-Content unterstützt, so dass für die Zukunft auch eine Migration des Internetauftritts in das Alfresco-System denkbar wäre. Alternativ könnte Alfresco auch im Rahmen des Single Source Publishing Konzeptes als Content-Quelle für das TYPO3-System genutzt werden. Die Berliner Philharmoniker nutzen bereits dieses Konzept, wie aus einer Case Study der Firma form4 GmbH hervorgeht.\footcite{form4_alfresco_2015}

Hinsichtlich des Dokumenten-Managements wird zunächst ein strategisch günstiger Migrationsplan zur Einführung von Alfresco erstellt. Dabei muss auch die Entscheidung getroffen werden, welche Edition von Alfresco sinnvoll für die Hochschule ist.

\paragraph{Migrationsplan}\mbox{}\\\\
Da nach der Migration alle Dokumente zentral verwaltet werden, erscheint es sinnvoll, Alfresco als hochverfügbaren Cluster auszulegen. Gegebenenfalls ist auch der Einsatz eines SAN (Storage Area Networks) mit räumlich getrennten Speichereinheiten und entsprechend angepasstem Backup- und Restore-Konzept in Erwägung zu ziehen.

Grundsätzlich stehen von Alfresco die kostenlose Community Edition und die kostenpflichtige Enterprise Edition zur Verfügung. Ein Vergleich der beiden Editionen findet sich auf der Website\footcite{alfresco_community_edition_2015} von Alfresco.

Folgt die IT-Leitung der Hochschule Emden/Leer dem Vorschlag einer hochverfügbaren Realisierung, muss die Enterprise Edition eingesetzt werden, da nur sie die Möglichkeit des Clusterings bietet. Hierbei ergibt sich der Vorteil, dass für diese Edition Support seitens des Herstellers geboten wird und die wichtige Frage nach Service Level Agreements (SLA) damit gelöst werden kann. Zusätzlich gibt es Zertifizierungsschulungen für Entwickler und Administratoren, welche im Rahmen des Change Managements sinnvoll sind.

Das Alfresco-System wird komplett neu aufgesetzt und die (derzeit) auf verschiedenen Systemen verteilten Dokumente werden nach und nach in das Alfresco-System migriert.

\subparagraph{Hardwareanforderungen}\mbox{}\\\\
Die Hardwareanforderungen richten sich stark nach den in Alfresco genutzten Modulen, bzw. ob die Community oder Enterprise-Edition genutzt wird. Detailliert Hardwareanforderungen können nach Festlegung der Edition in der Alfresco-Dokumentation\footcite{alfresco_documentation_2015} eingesehen werden. 

Die Hardwareanforderungen sind unter anderem abhängig von:

\begin{itemize}
	\item dem benötigten Anwendungsfall (welche Komponenten genutzt werden)
	\item der Anzahl der gleichzeitig zugreifenden Benutzer
	\item dem Speicherort der Dokumente
	\item dem Betrieb im Rahmen einer Hochverfügbarkeitslösung
	\item dem Einsatz von Load Balancern
	\item dem Einsatz dedizierter Transformation Server
	\item der Nutzung von Clustern für zu nutzende Interfaces
	\item dem Einsatz von Caching-Verfahren
\end{itemize}

\begin{figure}[h!]
	\centering
	\includegraphics[width=\textwidth]
	{kapitel/gruppe4_1/bilder/deployment_diagramm_alfresco}
	\caption{Mögliche Hardwarestruktur für Alfresco nach Cabaceira, 2014}
	\label{fig_deployment_alfresco}
\end{figure}

Die Abbildung \ref{fig_deployment_alfresco} zeigt eine mögliche Hardwarestruktur für ein Alfresco-System nach Cabaceira.\footcite{cabaceira_alfresco_2015}

\subparagraph{Entwicklungssystem}\mbox{}\\\\
Das Entwicklungssystem wird nach den benötigten Hardwareanforderungen aufgesetzt. Hierauf erfolgt die Implementierung von Alfresco inkl. den bei Bedarf benötigten Schnittstellen. Nach Fertigstellung kann das Entwicklungssystems direkt als Produktivsystem genutzt werden, da die Datenmigration anschließend erfolgt, wie im folgenden Abschnitt beschrieben.

\subparagraph{Migration}\mbox{}\\\\
Für die Migration bietet sich in diesem Fall die Migrationsstrategie Butterfly Methodology an, da hierbei nach und nach die verschiedenen Altsysteme in das neue System überführt werden können. Da es sich beim Entwicklungssystem um ein "leeres" System handelt, wird es nach Fertigstellung als Produktivsystem genutzt. Hierin erfolgt dann nach und nach die Migration der Dokumente aus den unterschiedlichen Altsystemen.

\subparagraph{Produktivsetzung}\mbox{}\\\\
Wie bereits oben beschrieben, erfolgt die Produktivsetzung direkt nach Abnahme des Entwicklungssystems und erfolgt durch dessen Übernahme.




\chapter{Kosten und Zeit}

Autoren: Benedikt Buchner, Sebastian Hanna, Klaus Landsdorf

Ich bin ein einleitender Satz, der noch geschrieben werden muss.

% !TEX root = ../../../main.tex
% !TEX encoding = UTF-8 Unicode
% !TEX encoding = UTF-8

\section{Kostenarten - KL}
% \textit{Autor: Klaus Landsdorf}

\label{section_kostenarten}
Kalkulationen benötigen bestimmte eindeutige Kostenarten, die in einem Kostenartenplan aufgestellt und in einer Kostenartenrechnung kontrolliert werden können. Die eindeutigen Kostenarten können in Kostenartenkategorien bzw. Kostenartengruppen zusammenfließen. Im folgenden soll festgehalten werden, was man für eine Kalkulation und deren Kontrolle w\"ahrend der Umsetzung bedenken sollte.

\begin{figure}[h!]
	\centering
	\includegraphics[width=\textwidth]{kapitel/gruppe4_2/bilder/beispiel_kostenarten_TCO}
	\caption{Beispiel von Kostenarten in der TCO-Methode}
	\label{fig_kostenarten_TCO}
\end{figure}

Die Kostenarten der Abbildung \ref{fig_kostenarten_TCO} nach Hansen\footnote{\autocite[495]{hansen_business_2009}, \autocite[Vgl.][314, 355]{muller2013betriebswirtschaftslehre}} finden sich in der von Krcmar benannten TCO-Methode - \enquote{Total Cost of Ownership} - wieder. Aus den bewerteten Daten der Kostenarten können periodische Durchschnittswerte ermittelt werden, aus denen dann für die Zukunft neue Abschätzungen gewonnen werden.

\clearpage

In einer ABC-Analyse kann eine weitere Klassifizierung vorgenommen werden, um aufzuzeigen welche Kostenarten auf jeden Fall (A-Klasse) anfallen, welche im besten Fall noch erledigt werden sollen (B-Klasse) und welche man optional (C-Klasse) aufwenden sollte.

Die Kostenarten in der Tabelle \ref{tab_gliederung_kostenarten} sind die erweiterten Grundelemente der Abbildung \ref{fig_kostenarten_TCO} zur Wertsteigerung durch Wertschöpfung\footnote{\autocite[125-128]{reim_erfolgsrechnung_2015}}, die laut Reim in die Kostenartenrechnung fließen sollten. Die Tabelle \ref{tab_gliederung_kostenarten} soll, m\"oglichen Projekten des integrierten Informationsmanagements der Hochschule Emden/Leer, als wiederverwendbare Übersicht dienen, damit in den notwendigen Kalkulationen alle Kosten erfasst werden. Wozu die dient hat Reim in einem kurzen Satz zusammengefasst: „Die Kostenartenrechnung erfasst, systematisiert und periodisiert die Kosten.“\footnote{\autocite[137-147]{reim_erfolgsrechnung_2015}} 

\begin{table}[h!]
\begin{tabularx}{\textwidth}{|X|X|}
	\hline \textbf{Gliederungsmerkmale nach} & \textbf{Kostenartengruppen} \\ 
	\hline der Art der verbrauchten Einsatzgüter & \begin{itemize}
		\item Materialkosten
		\item Personalkosten
		\item Fremdleistungskosten
		\item (Kalkulatorische) Abschreibungen
		\item (Kalkulatorische) Kapitalkosten
		\item Kalkulatorische Zusatzkosten
		\item Kostensteuern und Gebühren
	\end{itemize} \\ 
	\hline der Beschäftigungsabhängigkeit & \begin{itemize}
		\item Variable Kosten
		\item Fixe Kosten
	\end{itemize} \\ 
	\hline der Art der Verrechnung & \begin{itemize}
		\item Einzelkosten
		\item Gemeinkosten
		\item Sondereinzelkosten
	\end{itemize} \\ 
	\hline 
\end{tabularx}
	\caption{Auszug der Kostenarten zur Wertsteigerung nach Reim}
	\label{tab_gliederung_kostenarten}
\end{table}

Ein systematischer und periodischer Blick auf die Kosten, wird f\"ur die Planung und sp\"atere Kontrolle der ben\"otigten Mittel eine passende Hilfe f\"ur den CIO sein.

\input{kapitel/gruppe4_2/sections/subsection1_1}
\input{kapitel/gruppe4_2/sections/subsection1_2}








	\chapter{Zusammenfassung - AW}
\textit{Autor: Andreas Willems}

\ldots
	\printbibliography[heading=bibintoc, title=Literatur- und Quellenverzeichnis]

%\addcontentsline{toc}{chapter}{Listings}
\listoftables
\listoffigures
%\lstlistoflistings
	\appendix
\end{document}