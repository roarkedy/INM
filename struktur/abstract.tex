\begin{abstract}
	\paragraph*{Abstract}
	$\;$ \\
	$\;$ \\
	\textit{Autor: Andreas Willems}
	$\;$ \\
	$\;$ \\
	% Ziel der Arbeit
	Ziel dieses Gutachtens ist die Untersuchung des Informationsmanagements an der Hochschule Emden/Leer
	im Hinblick auf eine potenzielle Neuordnung.
	
	% Wiedergabe der Struktur
	% Grundlagen 1.1
	Der Grundlagenteil dieser Arbeit befasst sich zunächst mit dem gegenwärtigen Stand der Literatur in Bezug auf das Informationsmanagement
	speziell im Hinblick auf Hochschulen und betrachtet im Weiteren Trends des Informationsmanagements an Hochschulen und Ergebnisse der
	Einführung eines Informationsmanagements an anderen Hochschulen.
	
	% Grundlagen 1.2
	% Grundlagen 1.3
	% Ist-Zustand und Soll-Konzept
	Im weiteren Verlauf der Arbeit wird der gegenwärtige Stand des Informationsmanagements an der Hochschule Emden/Leer analysiert und
	daraus ein Konzept zur Neuordnung entwickelt.
	% Überführung und Kosten/Zeit
	Dieses Konzept wird zuletzt auf seine Umsetzbarkeit hin untersucht und Schätzungen hinsichtlich zeitlichem und finanziellem Aufwand 
	angestellt.
	
	% Wiedergabe der Ergebnisse
    % Ergebnis Gruppe 2
	Die Analyse der gegenwärtigen Situation kommt zu dem Ergebnis, dass die Hochschule Emden/Leer über kein
	Informationsmanagement verfügt.
	
	% Ergebnisse Gruppe 3
	Das entwickelte Soll-Konzept setzt Schwerpunkte in der Erweiterung von externen und internen Marketingmaßnahmen,
	in der Einführung einer zentralen Logdatei für Supportmaßnahmen sowie daraus abzuleitenden Feedbackmaßnahmen zur
	Prozessverbesserung und in der Optimierung von Hard- und Software mit dem immer wiederkehrenden Aspekt eines Single-Sign-Ons.
	
	% Ergebnisse Gruppe 4.1
	Die nachfolgende Umsetzungsplanung konzentriert sich vornehmlich auf das Change-Management zur Begleitung der vorgeschlagenen 
	Veränderungen und führt zwei Migrationsbeispiele aus. Das erste Beispiel beschreibt die Einführung einer responsiven Webseite mit 
	zugrundeliegendem Content-Management-System TYPO3. Das zweite Beispiel führt die Einführung des Dokumentenmanagementsystems
	Alfresco aus.
	
	% Ergebnisse Gruppe 4.2
	Die zum Schluss durchgeführte Kosten- und Zeitschätzung kommt zu dem Ergebnis, dass die Einführung des Dokumentenmanagementsystems
	zu Kosten von ca. 210.000 EUR über vier Jahre führen wird.
	Das Redesign und der Relaunch der Homepage der Hochschule werden mit mindestens 50 Manntagen beziffert,
	die Erstellung einer Facebook-Seite mit 21,5 Manntagen.
	
	
	
	
	
\end{abstract}